\documentclass[12pt]{article}
\textwidth 7.0in
\oddsidemargin -0.4in
\evensidemargin -0.4in
\textheight 9.0in 
\pagestyle{empty}
\usepackage{enumerate}
\begin{document}
% 8.5in paper width -2x1in margin = 6.5in text width
\setlength{\topmargin}{-2mm}



% 11in paper height -2x1in margin = 9in text height


\begin{center}{\bf Homework 0 - Math 531, Frank Thorne (thornef@mailbox.sc.edu)}
\end{center}
\begin{center}
{\bf Due Wednesday, August 29}
\end{center}

This homework is {\bf optional}. If you do hand it in, it will be graded {\itshape extremely}
leniently (which the rest of the homeworks won't be).

This homework involves {\itshape writing proofs}. This is an essential skill for Math 531.
If you find this homework difficult, please be warned that Math 531 might be an uphill climb,
and seek help from me and others if you need it.

With all proofs, please explain yourself clearly and in complete English sentences.
\\
\\
\begin{enumerate}[(1)]
\item 
Prove that a triangle cannot have two obtuse angles. ({\itshape Obtuse} means greater than 90 degrees.)

\item
Prove that the sum of two even integers is even, that the sum of two odd integers is even, and
that the sum of an even and an odd integer is odd.

\item
Prove that $1 + 3 + 5 + \cdots + (2n - 1) = n^2$. Here is one idea for a nice proof: Draw a big
square and divide it into $n$ rows and $n$ columns (i.e. into $n^2$ small squares). There is one
square in the top left, there are three touching it, and so on.

\item
Prove that the area of a unit circle is somewhere between 1 and 4. (The answer is $\pi$, but you are
not allowed to quote this fact without proving it.)

\item 
You draw three lines through a circle, so that each line intersects the circle twice. In how many regions
might this divide the circle? Determine the complete list of possibilities, with proof.

\item
Prove that there is no positive rational number (i.e., fraction) which is smaller than all other positive rational
numbers.

\end{enumerate}

\end{document}
