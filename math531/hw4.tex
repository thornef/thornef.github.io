\documentclass[12pt]{article}
\textwidth 7.0in
\oddsidemargin -0.4in
\evensidemargin -0.4in
\textheight 9.0in 
\pagestyle{empty}
\usepackage{enumerate}
\begin{document}
% 8.5in paper width -2x1in margin = 6.5in text width
\setlength{\topmargin}{-2mm}



% 11in paper height -2x1in margin = 9in text height


\begin{center}{\bf Homework 4 - Math 531, Frank Thorne (thornef@mailbox.sc.edu)}
\end{center}
\begin{center}
{\bf Due Friday, October 12}
\end{center}

Note: For proofs in this homework, write them roughly in the style of Euclid.
You do not have to follow his style exactly, and feel free to describe triangles as being
congruent (which he did not), but please do not depart too much from Euclid.

Also, don't appeal to unproved propositions beyond where we are in the text.
\\
\\
\begin{enumerate}[(1)]
\item 
The proof of Proposition 5 appeals to eight previous facts (see the margins: I.Def.20, I.Post 2, etc.)
For each of these, explicitly explain how this fact is being used.

\item
In Proposition 2, there are two possibilities for how an equilateral triangle $DAB$ might be constructed on $AB$.
Draw two pictures with $A$, $B$, $C$ in roughly the same spots in each, and draw the remaining picture both ways, 
with $D$ constructed on either side of $AB$. Explain the difference, if any, in the resulting proof.

\item
Write a proof of Proposition 3 which does not use Proposition 2. (Hint: You will end up essentially duplicating
the proof of Proposition 2 in your proof.)

\item
In the commentary, Joyce says: ``Frequently, though, one end of the line $C$ is already placed at $A$, and then the
construction of $I.2$ isn't required. In that case, only one circle needs to be drawn.

Explain.
\item
Proposition 6 is the same as Problem 1B.1 in Isaacs. How might you have solved Problem 1B.1 in a different way?
Why would this proof be inappropriate here in Euclid?

\item
Prove Proposition 7 in case $D$ lies inside triangle $ABC$. (See the criteria.) Also, describe what other special
cases can occur and prove the proposition in these cases as well.

\end{enumerate}

\end{document}
