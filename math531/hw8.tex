\documentclass[12pt]{article}
\textwidth 7.0in
\oddsidemargin -0.4in
\evensidemargin -0.4in
\textheight 9.0in 
\pagestyle{empty}
\usepackage{enumerate}
\begin{document}
% 8.5in paper width -2x1in margin = 6.5in text width
\setlength{\topmargin}{-2mm}



% 11in paper height -2x1in margin = 9in text height


\begin{center}{\bf Homework 8 - Math 531, Frank Thorne (thornef@mailbox.sc.edu)}
\end{center}
\begin{center}
{\bf Due Friday, December 7}
\end{center}
\\
\\
For the first three problems, draw four equally spaced points $A$, $B$, $C$, and $D$, on a line, in that order.
\begin{enumerate}[(1)]
\item
Compute the cross ratio $cr(A, B, C, D)$.
\item
Compute the cross ratios $cr(B, A, C, D)$ and $(D, C, B, A)$.
\item
There are twenty-four different ways to order the points $A$, $B$, $C$, and $D$. Determine how many different values of the cross ratio you can get, and compute all of the possible values.
\item
Draw configurations of four collinear points for which the cross ratio $cr(A, B, C, D)$ is (a) very large, and (b) very small. 
\item
Draw (using a ruler) three points $A$, $B$, $C$ on a line with $AB = 3$, $BC = 2$. Use cross ratios to compute points
$D$, $E$, $F$, $G$ so that these seven points all appear evenly spaced in perspective. In addition, compute the distance to
the ``point at infinity'' (i.e., the limit of infinitely many such points) and indicate it on your diagram.
\item
Draw a 3-d perspective drawing of a tunnel, road, railroad, or other such diagram. You may reproduce the diagram done in class
(with the distances actually measured properly) or do something else if you prefer. Your drawing should include at least one
instance of repeated lengths which appear to be the same when viewed from perspective.

Provide the details of any computations, and explain any steps in your drawing that aren't routine.

({\bf Bonus.} Do something especially cool.)
\item
Isaacs, Ch. 3C. Do the second of the exercises labeled 3C.2 (in my copy of the book, the problem numbers are screwed up), 3C.8.
{\bf Bonus.} 3C.4, 3C.10.
\item
Isaacs, 4A.1, 4A.4. {\bf Bonus:} 4A.6.


\end{enumerate}

\end{document}
