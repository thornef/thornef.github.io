\documentclass[12pt]{article}
\textwidth 7.0in
\oddsidemargin -0.4in
\evensidemargin -0.4in
\textheight 9.0in 
\pagestyle{empty}
\usepackage{enumerate}
\begin{document}
% 8.5in paper width -2x1in margin = 6.5in text width
\setlength{\topmargin}{-2mm}



% 11in paper height -2x1in margin = 9in text height


\begin{center}{\bf Homework 5 - Math 531, Frank Thorne (thornef@mailbox.sc.edu)}
\end{center}
\begin{center}
{\bf Due Wednesday, October 17}
\end{center}
({\bf subject to additions})

Homeworks will be accepted on October 22 if you prefer.

Note: For proofs in this homework, write them roughly in the style of Euclid.
You do not have to follow his style exactly, and feel free to describe triangles as being
congruent (which he did not), but please do not depart too much from Euclid.

Also, don't appeal to unproved propositions beyond where we are in the text.
\\
\\
\begin{enumerate}[(1)]
\item
Euclid's proof of Proposition 13 does not cover all possible cases. Determine what case(s) are not covered,
and write a proof covering these case(s).

\item
Read the second part of the proof of Proposition 26, and rewrite the proof using Euclid's logic but in a contemporary style.

\item
Euclid proves in Proposition 32 (finally!) that the sum of the angles in a triangle is 180 degrees (``equal to two right angles'').
Extend the proof of Proposition 16 to cover this case. You will need some facts about parallel lines. Find these
facts in Euclid's propositions (they're in the twenties), state which propositions you need, and quote them in your proof.

In addition, read the proof of Proposition 32 and compare with your proof.

\item (Bonus.)
Euclid does not allow straight angles (i.e. the angle made by a line is 180 degrees).
What if he did? Suggest a modified version of appropriate parts of the Elements
that would allow Euclid to talk about straight angles. Your answer should include revised or additional
definitions and/or postulates. Then use these make simplifications in at least two proofs (Propositions 13, 14, and 15 
are a good place to look).

\end{enumerate}

\end{document}
