\documentclass[12pt]{article}
\textwidth 7.0in
\oddsidemargin -0.4in
\evensidemargin -0.4in
\textheight 9.0in 
\pagestyle{empty}
\usepackage{enumerate}
\begin{document}
% 8.5in paper width -2x1in margin = 6.5in text width
\setlength{\topmargin}{-2mm}



% 11in paper height -2x1in margin = 9in text height


\begin{center}{\bf Final Examination -- Math 531}
\end{center}
\begin{center}
{\bf Due Saturday, December 15}
\end{center}

\begin{enumerate}[(1)]
\item
Prove that a quadrilateral is a parallelogram if and only if its diagonals bisect each other.
\item
Let $AX$ be the angle bisector of $\angle A$ in $\triangle ABC$. Prove that
\begin{equation}
\frac{BX}{XC} = \frac{AB}{AC}.
\end{equation}

\item
Let $P$ be a point inside a circle, and let $XY$ and $UV$ be two chords going through $P$.
Prove that $PX \cdot PY = PU \cdot PV$.

\item
Given acute angled $\triangle ABC$ as in Figure III.1, extend the altitudes from $A$, $B$, $C$
to meet the circumcircle at $X$, $Y$, and $Z$ respectively. Prove that $AX$ bisects $\angle Z X Y$.

{\itshape Hint.} Look for congruent triangles.
\item
Suppose that $AX$ is a median of a triangle, and $G$ is the centroid. Prove that $AG = 2 GX$.

\item
Prove that the three angle bisectors of a triangle are concurrent.
\item
Let $A, B, C, D$ be four collinear points and suppose that $P$ is a point not on the line
through them. Prove that
$$cr(A, B, C, D) = \frac{ \sin(\angle APC) \sin( \angle BPD)}{ \sin( \angle APD) \sin(\angle BPC)}.$$
\end{enumerate}

\end{document}
