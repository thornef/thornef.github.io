\documentclass[12pt]{article}
\textwidth 7.0in
\oddsidemargin -0.4in
\evensidemargin -0.4in
\textheight 9.0in 
\pagestyle{empty}
\usepackage{enumerate, url}
\begin{document}
% 8.5in paper width -2x1in margin = 6.5in text width
\setlength{\topmargin}{-2mm}



% 11in paper height -2x1in margin = 9in text height


\begin{center}{\bf Study Guide - Math 531, Frank Thorne (thornef@mailbox.sc.edu)}
\end{center}
\begin{center}
{\bf Final Exam, Saturday, December 7}
\end{center}
The final exam will have three parts on it.

The first part will be {\itshape computations}, worth somewhere between 20 and 25 points. You will asked to compute distances,
angles, arc measures, cross ratios, and/or other numerical quantities from diagrams. No proofs will
be required, but for incorrect answers, justifications may be worth partial credit.
\\
\\
The second part will be {\itshape constructions}, worth somewhere between 20 and 25 points. You will be asked to construct
figures using compass and straightedge. This includes figures discussed after the second midterm, including incircles,
circumcircles, the orthocenter, the midpoint, the medial triangle, and the Gergonne point. You may also be asked to construct
simple perspective drawings, in which case you can also use a ruler.

Full proofs will not be required, but you should explain the steps you take. For perspective drawings, you should show your cross ratio computations.
\\
\\
The third part will be {\itshape proofs}, worth somewhere between 50 and 60 points. This will consist of a variety of
propositions to be proved, taken from or similar to the propositions and theorems in the book, and the homework problems
(either assigned homework or not). The level of difficulty will vary.
\\
\\
There will also be {\itshape extra credit}, but it will be extremely difficult and you should attempt the rest of the exam first.
\\
\\
Because there will be several moderately difficult proofs on the exam, {\bf there will be a curve}: A = 86+, B = 72+, C = 58+.
Your final exam grade will be rescaled to the scale given on the website and then averaged in with your other grades. If you solve
any extra credit problems, then the points will be added after your grade is curved.
\\
\\
{\bf List of material and what to study.}
\begin{enumerate}[(1)]
\item 1B. We covered this. You should be very familiar with this and be able to do any of the exercises.
\item 1C. We covered most of this, but we treated the material in (1.5) very lightly, and we did not talk about
non-convex polygons at all. 
\item 1D. We covered this section thoroughly. You should be able to prove all of the statements in the text,
and do all of the exercises.
\item 1E. This is important. I don't believe we discussed (1.14) (perhaps John discussed it while I was away).
Be sure to know the law of sines and (1.12). You should be able to do any of these exercises (although these are a
bit harder). Note also that John proved Heron's formula while I was gone. Never mind the proof of Heron, but it's good
to know what it says.
\item 1F. Note that (1.15) tells you how to find the circumcircle. (Be ready to do this with compass and straightedge.)
This section was long, and is important. You should know it thoroughly. We did not present (1.20) but this is the kind
of problem you now know how to solve. All the exercises are fair game, although some of them would be on the more difficult
end.
\item 1G. Polygons in circles. Know the formula for the angle of a regular $n$-gon. We proved that a polygon formed with
$n$ equally spaced points is regular (all sides and all angles) equal -- this is presented in the text of 1G.

The majority of 1G is interesting material which we didn't discuss. We did, however, discuss how to find the side length
of regular triangles, squares, hexagons, and octagons when inscribed in unit circles, so you should know that.
\item 1H. Similarity. This is important. The proof of Theorem 1.28 is surprisingly subtle, and would be one of the difficult
questions if it were asked. The statement of Lemma 1.29 is extremely important. (1.30) is a typical problem you should know
how to solve, and note that Ceva's theorem would work as well. (1.34) is interesting but difficult. (1.35) is interesting and
important, and the proof is relatively easy. The alternate proof of the Pythagorean Theorem is worth studying. Any of the
exercises would be fair game, some are more difficult than others.

\item {\bf Euclid.} You should be able to prove Propositions 1-20, 22, 23, 26-31, although you don't need to give
Euclid-style proofs and should feel free to appeal to facts Euclid didn't. Note that many of these are redundant with
what's in Isaacs' book.

\item There will not be any questions about Hilbert's axiomatic approach to geometry.

\item 2A. You should know everything here, although we didn't really talk about reflection or the material at the end
of the chapter. (2.3) could be useful in computations.
\item 2B. Know Theorem 2.7 and its proof, and be able to reproduce (2.). You should also know what the medial triangle is.

We did {\itshape not} discuss centers of mass, so 2B.2 through 2B.5 (and most of the material discussed in this section)
are off the table.

\item 2C. Know (2.9) and (2.10) and their proofs. We didn't discuss the nine-point circle (except very briefly).
Exercises 2C.1-2C.3 are on the table.

\item 2D. We dicussed the Law of Cosines (2.17), know the proof and how to use this in simple computations. We skipped
the rest of the chapter.

\item 2E. We skipped (2.29), (2.31), (2.32), (2.33). You should be able to do any of the exercises.

\item We skipped 2F, 2G, 2H, and 3A.

\item 3B. This is too difficult to remember.

\item 3C. We covered this chapter thoroughly. You should know all of it. All of the exercises would be fair game, although
some would be difficult.

In addition, we discussed how cross ratios can be used in perspective drawing, and you should know that too.

\item 4A. Know Ceva's theorem, its proof, and its applications. Also know the definition of the Gergonne point.

We discussed (4.3) implicitly, but did not draw much attention to it. We did not discuss the center of mass material or (4.4).
Be able to do any of the exercises, unless you need (4.4).

We briefly discussed the remainder of Chapter 4, but this won't be on the exam.
\end{enumerate}

\end{document}
