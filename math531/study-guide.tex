\documentclass[12pt]{article}
\textwidth 7.0in
\oddsidemargin -0.4in
\evensidemargin -0.4in
\textheight 9.0in 
\pagestyle{empty}
\usepackage{enumerate, url}
\begin{document}
% 8.5in paper width -2x1in margin = 6.5in text width
\setlength{\topmargin}{-2mm}



% 11in paper height -2x1in margin = 9in text height


\begin{center}{\bf Study Guide - Math 531, Frank Thorne (thornef@mailbox.sc.edu)}
\end{center}
\begin{center}
{\bf Exam: Wednesday, October 31}
\end{center}

{\bf Euclid:} You will be provided with a printout of the webpage
\url{http://aleph0.clarku.edu/~djoyce/java/elements/bookI/bookI.html},
but not any of the linked pages. This lists all of the definitions,
postulates, common notions, and propositions from Euclid.

You will be given five propositions from 1-20, 22, 23, 26-31, and be asked to prove any
three of them.
\\
\\
{\bf Constructions.} Be able to construct any of the following using straightedge
and compass. Explain what you are doing at every step. Be able to prove that
your constructions work (although this won't necessarily be asked.)

Other constructions might appear (anything on HW6 is also fair game), but this covers most of them.
\begin{enumerate}[(1)]
\item
Bisect an angle or a line segment.
\item
Draw the perpendicular bisector of a given line segment.
\item
Given a line $\ell$ and a point $P$ on $\ell$, draw the perpendicular to $\ell$
through $P$.
\item
Same as previous question, where $P$ is not on $\ell$.
\item
Construct a parallel to a line $\ell$ through a point $P$.
\item
Given a triangle, duplicate it elsewhere on your diagram.
\item
Construct an equilateral triangle, a square, and a regular hexagon. (Either given 
a circle to start, in which case the figure should be inscribed, or given 
one of the edges to start.)
\item
Given line segments of length $a$ and $b$, construct segments of length $a + b$,
$a - b$, $\sqrt{ab}$, $n a$ or $a/n$ for any positive integer $n$, or $a/b$.
Iterate this procedure; for example, construct a segment of length $\sqrt{3 + \sqrt{3}}$.
\item
Given all or part of a circle, find its center.
\item
Construct angles of 15, 30, 45, 60, and 75 degrees.
\end{enumerate}

\end{document}
