\documentclass[12pt]{article}
\usepackage{amssymb,amsmath,amsthm}
\usepackage{enumerate, verbatim, url}
\textwidth 7.0in
\oddsidemargin -0.4in
\evensidemargin -0.4in
\textheight 9.0in 
\pagestyle{empty}
\begin{document}
% 8.5in paper width -2x1in margin = 6.5in text width
\setlength{\topmargin}{-2mm}



% 11in paper height -2x1in margin = 9in text height


\begin{center}{\bf The Geometry of Numbers (Spring 2014): Homework 1}
\end{center}
\begin{center}Frank Thorne, thorne@math.sc.edu
\end{center}
\begin{center}
{\bf Due Monday, January 27}
\end{center}

Late homework will be accepted, but more will be posted soon and so it might be more interesting to just
turn in whatever you have.

200 points total are necessary for an A, and there will be many more problems, so you may skip problems
which you are not interested in.

Asterisks indicate problems representative of what might appear on the comprehensive exam.
Plusses indicate problems whose solutions will likely involve background beyond what has been taught here
and in 701/702.
\\
\\
\begin{enumerate}
\item (* 5 points)
Adapting the solution of the circle problem, directly determine the number of lattice points within the ellipse
$x^2 + 5y^2 = N$. 

With this problem, and all others, obtain an explicit bound for the error term. Implied constants
(i.e. $O$-notation) are okay, but specify (or avoid if possible) any dependence on other parameters.

\item (* 5 points)
Adapting the solution of the circle problem, directly determine the number of lattice points within the sphere
$x^2 + y^2 + z^2 = N$.

\item (5+ points)
State and directly prove a generalization of the previous two problems.

\item (5 points)
Give another proof of the previous two questions by explicitly verifying that they satisfy the conditions
of Davenport's lemma.

\item (* 10 points)
Let $p$ be an odd prime. Determine the number of solutions to $x^2 + y^2 = n$, where $n \leq N$ and $n$
is a quadratic residue mod $p$.

Be sure to compute and specify the dependence of your error term on $p$.

\item (* 5 points)
Repeat the previous exercise, but substitute the condition $n \equiv 2 \mod p$ for the quadratic residue
condition. What changes?

\item (5 or (+) 10 points)
Define the {\itshape Dirichlet series}
\[
\zeta(s) = 1 + 2^{-s} + 3^{-s} + 4^{-s} + \cdots,
\]
\[
L(s, \chi_{-4}) = 1 - 3^{-s} + 5^{-s} - 7^{-s} + \cdots,
\]
\[
\zeta_{\mathbb{Z}[i]}(s) = \frac{1}{4} \sum_{(x, y) \neq (0, 0)} (x^2 + y^2)^{-s}.
\]
You may regard these as formal infinite series, or you can regard $s$ as a complex variable, in which 
case it is not too hard (but a bit messy) to prove that they all converge absolutely for $\Re(s) > 1$.
Moreover, all of them have meromorphic continuation to the complex plane.

Investigate the truth of the identity
\[
\zeta(s) \cdot L(s, \chi_{-4}) = \zeta_{\mathbb{Z}[i]}(s).
\]
Begin by working out and checking the first 20 or 30 terms (or more by computer). If you are feeling
plucky and have some additional background, attempt to prove it.

\item (10 points)
In a page or so, describe what you learned from {\bf Ben Antieau}'s colloquium lecture, along with some
additional topics (if any) that his lecture led you to want to learn.

\item (10 points)
In a page or so, describe what you learned from {\bf Claudiu Raicu}'s colloquium lecture, along with some
additional topics (if any) that his lecture led you to want to learn.

If you have any comments about any of our job candidates which you would like passed on to the hiring committee, please write them with your homework or e-mail them to me or to Adela Vraciu (\url{vraciu@math.sc.edu}).


\end{enumerate}

\end{document}
