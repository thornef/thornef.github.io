\documentclass[12pt]{article}
\usepackage{amssymb,amsmath,amsthm}
\usepackage{enumerate, verbatim, url}
\textwidth 7.0in
\oddsidemargin -0.4in
\evensidemargin -0.4in
\textheight 9.0in 
\pagestyle{empty}
\begin{document}
% 8.5in paper width -2x1in margin = 6.5in text width
\setlength{\topmargin}{-2mm}



% 11in paper height -2x1in margin = 9in text height


\begin{center}{\bf The Geometry of Numbers (Spring 2014): Homework 4}
\end{center}
\begin{center}Frank Thorne, thorne@math.sc.edu
\end{center}

Asterisks indicate problems representative of what might appear on the comprehensive exam.
Plusses indicate problems whose solutions will likely involve background beyond what has been taught here
and in 701/702.
\\
\\
\begin{enumerate}
\item (* 5 points)
Verify directly, via brute force, that the discriminant of a binary cubic form is
$SL_2(\mathbb{Z})$-invariant.

(You are welcome to outsource arithmetic, etc. to Sage or other software, but please use it only
for basic algebra and do not call any highbrow routines.)

\item (* 3 points)
Let $u^3 + a_2 u^2 v + a_3 u v^2 + v^3$ be a binary cubic form with first and last coefficients $1$.
Prove that its discriminant is equal to the polynomial discriminant obtained by setting either $u$ or $v$
equal to $1$.

\item (* 10 points)
Do the exercise spelled out on p. 23.4 of the lecture notes, relating discriminants of forms to discriminants
of polynomials.

\item (* 5 points)
Carry out the details of the computation given on p. 23.5 of the lecture notes.

\item (* 10 points)
Describe what the Delone-Faddeev correspondence says over (some or all of) the following fields:
$\mathbb{C}$, $\mathbb{R}$, $\mathbb{Q}$, $\mathbb{F}_p$, $\mathbb{Q}$, $\mathbb{C}(t)$,
$\mathbb{Q}_p(t)$. Describe both sides of the correspondence, and explain what conclusions
Delone-Faddeev allows you to draw, in case any of them are nontrivial.

\item (5 points)
Prove that there are $\frac{1}{3} (p^2 - 1) (p^2 - p)$ irreducible binary cubic forms over
$\mathbb{F}_p$. (Hint: use Delone-Faddeev.)

\item (12 points)
Formulate the natural generalization of Delone-Faddeev to quartic forms and fields, and illustrate
by counterexample that it does not hold.

\item (* 5 points)
Write down some cubic rings (including some of the weird ones) and compute their discriminants.

\item (* 15 points)
Work out several explicit examples of the Delone-Faddeev correspondence over $\mathbb{Z}$. Your examples
should include reducible and irreducible binary cubic forms, including a binary cubic form which factors
as the product of a linear times a quadratic; integral domains, rings with zero divisors but
no nilpotents, and rings with nilpotents. Compute the relevant discriminants, and summarize your conclusionsl.

\item (* 10 points)
Is the following true or false?

Consider the cubic ring $\mathbb{Z}[\alpha]$, where $\alpha^3 + b \alpha^2 + c \alpha + d = 0$.
Then, the corresponding cubic form is $u^3 + b u^2 v + c u v^2 + d v^3$.

If this is false (or imprecisely stated), find a better version of this statement if possible. 

\end{enumerate}

\end{document}
