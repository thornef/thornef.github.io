\documentclass[11pt]{amsart}
\usepackage{amssymb,amsmath,amsthm}
\usepackage[mathscr]{euscript}
\usepackage{enumerate, verbatim, url}
\oddsidemargin = 0.0cm \evensidemargin = 0.0cm \textwidth = 6.5in
\textheight =8.5in
\newtheorem{case}{Case}
\newcommand{\caseif}{\textnormal{if }}
\newcommand{\leg}[2]{\genfrac{(}{)}{}{}{#1}{#2}}
\newcommand{\bfrac}[2]{\genfrac{}{}{}{0}{#1}{#2}}
\newcommand{\sm}[4]{\left(\begin{smallmatrix}#1&#2\\ #3&#4 \end{smallmatrix} \right)}
\newtheorem{theorem}{Theorem}
\newtheorem{lemma}[theorem]{Lemma}
\newtheorem{corollary}[theorem]{Corollary}
\newtheorem{conjecture}{\bf Conjecture}
\newtheorem{proposition}[theorem]{Proposition}
\newtheorem{definition}[theorem]{Definition}
\renewcommand{\theequation}{\thesection.\arabic{equation}}
\renewcommand{\thetheorem}{\thesection.\arabic{theorem}}
\theoremstyle{remark}
\newtheorem*{theoremno}{{\bf Theorem}}
\newtheorem*{remark}{Remark}
\newtheorem*{example}{Example}
\numberwithin{theorem}{section} \numberwithin{equation}{section}

\newcommand{\ord}{\text {\rm ord}}
\newcommand{\li}{\text {\rm li}}
\newcommand{\putin}{\text {\rm \bf ***PUT IN***}}
\newcommand{\st}{\text {\rm \bf ***}}
\newcommand{\mfG}{\mathfrak{G}}
\newcommand{\mfF}{\mathfrak{F}}
\newcommand{\pr}{\text {\rm pr}}
\newcommand{\sfpart}{\text {\rm sf}}
\newcommand{\calM}{\mathcal{M}}
\newcommand{\calW}{\mathcal{W}}
\newcommand{\calC}{\mathcal{C}}
\newcommand{\calO}{\mathcal{O}}
\newcommand{\GG}{\mathcal{G}}
\newcommand{\FF}{\mathcal{F}}
\newcommand{\QQ}{\mathcal{Q}}
\newcommand{\Mp}{\text {\rm Mp}}
\newcommand{\frakG}{\mathfrak{G}}
\newcommand{\Qmd}{\mathcal{Q}_{m,d}}
\newcommand{\la}{\lambda}
\newcommand{\R}{\mathbb{R}}
\newcommand{\C}{\mathbb{C}}
\newcommand{\F}{\mathbb{F}}
\newcommand{\Cl}{{\text {\rm Cl}}}
\newcommand{\Tr}{{\text {\rm Tr}}}
\newcommand{\rk}{{\text {\rm rk}}}
\newcommand{\Q}{\mathbb{Q}}
\newcommand{\Qd}{\mathcal{Q}_d}
\newcommand{\Z}{\mathbb{Z}}
\newcommand{\N}{\mathbb{N}}
\newcommand{\SL}{{\text {\rm SL}}}
\newcommand{\GL}{{\text {\rm GL}}}
\newcommand{\textmod}{{\text {\rm mod}}}
\newcommand{\textMod}{{\text {\rm Mod}}}
\newcommand{\sgn}{\operatorname{sgn}}
\newcommand{\calP}{\mathcal{P}}
\newcommand{\PSL}{{\text {\rm PSL}}}
\newcommand{\lcm}{{\text {\rm lcm}}}
\newcommand{\Gal}{{\text {\rm Gal}}}
\newcommand{\add}{{\text {\rm add}}}
\newcommand{\sub}{{\text {\rm sub}}}
\newcommand{\Sym}{{\text {\rm Sym}}}
\newcommand{\End}{{\text {\rm End}}}
\newcommand{\Frob}{{\text {\rm Frob}}}
\newcommand{\Disc}{{\text {\rm Disc}}}
\newcommand{\Stab}{{\text {\rm Stab}}}
\newcommand{\Op}{\mathcal{O}_K}
\newcommand{\h}{\mathfrak{h}}
\newcommand{\G}{\Gamma}
\newcommand{\g}{\gamma}
\newcommand{\zaz}{\Z / a\Z}
\newcommand{\znz}{\Z / n\Z}
\newcommand{\ve}{\varepsilon}
\newcommand{\Div}{{\text {\rm Div}}}
\newcommand{\tr}{{\text {\rm tr}}}
\newcommand{\odd}{{\text {\rm odd}}}
\newcommand{\bk}{B_k}
\newcommand{\rr}{R_r}
\newcommand{\sump}{\sideset{}{'}\sum}
\newcommand{\gkr}{\mathfrak{g}_{k,r}}
\newcommand{\re}{\textnormal{Re}}
\newcommand{\im}{\textnormal{Im}}
\newcommand{\Res}{\textnormal{Res}}
\newcommand{\calS}{\mathcal{S}}
\newcommand{\Aut}{\textnormal{Aut}}
\def\H{\mathbb{H}}



\begin{document}
\title[The Geometry of Numbers]
{The Geometry of Numbers}
\author{Frank Thorne}
\address{Department of Mathematics, University of South Carolina,
1523 Greene Street, Columbia, SC 29208}
\email{fthorne@math.stanford.edu}

\maketitle
Office hours: Tuesdays, 2:00-5:00.
\\
\\
{\bf Learning outcomes.} The student will master a variety of classical methods used in
number theory, concerning the solution to counting problems in terms of lattice points, 
and will also be exposed to contemporary research in the subject.
tools used in modern algebraic number theory, and will also
learn their connection to classical number theoretic questions concerning the distribution of primes and the solutions of
Diophantine equations.
\\
\\
{\bf References.} Useful source materials include Cox, {\itshape Primes of the form
$x^2 + ny^2$}, Siegel's book on the geometry of numbers, and other references to be announced.
\\
\\
{\bf Instructional Delivery Strategy.} Lecture.
\\
\\
{\bf Grading. (Revised)} Large numbers of problems will be proposed, and the students will be required
to solve only some of them. 140+ for an A, 80+ for a B.
The number of points is somewhat arbitrary and does not reflect a particular percentage
of the number of problems assigned.
\\
\\
{\bf Prerequisites.} Graduate standing in the math department or permission of instructor. 

Students should have taken 703/704 (algebra) or its equivalent elsewhere, or be concurrently enrolled in it. Also, students will be assumed to know elementary number theory, for example
as taught by Trifonov in the fall.

Some material presented will rely on more advanced topics (algebraic geometry, algebraic number theory, etc.)
Since students will not be required to solve all homework problems, students will still be able to succeed
without this background material.
\\
\\
{\bf List of topics.} The following is a rough outline of some of the topics we are likely to cover. We are likely not to have time for all of them.
\begin{enumerate}[1.]
\item {\bf Warmup: the circle and hyperbola problems.}
Here are two classical questions:
\begin{center}
How many points $(x, y)$ in the plane satisfy $x^2 + y^2 < N$?

\smallskip
How many points $(x, y)$ in the plane satisfy $x > 0, \ y > 0, xy < N$?
\end{center}
These are called, naturally enough, the {\itshape circle and hyperbola} problems respectively. The first problem
counts sums of two squares, or (equivalently) Gaussian integers (i.e., elements of $\mathbb{Z}[i]$), and the
second problem asks for the average of the {\itshape divisor function} $d(n)$. 

We will begin our course by proving asymptotic formulas for each. Then we will look at the error terms. We will
obtain a good error term for the circle problem, but we will ask if we can do better. (We can, if we use
{\itshape Poisson summation}, or (basically equivalently) the {\itshape functional equation for Dedekind's zeta
function}). 

For the hyperbola problem the error term will be much more problematic. In fact our analysis will find a
{\itshape secondary term}, which may be explained by looking at the {\itshape geometry of the cusp},
or at the {\itshape zeta function}.

Later, we will introduce and prove {\itshape Davenport's lemma} \cite{D}, which generalizes our method for solving
the circle problem. We will then consider what happens if we vary our problem. For example, suppose that we want
$x^2 + y^2$ or $xy$ to be a quadratic residue modulo $7$? Will the error terms go up or down?

\item {\bf Gauss: Quadratic forms.} 
We will then turn to the subject of {\itshape quadratic forms}, and its investigation by many, especially Gauss: 
representations of integers, class numbers, fundamental domains, Diophantine
questions, and investigating class numbers individually and on average. These questions are the prototype for
more difficult questions that were asked later (e.g. those appearing in Bhargava's work).

Our investigation will be thorough and highly accessible to beginners. One good book is Cox \cite{cox}, and there
are plenty of others.

\item {\bf Dirichlet: The class number formula.}
We will prove Dirichlet's class number formula, in both the imaginary and the more difficult real case. We will
see how the real case presents additional difficulties and how these are resolved. A thorough understanding of
this will be very helpful in understanding more technical geometry of numbers arguments.

\item {\bf Minkowski and Hermite: Bounding number fields by discriminant.}
We will cover classical theorems of Minkowski and Hermite on bounding the number of number fields of bounded
discriminant. Since Minkowski's theorem was thoroughly covered in Spring 2013 (in 788P) it will get an abbreviated
treatment here.

\item {\bf Siegel on quadratic forms.}
We then plan to 
turn to Siegel's investigation of {\itshape indefinite} quadratic forms. and see how he computed
the average class number, weighted by the regulator. An important point is to see why he needed this weighting.
Indeed, we will discover an annoying phenomenon: for real quadratic fields you cannot generally count the
class group independently from the regulator.

\item {\bf Davenport and Heilbronn.}
We will study Davenport and Heilbronn's parameterization of cubic rings and fields in terms of binary cubic forms.
As always there
are two questions to be studied: (1) How does one set up such a parameterization, and (2) how does one
then count the resulting lattice points?

\item {\bf Contemporary work: Bhargava, Wood, Shankar, Ho, ...}
A lot of exciting work is being conducted in this subject {\itshape right now.} As time permits we will give
a quick overview of this subject. I also hope to have one or two guest lecturers come and address our number
theory seminar.

(This is a highly suitable topic for Ph.D. dissertations.)
\end{enumerate}

{\bf Attendance policy.} No penalty will be enforced for missed classes.

\begin{thebibliography}{99}

\bibitem{BST} M. Bhargava, A. Shankar, and J. Tsimerman,
\emph{On the Davenport-Heilbronn theorem and second order terms}, on the arXiv.

\bibitem{Cox} D. Cox,
\emph{Primes of the form $x^2 + n y^2$}, second edition.

\bibitem{D} H. Davenport,
\emph{On a principle of Lipschitz.}
\end{thebibliography}

\end{document}


