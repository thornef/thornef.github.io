\documentclass[12pt]{article}
\usepackage{amssymb,amsmath,amsthm}
\usepackage{enumerate, verbatim, url}
\textwidth 7.0in
\oddsidemargin -0.4in
\evensidemargin -0.4in
\textheight 9.0in 
\pagestyle{empty}
\begin{document}
% 8.5in paper width -2x1in margin = 6.5in text width
\setlength{\topmargin}{-2mm}



% 11in paper height -2x1in margin = 9in text height


\begin{center}{\bf The Geometry of Numbers (Spring 2014): Homework 2}
\end{center}
\begin{center}Frank Thorne, thorne@math.sc.edu
\end{center}

Asterisks indicate problems representative of what might appear on the comprehensive exam.
Plusses indicate problems whose solutions will likely involve background beyond what has been taught here
and in 701/702.
\\
\\
\begin{enumerate}
\item (* 5 points)
Let $ax^2 + bxy + cy^2$ be a quadratic form of discriminant $D \neq 0$. Prove that it is positive definite if and only if its discriminant
$D > 0$ and $a > 0$. In addition, describe what happens if $D = 0$.

\item (* 5 points)
Can a quadratic form be indefinite over $\mathbb{R}$, but only represent positive integers when $x, y \in \mathbb{Z}$?

\item (* 5 points)
Prove that the action of $GL_2(\mathbb{Z})$ defined in lecture does {\itshape not} define a {\itshape left} action
on binary quadratic forms. 

In other words, find $g, g'$ and $f$ for which (if a left action was defined) we would have
$g(g'(f)) \neq (gg')(f)$.

\item (* 5 points)
Prove {\itshape directly} (i.e. do not quote the reduction theorem) that the quadratic forms
$x^2 + 5y^2$ and $2x^2 + 2xy + 3y^2$ are not equivalent.

\item (10 points)
Filling in all of the remaining details only sketched in class, 
present a complete proof that every primitive positive definite
quadratic form is properly equivalent to a unique reduced form.

\item (2 points each, up to 10)
Compute $h(D)$ for $D = -7. -8, -163, -67, \cdots$.

\item (* 5 points)
Find some $D$ for which $h(D) > 5$.

(Hint: Consider using Dirichlet's class number formula to guess how big $h(D)$ will be, then use the reduction theory.)

\item (* 10 points total, includes partial credit)
Compute the automorphism group of an arbitrary positive definite quadratic form.

(At least do $x^2 + xy + y^2$, that is the interesting one. Also do $x^2 + n y^2$ for $n > 1$.)

\item (15 points)
Give a complete proof of the formula for $r_D(n)$, possibly with conditions like $D$ is odd, coprime to $n$, etc.
The Cox exercise passed out in class gives an excellent blueprint, or roll your own.

\item (* 10 points)
Find the fundamental units for all positive discriminants $D < 14$, and write down the corresponding automorphism group.

\item (10 points)
In a page or so, describe what you learned from {\bf Nathan Ilten}'s colloquium lecture, along with some
additional topics (if any) that his lecture led you to want to learn.

\item
You will also get credit for {\bf any and all} related exercises in Granville's notes or Cox's book which you hand in.
{\bf All} of them are interesting and highly relevant to this course and to number theory in general.

\item ({\bf Bonus})
Read Granville's notes, and submit a list of typos, corrections, or mistakes (if you find any). I will submit a list
to him with names of submitters included.

\end{enumerate}

\end{document}
