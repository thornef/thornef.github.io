\documentclass[12pt]{article}
\usepackage{amssymb,amsmath,amsthm}
\usepackage{enumerate, verbatim, url}
\textwidth 7.0in
\oddsidemargin -0.4in
\evensidemargin -0.4in
\textheight 9.0in 
\pagestyle{empty}
\begin{document}
% 8.5in paper width -2x1in margin = 6.5in text width
\setlength{\topmargin}{-2mm}



% 11in paper height -2x1in margin = 9in text height


\begin{center}{\bf The Geometry of Numbers (Spring 2014): Homework 3}
\end{center}
\begin{center}Frank Thorne, thorne@math.sc.edu
\end{center}

Asterisks indicate problems representative of what might appear on the comprehensive exam.
Plusses indicate problems whose solutions will likely involve background beyond what has been taught here
and in 701/702.
\\
\\
\begin{enumerate}
\item (* 10 points)
On page 13.2 of my notes online, a graph was drawn indicating the lattice points we counted
in our proof of Dirichlet's class number formula.

Choose your favorite positive discriminant $D \equiv 1 \pmod 4$ and a quadratic form
$a x^2 + bxy + cy^2$ of discriminant $D$. Note that this form will necessarily have
$b \neq 0$.

For your particular choice of quadratic form, draw the analogous graph. How many points
are contained in it, as a function of $N$? State the result and explain the connection
to Dirichlet's class number formula.

\item (10+ points)
Conduct numerical experiments concerning $h(D)$ for a wide variety of discriminants $D$.
Report on what you find. What is $h(D)$ on average? Are there any factors (e.g. when $D$
lies in certain residue classes) that cause $h(D)$ to be big? What is the largest that
$h(D)$ looks like it ever gets?

For an additional 10 points, do numerical experiments {\itshape both} by writing your
own program from scratch to compute class numbers, and {\itshape also} by programming
in PARI/GP, Sage, or some other such language and using the built-in functionality.
(Confirm that they agree!!)

\item (* 10 points)
Describe the results of applying Davenport's lemma for a variety of convex bodies. In
particular, what do you get for the following? Work out the main terms and the error terms
in:
\begin{itemize}
\item The $n$-dimensional sphere of radius $r$,
\item An $n$-dimensional ellipsoid of radii $r_1, r_2, \dots, r_n$,
\item An $n$-dimensional cube of side length $r$.
\end{itemize}
In addition, find a convex body where the error term is larger than the main term, where
the main term predicts that the body should contain many points, but which in fact contains
no lattice points at all.

\item (10 points)
Prove the `compact' version of Minkowski's convex body theorem:

Let $T \subseteq \mathbb{R}^n$ be a compact, convex, symmetric, 
measurable set, and $\Lambda$ a full lattice
in $\mathbb{R}^n$. Then, if $\mu(T) \geq 2^n {\textrm{Vol}}(\Lambda)$ then $T$
contains a nonzero point of $\Lambda$.

(This was proved in Lecture 17 without the compactness assumption, and with strict inequality
required for $\mu(T)$.)

\item (* 5 points)
Is Minkowski's theorem sharp? Either: For any fixed $\delta$, find a lattice $\Lambda$
and some $T$ for which $\mu(T) > (1 - \delta) 2^n {\textrm{Vol}}(\Lambda)$, or provide
some kind of evidence that the result can be further improved.

\item (20 points)
Prove that if $K$ is a cubic field, then $|{\textrm{Disc}}(K)| \geq 23$.

(I don't think it is so hard, but I want to encourage everyone to do it so it is worth a lot of
points.)

\end{enumerate}

\end{document}
