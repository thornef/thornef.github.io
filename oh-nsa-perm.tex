\documentclass[11pt]{article}
%%%%%%%%%%%%%%%%%%%%%%%%%%%%%%%%%%%%%%%%%%%%%%%%%%%%%%%%%%%%%%%%%%%%%%%%%%%%%%%%%%%%%%%%%%%%%%%%%%%%%%%%%%%%%%%%%%%%%%%%%%%%%%%%%%%%%%%%%%%%%%%%%%%%%%%%%%%%%%%%%%%%%%%%%%%%%%%%%%%%%%%%%%%%%%%%%%%%%%%%%%%%%%%%%%%%%%%%%%%%%%%%%%%%%%%%%%%%%%%%%%%%%%%%%%%%
\usepackage{amsfonts}
\usepackage{amssymb}
\usepackage{amsmath}
\usepackage{color,hyperref}
\usepackage{graphics}
\usepackage{times}
\usepackage{fancyhdr}
\usepackage{datetime}



\definecolor{darkblue}{RGB}{0,0,50}
\hypersetup{colorlinks,breaklinks,
            linkcolor=darkblue,urlcolor=darkblue,
            anchorcolor=darkblue,citecolor=darkblue}


%%%%%%%%%%%%%%%%%%date in format Sunday, October 25, 2011$$$
\usdate
\def\theday	{\dayofweekname{\day}{\month}{\year}}
\def\mydate	{\theday , \today}


%%%%%%%%%%%%opening%%%%%%%%%
\def\person		{So and So}
\def\role		{An important person}
\def\deptname		{Some Department}
\def\university		{Blank College or University}
\def\inlinename		{the Blank College or University}
\def\citystatezip	{City, State, zippy}
\def\addressing		{So and So}
\def\salutation		{Dear \addressing ,}

%%%%%%%%%%make your own macros here%%%%%%%%%%%%%%%%%



%%%%%%%%%%%%%%%%%%%%%%%%%%%%%%%%%%%%%%%%%%%%%%%%%%%








%%%%%%%%%%return address%%%%%%%%%%
\def\myname		{Frank Thorne}
\def\mydeptname		{Department of Mathematics}
\def\myaffiliation	{University of South Carolina}
\def\mystreet		{1523 Greene Street}
\def\mycitystatezip	{Columbia,\ SC\ \ 29208}
\def\myphone		{{\it Phone:} (803)404-4057 (home)}
\def\office		{(803)777-4224}
\def\fax		{{\it Fax:}(803)777-3783}
\def\email		{thorne@math.sc.edu}
\def\url		{http://www.math.sc.edu/$\sim$thornef} % NOTE: use $\sim$ for tilde
%%%%%%%%%%%%%%%%%%%%%%%%%%%%%%%%%%%%

%%%%%%%%%%%%%%%margins%%%%%%%%%%%%%%%%%%
\oddsidemargin	 0in
\evensidemargin  0in
\topmargin	-0.75in
\textwidth	 6.5in
\thispagestyle{fancy}
\textheight	 6.25in
\headwidth	\textwidth
\headheight 	1.25in
\headsep 	1in	%This controls how far below the letterhead the text of the letter will begin
\parindent      0ex	%controls how the indentation of new paragraphs behaves
\parskip		6pt	%controls the space between paragraphs



\addtolength{\textheight}{.5in}
%%%%%%%%%%%%%%%%letterhead%%%%%%%%%%%%%%%%
\fancyfoot{}
\renewcommand{\headrulewidth}{0pt}
\renewcommand{\footrulewidth}{0pt}

%%%usc logo
\fancyhead[L]{\resizebox{2.5in}{!}{\includegraphics{USC_Linear.jpg}} \vspace{0.6in}}

%%%%%name, line, date%%%%%%
%\fancyhead[C]{
%\hspace{1.5in}\raggedright\textbf{\LARGE \bf \myname}\vspace{-6pt}\\
%\hspace{1.5in}\raggedright\hrulefill \vspace{2.5pt}\\
%\hspace{1.5in}\raggedright\textbf{\normalfont\large \mydate} 
%%%%%%the below space moves the above content so the line matches up with line in logo
%\vspace{.149in}}

%%%%%%%%%%return address on right%%%%%%%
\fancyhead[R]{
\parbox[t]{3.1in}{ \footnotesize   \em \vspace{0.5in}
				\myname \\
				\mydeptname   \\
				\myaffiliation \\
				\mystreet \\
				\mycitystatezip \\
				\email \\
\\ 
				\mydate
				}\hspace{-1.3in}
\vspace{0.2in}}








\begin{document}
\pagestyle{fancy}

\vskip 30pt
{Kim Sacra \\ 
National Security Agency \\ 
9800 Savage Road, Suite 6844 \\
Fort Meade, MD 20755-6844 \\
ATTN: Mathematics Hiring Manager \\
{\it kcsacra@nsa.gov}
}

\enlargethispage

\vskip 10pt
Dear Ms. Sacra (or whomever else it may concern),
\vskip 10pt

I am pleased to strongly recommend {\bf Richard Oh} for a permanent position with the National Security Agency.
Richard is currently working with you on a summer internship, so you understand
why I am recommending him highly. Here I describe what I've
observed as his Ph.D. advisor at the University of South Carolina.
\\
\\
{\bf First impressions.}
I began at USC in Fall 2011, and I met Richard in his second year,
when he took my graduate course on analytic number
theory. Richard, along with one other student who was approximately his equal, was the strongest out of
a class of nine.

My course was rather ambitious. I decided to use Davenport's book {\itshape Analytic
Number Theory}, as opposed to gentler introductions to the subject which had been
used at USC in the past. My course covered a variety of elementary and complex analytic topics
in the subject (indeed, more than at least one of my colleagues seemed to think wise), 
and the lectures and the homeworks 
made full use of technical tools such as Fourier transforms and complex contour integration.
I assigned weekly problem sets as well as a term project.

Richard's overall homework score was either the highest or the second highest in the class. (I regret
that I have misplaced my grade sheet and don't remember more precisely.) He consistently solved
difficult optional problems that were avoided by many others, and he wrote up his solutions
impeccably well: indeed, I would rate his ability to write mathematics clearly at least as high as the 
average math professor.
His term project was also, in my opinion, the best in the class. He chose to tackle the research
literature and present a
paper of Dorian Goldfeld, giving a simple proof that $L(1, \chi_d) \gg d^{- \epsilon}$. Goldfeld's
writeup was quite short, and Richard thoroughly mastered his paper and wrote a much longer paper
explaining all of the additional details in the proof. Richard also gave an excellent lecture
on this paper; it was clear that he understood both the details and the big picture.

I was delighted that Richard decided to write his thesis with me. He has
demonstrated not only perseverence and excellent mathematical ability, but also independence. 
After my analytic number theory course, I suggested a research problem related to his term project,
but he has decided that he would prefer to seek out problems related to cryptography.
\\
\\
{\bf Richard's research interests.}
Richard has sought out and mastered a variety of demanding reading on analytic number theory and
its relation to cryptography. After he decided to work on cryptography,
Richard showed me Shparlinski's book {\itshape Cryptographic Applications of Analytic Number Theory},
which he apparently found on his own, as well as a list of open problems in the subject. After
skimming the book, I suggested that he learn more about exponential sums, which are a mainstay of
analytic number theory and which feature prominently in the book. I asked him to read the treatment
of exponential sums in Iwaniec and Kowalski's {\itshape Analytic Number Theory}, a beautiful book which
nevertheless 
caused me severe headaches when I was a graduate student.
He came back with a long list of good questions.

Based on his independent reading, he chose to investigate a cryptographic hash algorithm known as the Very
Smooth Hash (VSH). This algorithm was introduced by Contini, Lenstra, and Steinfeld in 2005, and is of interest
because it runs quickly and also admits proofs of security (which, to my knowledge, popular algorithms such as MD5 and SHA1 do not). Richard has been investigating the problem of proving
explicit lower bounds for 
the security of VSH.

Such bounds translate into analytic number theory problems, and Richard has been carefully studying multiple papers
which discuss VSH and some of the underlying number theory. He has paid particular attention to a paper of
Pappalardi, studying the order of finitely generated subgroups of $\mathbb{Q}^{\times} \pmod{p}$, to the point
of also studying one of the papers which underlie it, with an eye towards improving Pappalardi's results, and 
potentially leading to a lower bound on the security of VSH.

I cannot tell you whether this line of proposed research will succeed, but I certainly find it promising.
In any case I have been impressed by his investigations
so far -- by his competence, by his thoroughness, and by his enthusiasm. 
\\
\\
{\bf Computational ability.} Richard's work has involved numerical experiments, and he has used
SAGE, C++ (with libraries to handle arbitrary precision arithmetic), and shell scripting to generate data related
to exponential sums occurring in the VSH algorithm, 
and gnuplot to graph it. Although I did not look at his code, 
but I did look at the graphs of his data -- he found fascinating behavior in exponential sums related to VSH,
demonstrating that (completely on his own) he has found a deeply interesting area of number theory to investigate.
\\
\\
{\bf Undergraduate research.} Richard completed an undergraduate research project at Emory, which quite impressed me. He read a
modern research paper in cryptography (which was apparently confusingly written) and wrote a much longer
version
of the same paper, explaining the method in fuller detail and also implementing the algorithm in the
SAGE programming language.

Oversimplifying somewhat, his project is as follows. (If I say anything inaccurate, the error is mine
and not Richard's; he is much
more knowledgeable about cryptography than I am.) The RSA cryptosystem relies on a private key and a
public key, which
can (essentially) be represented as large integers. When a user creates a keypair, he (or she) generates
two large (and related) integers and
broadcasts the public key. If another user somehow manages to determine the private key, then he (or
she) has broken the cryptosystem
and can impersonate the original user. 

Richard's paper describes an attack which breaks RSA when the private key is numerically small, relative
to its allowable range.
RSA is based on number theory, and a method using continued fractions can be applied (when the private
key is small) to determine the private
key from the public key.

When Richard described this to me, it seemed interesting but unlikely to be of any practical
consequence. Not so. Apparently,
small private keys lead to a faster implementation, and Richard
determined that the Atlanta subway system (MARTA) used RSA cryptography with this shortcut in its
farecards.
Exploiting this vulnerability, he used his own software (with existing hardware) to produce a physical
farecard allowing him unlimited travel all over Atlanta. 
(He assures me
that after testing his farecard with a single free ride, he resumed paying for his transportation.)

Richard's work demonstrated a shrewd understanding of
the relationship between academic
research and real-world practice, which I imagine would be quite valuable at the NSA.
\\
\\
{\bf Oral presentation skills.} Richard has given two hour-long talks during our number theory seminar.
This is not required of our graduate students; Richard sought me out and asked to speak. Due to travel
obligations I was only able to attend one of his lectures; he gave a clear talk, demonstrating enthusiasm, mastery of his material, and an ability to put himself in the mind of an audience member.
\\
\\
{\bf Work with others.} Richard works well with his fellow graduate students. I asked for testimonials, and
one of them, Heather Smith, writes:
\begin{quote}
Richie and I worked quite well together in studying for classes, quals [qualifying exams], and comps [comprehensive exams]. Richie listens carefully to others.
He is an active listener, asking good questions and encouraging promising ideas.  He is able to explain his own
thoughts clearly. He has a knack at communicating on just the right level, based on your previous background. This
attests to the fact that he is a good listener. I have enjoyed working with Richie these past few years, sharing
knowledge and friendship. 
\end{quote}
{\bf Personality and character.} 
Finally, I should mention that I quite admire Richard personally; he is friendly, cheerful, and takes a
very positive attitude towards his work. He also takes the initiative outside of work; for example,
he has organized
a weekly department frisbee game (which I have cheerfully, if not always skillfully, participated in). 
I have always
known him to be completely honest, and 
I have no reservations whatsoever about his character.
\\
\\
I have a great deal of admiration for, and confidence in, Richard, and working with him has been one of
the highlights of my time at USC. If he stays in our Ph.D. program for the full five years (through 2015),
then I am confident that he will write an outstanding thesis on 
applications of analytic number theory to
cryptography -- and this, despite having a thesis advisor with very limited knowledge of cryptography.

If he graduates early, in 2014, then he will be obliged to cut his thesis a bit short, but I still 
expect it to be very good. In any case
I will be very proud to have worked with him.

I would not have nearly this level of confidence in most students in our department, but Richard's
ability, motivation, and independence cause him to stand out from the crowd. He has my overwhelming
recommendation for a permanent position with the NSA.

\hspace{.5\textwidth}\parbox[t]{2.95in}{
		    Sincerely,\\
		   \includegraphics{mysignature.jpg}\\ 

		    Frank Thorne\\
		    Assistant Professor of Mathematics
		   }
%if you want to sign the document yourself, replace Sincerely,\\ with Sincerely,\vspace{0.5in} and delete the line \includegraphics{}

\vfill





\end{document}
