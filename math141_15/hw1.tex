\documentclass[12pt]{article}
\textwidth 7.0in
\oddsidemargin -0.4in
\evensidemargin -0.4in
\textheight 9.0in 
\pagestyle{empty}
\usepackage{enumerate}
\begin{document}
% 8.5in paper width -2x1in margin = 6.5in text width
\setlength{\topmargin}{-2mm}



% 11in paper height -2x1in margin = 9in text height


\begin{center}{\bf Homework 1 - Math 141, Frank Thorne (thornef@mailbox.sc.edu)}
\end{center}
\begin{center}
{\bf Due Friday, August 31}
\end{center}

{\bf Important:} As with everything else in life, being right is not enough.
Please show your work, write in complete sentences, and explain your reasoning clearly. 
\\
\\
{\bf Required problems.}

\begin{enumerate}[(a)]
\item 
Stewart, Ch. 1.1, 1, 5-6, 12, 13.

\item
What is a function? (This is the most important question in all of mathematics.)

\item
Stewart, Ch. 1.2, 10, 16.

\item
Simplify $\frac{1}{x + 1} - \frac{1}{x}$.

\item
Simplify $(abc)^{10} (a^5 b^3 d^{-2})^{-2}$.

\item
Simplify $\frac{ \frac{1}{x + h} - \frac{1}{x}}{h}$.

\item
Simplify $\frac{(x + h)^2 - x^2}{h}$.

\item
Simplify $\frac{(xy^2)^2}{(x^2y)^2}$.

\item
Simplify $(x + 2)(x + 3) + (x + 2)(x - 3)$.

\item
Simplify $(x + 1)^2 (x + 2)^3 + (x + 1)^3 (x + 2)^2$.

\item
Factor $x^2 - a^2$.

\item
Factor $x^3 - a^3$.

\item
Factor $x^3 + a^3$.

\item
Define the trigonometric 
functions $\sin(x)$, $\cos(x)$, $\tan(x)$, $\sec(x)$, $\csc(x)$,
and $\cot(x)$.

\item
Determine (with a brief explanation) the values of each of the trigonometric functions above for $x = \pi/3$ and $x = 3 \pi/4$.

\item
Stewart, Ch. 1.3, 11-14 ({\bf show your work}), 31, 32, 53, 56.

\item
Define the exponential and logarithmic functions $e^x$ and $\ln x$. 

\item
Stewart, Ch. 1.5, 9-10.

\item
Define the term {\itshape inverse function}. Give an example of a function that has an inverse,
and of a function that does not.

\item
Define the logarithmic functions $\log_a(x)$ and $\ln(x)$.

\item
Stewart, Ch. 1.6, 18 (in addition, graph the inverse of $f$), 21-24, 47-50.

\end{enumerate}
{\bf Additional problems.}
\begin{enumerate}[(a)]
\item
Stewart, Ch. 1.3, 15-16.

\item
Stewart, Ch. 1.5, 11, 12.

\item
Stewart. Ch. 1.6, 5, 6, 20, 50, 52, 53.
\end{enumerate}
{\bf Bonus} (1 point).
\begin{enumerate}[(a)]
\item
Simplify the expression
$$(x - a) (x - b) (x - c) \cdots (x - z).$$
\end{enumerate}

\end{document}
