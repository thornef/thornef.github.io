\documentclass[12pt]{article}
\textwidth 7.0in
\oddsidemargin -0.4in
\evensidemargin -0.4in
\textheight 9.0in 
\pagestyle{empty}
\usepackage{enumerate, url, amssymb, amsmath}
\begin{document}
% 8.5in paper width -2x1in margin = 6.5in text width
\setlength{\topmargin}{-2mm}



% 11in paper height -2x1in margin = 9in text height


\begin{center}{\bf Problem Set 3 -- Arithmetic Geometry, Frank Thorne (thorne@math.sc.edu)}
\end{center}
\begin{center}
{\bf Due Friday, February 12, 2016}
\end{center}
Choose one. (Not that I discourage you from solving both.)

\begin{enumerate}[(1)]
\item
(Do not use the Riemann-Roch theorem for any of this. Please argue directly.)
\begin{enumerate}[(a)]
\item
Prove that the degree $0$ part of the Picard group of $\mathbb{P}^1$ is trivial.
\item
Let $L$ be a line in $\mathbb{P}^2$. 
Prove that the degree $0$ part of the Picard group of $L$ is trivial.
\item
Let $V = V(X^2 + Y^2 - Z^2)$ in $\mathbb{P}^2$. 
Prove that the degree $0$ part of the Picard group of $V$ is trivial.
\item
Let $E$ be an elliptic curve. 
Prove that the degree $0$ part of the Picard group of $V$ is {\itshape not} trivial.

You might follow (but in your own words please) the elegant proof here:
\url{http://www.mathematik.uni-kl.de/~gathmann/class/alggeom-2014/chapter-14.pdf}
\end{enumerate}

\item
(Suitable if you enjoy computer programming.)

Using Python, Java, Sage, PARI/GP, etc. (or any other language of your choice), write a computer program which accepts the following input:
\begin{itemize}
\item
The equation of an elliptic curve $E$ (which you may assume is of the form $y^2 = x^3 + Ax + B$).
\item
Two points $P_1$ and $P_2$ on $E$, each given as ordered pairs $(x_i, y_i)$. 
You may assume that neither is the point at infinity, but do not assume they are unequal.
\end{itemize}
Your program should return the coordinates of $P_1 + P_2$.

Please use rational number rather than floating point arithmetic if this is natively supported by your programming language of choice. But
{\itshape please do the arithmetic from scratch!} (i.e., don't use PARI/GP and Sage's built in functionality to do exactly this. Of course, this is a great
way to check your work.)
\end{enumerate}
\end{document}
