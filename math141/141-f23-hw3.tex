\documentclass[11pt]{article}
\textwidth 7.0in
\oddsidemargin -0.4in
\evensidemargin -0.4in
\textheight 9.0in 
\pagestyle{empty}
\usepackage{enumerate}
\begin{document}
% 8.5in paper width -2x1in margin = 6.5in text width
\setlength{\topmargin}{-2mm}



% 11in paper height -2x1in margin = 9in text height


\begin{center}{\bf Practice Problems 3 - Math 141, Frank Thorne (thornef@mailbox.sc.edu)}
\end{center}

{\bf Any of these problems that might appear on a quiz, assessment, or the final exam.} You will be 
expected to show your work, write in complete sentences, and explain your reasoning clearly. 

\medskip
{\bf Instructions.} You may use the definition, the `alternative formula', and
any other techniques {\bf in Chapters 3.1 or 3.2} at your discretion. The instructions `use the alternative formula' in 3.2, 23-26 are optional.
But {\bf do not use the differentiation
rules introduced in later chapters.}

{\bf The same is true for any question on a quiz, assessment, or exam which asks you to compute a derivative `directly from the definition'.}

\begin{enumerate}[(a)]

\item
Draw a graph of a function which is not differentiable, and geometrically explain why it is not differentiable.

\item
Give an equation of a function which is not differentiable, and algebraically explain why it is not differentiable.
(You can use the same function or a different function.)

\item
Give the definition of the {\itshape derivative} of a function
$f(x)$ at the point $x = a$. (Please give the algebraic definition,
using an equation.)

Draw a picture and explain why your
equation gives the slope of the tangent line to the graph of $f(x)$
at $x = a$.

\item
What is the {\itshape average rate of change} of a function $f(x)$ on an interval $[a, a + h]$, and what is the {\itshape 
instantaneous rate of change} of $f(x)$ at $x = a$?

Explain the relationship between these two concepts, and their relationship to the derivative.

\item
Thomas, Ch. 3.1, 11-24, 29-32.

\item
Thomas, Ch. 3.2, 1-31, 43-52.

\end{enumerate}

\end{document}
