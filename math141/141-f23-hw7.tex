\documentclass[12pt]{article}
\textwidth 7.0in
\oddsidemargin -0.4in
\evensidemargin -0.4in
\textheight 9.0in 
\pagestyle{empty}
\usepackage{skull}
\usepackage{enumerate}
\begin{document}
% 8.5in paper width -2x1in margin = 6.5in text width
\setlength{\topmargin}{-2mm}



% 11in paper height -2x1in margin = 9in text height


\begin{center}{\bf Practice Problems 7 - Math 141, Frank Thorne (thornef@mailbox.sc.edu)}

\medskip

\medskip

\medskip
$\skull$ $\skull$ {\bf WARNING!!} $\skull$ $\skull$
\end{center}
\vskip -0.1in
On exams, to receive full credit solutions to related rates questions must be {\bf clearly explained} and {\bf include a picture} where appropriate.

\begin{enumerate}[(a)]

\item
Thomas, Ch. 3.10, 11-42.

\item
A searchlight $L$ is 200 feet from a prison wall. It rotates at a constant rate of one revolution per 6 minutes.

An escaped felon is running along the wall trying to keep just ahead of the beam of light. At the
moment when the searchlight angle is 45 degrees, how fast does the prisoner have to run?

\item What do the words {\bf absolute maximum}, {\bf absolute minimum}, {\bf local maximum}, and {\bf local mini-
mum} mean?

\item What is the first derivative theorem for local extreme values? Why is it true?

\item Explain how to find the absolute extrema of a continuous function on a closed interval.

\item Thomas Ch. 4.1, 1-14, 21-36, 57-62.

\end{enumerate}

\end{document}
