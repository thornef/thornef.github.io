\documentclass[12pt]{article}
\textwidth 7.0in
\oddsidemargin -0.4in
\evensidemargin -0.4in
\textheight 9.0in 
\pagestyle{empty}
\usepackage{enumerate}
\begin{document}
% 8.5in paper width -2x1in margin = 6.5in text width
\setlength{\topmargin}{-2mm}



% 11in paper height -2x1in margin = 9in text height


\begin{center}{\bf Homework 7 - Math 141, Frank Thorne (thornef@mailbox.sc.edu)}
\end{center}

\begin{enumerate}[(a)]

\item
What do the words {\bf absolute maximum}, {\bf absolute minimum}, {\bf local maximum},
and {\bf local minimum} mean?

\item
What is the first derivative theorem for local extreme values? Give a brief,
informal explanation for why it is true.

\item
(Trick question. Explain why.) Explain how to find all the absolute maxima of a function.

\item
Explain how to find the absolute extrema of a continuous function on a closed interval.

\item
Thomas, Ch. 4.1, 1-4, 53-68 (even required, odd additional).

\item
What is the first derivative test for local extrema? Explain roughly why it is true.

\item
Thomas, Ch. 4.3, 1-6, 19-34 (even required, odd additional). In all cases {\bf
graph the function in question}.


\item
What does the first derivative tell you about the shape of a graph?

\item
What does the second derivative tell you about the shape of a graph?

\item
What is an inflection point? How do you find them? Why are they interesting?

\item
Thomas, Ch. 4.4, 9-18, 31-40 (even required, odd additional); 81-84 (all).

\end{enumerate}

\end{document}
