\documentclass[12pt]{article}
\textwidth 7.0in
\oddsidemargin -0.4in
\evensidemargin -0.4in
\textheight 9.0in 
\pagestyle{empty}
\usepackage{enumerate}
\begin{document}
% 8.5in paper width -2x1in margin = 6.5in text width
\setlength{\topmargin}{-2mm}



% 11in paper height -2x1in margin = 9in text height


\begin{center}{\bf Examination 3 - Math 141, Frank Thorne (thornef@mailbox.sc.edu)}
\end{center}
\begin{center}
{\bf Friday, November 30}
\end{center}

There will be seven or eight questions on the exam. Only one of the questions
marked (*) will appear. If the exam looks too long with eight questions, I will
remove one of the questions labeled (**).

\begin{enumerate}[(1)]
\item
One problem from (a)-(d) or Section 4.7 on Homework 9, selected at random.

\item
(**) One problem from Section 4.9 on Homework 9, selected at random.

\item
Problem (a), (b), or (l) on Homework 10, selected at random.
(If I ask (l), I {\bf might} ask only part of (l), but only if the
rest of the exam looks difficult.)

\item
(*) One problem from (c)-(k) on Homework 10, selected at random. With (c)-(e),
you will only be asked about four intervals.

\item
One problem from 49-52 of Stewart, Ch. 4.9, or 4 from Ch. 5.3, or
7 in the Chapter 5 Review, selected at random.

\item
(**) One problem from Section 5.3 of Stewart on Homework 10 (excluding 4), 
or Chapter 4.9, 57 or 63,
selected at random.

\item
(**) One problem from (b)-(d) or Section 5.4 on Homework 11, selected at random.

\item
(**) One problem from (e) on Homework 11 or Section 5.5 on Homeworks 11-12, selected at random.

\item (*)
One problem from Section 6.1 on Homework 12, selected at random.


\end{enumerate}

\end{document}
