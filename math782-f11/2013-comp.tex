
\documentclass[12pt]{amsart}
\usepackage{amssymb,amsmath,amsthm}
\usepackage{enumerate}
\usepackage{comment}
\usepackage{ifthen, url}
\newcommand{\textmod}{{\text {\rm mod}}}
\newcommand{\textMod}{{\text {\rm Mod}}}


\oddsidemargin = -1cm \evensidemargin = -1cm \textwidth = 6.8in
\textheight =8.5in
\topmargin =0.0in

\title{Comprehensive Exam in Analytic Number Theory (Fall 2013)}
\begin{document}           % End of preamble and beginning of text.
\maketitle
\vskip 1.0cm
\begin{enumerate}[1.]
\item
Refer to p. 8 of Davenport's book. Assume that $q$ is an odd prime $\equiv 3 \pmod{4}$,
\[
L(s) = \sum_{n = 1}^{\infty} \left( \frac{n}{q} \right) n^{-s},
\]
and
\[
G(n) = \sum_{m = 1}^{q - 1} \left( \frac{m}{q} \right) \exp(2 \pi i mn/q).
\]
\begin{enumerate}[(a)]
\item
Justify that $G \neq 0$ by computing $G^2$.
\item
Prove (1) on p. 8.
\item
Prove that the series in (2) converges for $|z| \leq 1$ and $z \neq 1$.
\item
Prove (6).
\item
Verify the statement immediately after (7).
\item
Dirichlet's class number formula for imaginary quadratic fields
says that, for $d < 0$, 
\[
h(d) = \frac{ w \sqrt{|d|}}{2 \pi} L(s, \chi_d),
\]
where $h(d)$ is the class number of $\mathbb{Q}({\sqrt{d}})$, and
\[
L(s, \chi_d) = \sum_{n = 1}^{\infty} \left( \frac{d}{n} \right) n^{-s}.
\]
Using all of these formulas (all of which were proved by Dirichlet), compute the
class number of $\mathbb{Q}(\sqrt{-31})$. (Pay close attention to how these $L$-functions are defined.)
\end{enumerate}
\vskip 0.5cm

\item
The nonvanishing of $L(1, \chi)$ is such a beautiful theorem that it's worth a second
proof. Refer to p. 37 of Iwaniec-Kowalski.
\begin{enumerate}[(a)]
\item
Justify the equation after (2.29).
\item
IK refer to `opening the convolution (2.29)'. Define the convolution of two arithmetic
functions, and say what two functions (2.29) is the convolution of.
\item
Justify each step of the proof after `We obtain', in substantially more detail
than Iwaniec and Kowalski do. For each of the $O$-terms, specify what variables,
if any, the implied constants depend on.

(Hint: My hand-written note `Require ...', which I wrote a few years ago, might be a useful hint,
but I wrote it in a somewhat sloppy way.)
\end{enumerate}
\vskip 0.5cm
\item
Explicitly describe all of the Dirichlet characters modulo 10.
\item
\vskip 0.5cm
Refer to Chapter 7 of Shakarchi and Stein's book.
\begin{enumerate}[(a)]
\item
Summarize the proof given there of the prime number theorem.
\item
With Lemma 2.4 as a model, prove a similar formula for
\[
\frac{1}{2 \pi i} \int_{c - i \infty}^{c + i \infty} \frac{a^s}{s(s + 1)(s + 2)}.
\]
For convergence issues, you are welcome, and indeed strongly encouraged, to use facts proved 
on pp. 192-193, e.g. to compare your integrals to theirs rather than to imitate their proof.
\item
Formulate an analogue of Proposition 2.3 which uses your variant of Lemma 2.4.
\item
Another analogue of Lemma 2.4 (Perron's formula) is that the integral
\[
\frac{1}{2 \pi i} \int_{c - i \infty}^{c + i \infty} \frac{a^s}{s} ds
\]
is equal to $0$ if $0 < a < 1$, and $1$ if $a > 1$. Do not prove this, but do prove
an analogue of Proposition 2.3 which directly gives a formula for $\psi(x)$.
\item
(bonus) All of this is related to a general phenomenon of interest to Fourier analysts.
How?
\item
Using this last analogue, if possible, would have made Stein and Shakarchi's 
proof of the prime number theorem
simpler, for example because it would eliminate the need for Proposition 2.2. However, there
is one point at which their argument would fail. Find it, and explain.
\end{enumerate}
\end{enumerate}

\end{document}
