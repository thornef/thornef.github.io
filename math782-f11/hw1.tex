\documentclass[12pt]{article}
\usepackage{amssymb,amsmath,amsthm}
\usepackage{enumerate, verbatim}
\textwidth 7.0in
\oddsidemargin -0.4in
\evensidemargin -0.4in
\textheight 9.0in 
\pagestyle{empty}
\begin{document}
% 8.5in paper width -2x1in margin = 6.5in text width
\setlength{\topmargin}{-2mm}



% 11in paper height -2x1in margin = 9in text height


\begin{center}{\bf Homework 1 - Analytic number theory}
\end{center}
\begin{center}Frank Thorne, thornef@webmail.sc.edu
\end{center}
\begin{center}
{\bf Due Friday, September 2}
\end{center}

There is a lot of work to do here! Most of the solutions I have not worked out myself in detail.
There is at least one problem I don't know how to solve.
Doing half the problems every week is pretty good.
\\
\\
\begin{enumerate}
\item (5 points)
(This problem is boring, but it is good hygiene to do it once in your life.)

(a) Let $a_n$, $n \geq 1$ be an infinite sequence. Come up with a good notion of what it should mean
for the product $\prod_n a_n$ to converge absolutely.

(b) Using your above definition, rigorously prove that if $s$ is a complex number with $\Re(s) > 1$,
the product $\prod_p \frac{1}{1 - p^{-s}}$ converges absolutely and is equal to $\sum_n \frac{1}{n^s}$.

\item (5 points)
Prove that for $\Re(s) > 1$ we have
\begin{equation}
\zeta(s) = s \int_1^{\infty} \frac{ \lfloor y \rfloor }{y^{s + 1}} dy
= \frac{s}{s - 1} - s \int_1^{\infty} \frac{ \{ y \} }{y^{s + 1}} dy.
\end{equation}
Prove further that the integral on the right converges absolutely for $Re(s) > 0$. (Prove additionally that it
is analytic as a function of $s$ if you have the complex analysis background.) This equation allows us to
define $\zeta(s)$ whenever $\Re(s) > 0$ and $s \neq 1$.

The unsolved {\itshape Riemann hypothesis} says that if $\zeta(s) = 0$ then $\Re(s) = \frac{1}{2}$. (Prove that,
and I will give you a lot of bonus points...)

\item (5 points)
This is basically the same proof we saw in class, but arranged slightly differently. (See p. 56 of Davenport.)
If we define $T(x) = \sum_{m \leq x} \lfloor x/m \rfloor$, prove that $T(x) = \sum_{n \leq x} \log n$.
Prove directly that 
\begin{equation}
T(x) - 2T(x/2) \leq \sum_{m \leq x} \Lambda(m)
\end{equation}
and conclude that 
\begin{equation}
\pi(x) \log x > x (\log 2 - o(1)).
\end{equation}

You can do the upper bound too if you like.
\item (10 points)
By considering the combination
\begin{equation}
T(x) - T(x/2) - T(x/3) - T(x/5) + T(x/30),
\end{equation}
prove better upper and/or lower bounds for $\pi(x)$.

\item (5 points)
Suppose that $\pi(x)$ is asymptotic to $C \frac{x}{\log x}$ for some $C$. Prove, using what we already know,
that $C$ must be 1.

(If you prefer, answer this question with $\psi(x)$ and $C x$ instead.)

\item (5 points)
Prove that (as asserted in lecture)
\begin{equation}
\sum_{p^e, \ \ e \geq 2} \frac{1}{p^e} = O(1),
\end{equation}
where the sum is over powers of primes (but excluding the primes themselves).

\item
(5 points, {\bf This one is important, please do it even if you skip a lot of questions}) 
Assuming that $\pi(x) \asymp \frac{x}{\log x}$, conclude (using partial summation) that $\psi(x) \asymp x$.
\end{enumerate}

\end{document}
