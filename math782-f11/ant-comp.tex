
\documentclass[12pt]{amsart}
\usepackage{amssymb,amsmath,amsthm}
\usepackage{enumerate}
\usepackage{comment}
\usepackage{ifthen, url}
\newcommand{\textmod}{{\text {\rm mod}}}
\newcommand{\textMod}{{\text {\rm Mod}}}


\oddsidemargin = -1cm \evensidemargin = -1cm \textwidth = 6.8in
\textheight =8.5in
\topmargin =0.0in

\title{Comprehensive exam in analytic number theory (Fall 2012)}
\begin{document}           % End of preamble and beginning of text.
\maketitle
Answer everything in the first part and as much of the second part as you can.
\section{The First Part}
\begin{enumerate}[1.]
\item
For a primitive Dirichlet character $\chi \ (\textmod \ q)$, define a character sum
$S_{\chi}(t) := \sum_{n \leq t} \chi(n).
$ The {\bf Polya-Vinogradov} inequality is the statement that
$|S_{\chi}(t)| \ll \sqrt{q} \log q.$
\\
\\
A paper of Goldmakher \cite{gold} offers the following nice proof that
$\max_{t \leq q} |S_{\chi}(t)| \gg \sqrt{q}$:
\vskip 0.1in
``A slick proof of this is to apply partial summation to the Gauss sum
$\tau(q) := \sum_{n \leq q} \chi(n) e(n/q)
$ and use the classical result that for primitive $\chi \ (\textmod \ q)$, $|\tau(\chi)| = \sqrt{q}$.''
\vskip 0.2in
(a.) Explain what the word ``primitive'' means, and show that it is necessary in the discussion above.

(b.) Prove that for primitive $\chi$, $|\tau(x)| = \sqrt{q}$. (Hint: evaluate $\tau(\chi) \overline{\tau(\chi)}$.)

(c.) Spell out the details of Goldmakher's argument.
\vskip 0.2in
\item 
(a.) Define the {\itshape convolution} $f \ast g$ of two arithmetic functions $f$ and $g$.

(b.) If $f$ and $g$ are multiplicative, prove that $f \ast g$ is as well.

(c.) Let $\mu(n)$ be the M\"obius function, and let $1$ be the constant function, equal to $1$ for every $n$. 

Evaluate $\mu \ast 1$, and thus obtain a simple identity for the Dirichlet series $\sum_n \mu(n) n^{-s}$.
\vskip 0.2in
\item
Assume that $x$ is a real number $> 1$, not an integer. Then the {\bf explicit formula} reads
$$\psi(x) = x - \sum_{\rho} \frac{x^{\rho}}{\rho} - \frac{ \zeta'(0)}{ \zeta(0)} - \frac{1}{2} \log(1 - x^{-2}).$$

(a.) Explain what the above means. Your answer should define the function $\psi(x)$ and say what the sum over $\rho$ is.

(b.) Give a sketch of how this formula is proved.

(c.) Suppose that you have proved that the above terms on the right, except for $x$, are all $o(x)$, so that
$\psi(x) \sim x$. Deduce an asymptotic formula (e.g. {\itshape the prime number theorem}) for the number of primes $\leq x$.

(d.) Given the above formula, prove that the zeta function has at least one nontrivial zero. (What does ``nontrivial'' mean?)

\end{enumerate}
\section{The Fun Part}
Answer as many questions as you can.
\\
\\
\begin{enumerate}[1.]
\item
The following appears in a still unpublished preprint of Bhargava and Shnidman.
\\
\\
\begin{quote}
By Theorem 14 and Lemma 16, it now suffices to count pairs $(b, c) \in L$, up to $SO_Q(\mathbb{Z})$-equivalence,
subject to the condition $Q'(b, c)^2 = (b^2 - bc + c^2)^2 < X$. The number of integral points inside the elliptic region
cut out by the latter inequality is approximately equal to its area $(2 \pi/\sqrt{3}) X^{1/2}$, with an error of at most $O(X^{1/4})$.
Meanwhile, being the (orientation-preserving) symmetry group of the triangular lattice, $SO_Q(\mathbb{Z})$ is isomorphic to
$C_6$, the cyclic group of order $6$. Since this is the cubic action, the cyclic subgroup $C_3 \subseteq SO_Q(\mathbb{Z})$
acts trivially. Up to equivalence, we thus obtain
$$\frac{2 \pi}{2 \sqrt{3}} X^{1/2} + O(X^{1/4})$$
points inside the ellipse.
\end{quote}
\vskip 0.1in
Evaluate the infinite sum
$$1 - \frac{1}{2} + \frac{1}{4} - \frac{1}{5} + \frac{1}{7} - \frac{1}{8} + \cdots$$
and explain the relation to the Bhargava-Shnidman excerpt.
\\
\\
\item
In his 1859 memoir, Riemann conjectured that the nontrivial zeroes $\rho = \beta + i \gamma$ 
of the zeta function satisfy $\beta = 1/2$. (If you somehow manage to prove this, you will definitely pass the exam.)
Riemann guessed this based on some numerical computations. He found that the first few zeroes are
$\rho = \frac{1}{2} + 14.134\cdots i$, $\rho = \frac{1}{2} + 21.012\cdots i$, $\rho = \frac{1}{2} + 25.010\cdots i$, $\dots$.

How might he have found these zeroes? Describe a method of proving this, subject to numerical computations that
could reasonably by done by hand.
\\
\\
\vskip 0.2in
\item
Let $f(n)$ be the characteristic function of integers $p_1^{e_1} p_2^{e_2} \cdots p_k^{e_k}$ such that, when written
as shown as a product of distinct primes, all the $e_i$ are odd.

Prove a formula for $\sum_{n \leq x} f(n)$ which is as explicit, and has as good of an error term, as possible.
\\
\\
\item
Explain something interesting about character sums which you have read in Iwaniec-Kowalski or elsewhere.

\end{enumerate}
\begin{thebibliography}{99}
\bibitem{gold}
L. Goldmakher, \url{http://arxiv.org/pdf/0911.5547v2.pdf}.
\end{thebibliography}
\end{document}