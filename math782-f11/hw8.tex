\documentclass[12pt]{article}
\usepackage{amssymb,amsmath,amsthm}
\usepackage{enumerate, verbatim, url}
\textwidth 7.0in
\oddsidemargin -0.4in
\evensidemargin -0.4in
\textheight 9.0in 
\pagestyle{empty}
\begin{document}
% 8.5in paper width -2x1in margin = 6.5in text width
\setlength{\topmargin}{-2mm}



% 11in paper height -2x1in margin = 9in text height


\begin{center}{\bf Homework 8 - Analytic number theory}
\end{center}
\begin{center}Frank Thorne, thornef@mailbox.sc.edu
\end{center}
\begin{center}
{\bf Due Friday, October 21}
\end{center}
\begin{enumerate}
\item(5 points)
It was proved in lecture (see also Ch. 18 of Davenport) that
\begin{equation}
\psi(x) = x + O\bigg(x\exp(-C_1 (\log x)^{1/2} )(\log x) + x\exp(- (\log x)^{1/2})(\log x)^2\bigg)
\end{equation}
for an effective constant $C_1$. Prove that this implies that
\begin{equation}
\psi(x) = x + O\bigg(x\exp(-C_2(\log x)^{1/2} )\bigg)
\end{equation}
and further
\begin{equation}
\pi(x) = \frac{x}{\log x} + O\bigg(x\exp(-C_3(\log x)^{1/2}) \bigg),
\end{equation}
where in your proof you determine values for $C_2$ and $C_3$ in terms of $C_1$.
(No need to determine the best constants.)

\item(5 points)
It was partially shown in class, and is shown entirely in Davenport, that
$$\sum_{\gamma > 0} \frac{\beta}{\beta^2 + \gamma^2} = \frac{-B}{2},$$
where $-B$ is an explicit constant $> 0.022$.

From this formula, prove explicit (no $\ll$, big-$O$) bounds for the number of zeroes in $\zeta(s)$
in the critical strip with $\Im(s) < T$.

({\itshape Caution:} We already implicitly used similar bounds to establish the existence of the
infinite product for $\xi(s)$, so this is not another approach to proving such bounds.)

\item(5 points)
The formula of the previous problem implicitly assumes that the zeroes of $\zeta(s)$
occur in complex conjugate pairs. Accordingly, you need to prove separately that
$\zeta(s)$ doesn't have any real zeroes between 0 and 1.

Prove this. 

\item(5 points)
Imagine that $\zeta(s)$ satisfied the Riemann hypothesis, except for four zeroes off the
critical line located at, say, $(.5 \pm .4) \pm 15i$. Show that this implies 
a version of the prime number theorem of the form
\begin{equation}
\psi(x) = x + \ast x^{9/10} + O(x^{1/2} \log^2 x).
\end{equation}
(You should determine what goes in the place of the asterisk.)

For 5 more points, compute, present, and explain numerical evidence that this version of the
prime number theorem is wrong. 

\item(5+ points)
Download the software ``L'' from Mike Rubinstein's webpage (easily found via Google),
and/or Tim Dokchitser's ``ComputeL''.
Run a variety of numerical experiments on the zeroes of the zeta function,
the values of $\zeta(1 + it)$, or anything else you're prepared to argue is
interesting. Report your findings.

\item(5 points)
If $a_i$, for $1 \leq i \leq k$, is a sequence of positive numbers between 0 and 1,
prove that $\sum_i a_i < 1 + \sum_{i < j} a_j a_j$.

\end{enumerate}

\end{document}
