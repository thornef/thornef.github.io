\documentclass[11pt]{amsart}
\usepackage{amssymb,amsmath,amsthm}
\usepackage[mathscr]{euscript}
\usepackage{enumerate, verbatim, url}
\oddsidemargin = 0.0cm \evensidemargin = 0.0cm \textwidth = 6.5in
\textheight =8.5in
\newtheorem{case}{Case}
\newcommand{\caseif}{\textnormal{if }}
\newcommand{\leg}[2]{\genfrac{(}{)}{}{}{#1}{#2}}
\newcommand{\bfrac}[2]{\genfrac{}{}{}{0}{#1}{#2}}
\newcommand{\sm}[4]{\left(\begin{smallmatrix}#1&#2\\ #3&#4 \end{smallmatrix} \right)}
\newtheorem{theorem}{Theorem}
\newtheorem{lemma}[theorem]{Lemma}
\newtheorem{corollary}[theorem]{Corollary}
\newtheorem{conjecture}{\bf Conjecture}
\newtheorem{proposition}[theorem]{Proposition}
\newtheorem{definition}[theorem]{Definition}
\renewcommand{\theequation}{\thesection.\arabic{equation}}
\renewcommand{\thetheorem}{\thesection.\arabic{theorem}}
\theoremstyle{remark}
\newtheorem*{theoremno}{{\bf Theorem}}
\newtheorem*{remark}{Remark}
\newtheorem*{example}{Example}
\numberwithin{theorem}{section} \numberwithin{equation}{section}

\newcommand{\ord}{\text {\rm ord}}
\newcommand{\li}{\text {\rm li}}
\newcommand{\putin}{\text {\rm \bf ***PUT IN***}}
\newcommand{\st}{\text {\rm \bf ***}}
\newcommand{\mfG}{\mathfrak{G}}
\newcommand{\mfF}{\mathfrak{F}}
\newcommand{\pr}{\text {\rm pr}}
\newcommand{\sfpart}{\text {\rm sf}}
\newcommand{\calM}{\mathcal{M}}
\newcommand{\calW}{\mathcal{W}}
\newcommand{\calC}{\mathcal{C}}
\newcommand{\calO}{\mathcal{O}}
\newcommand{\GG}{\mathcal{G}}
\newcommand{\FF}{\mathcal{F}}
\newcommand{\QQ}{\mathcal{Q}}
\newcommand{\Mp}{\text {\rm Mp}}
\newcommand{\frakG}{\mathfrak{G}}
\newcommand{\Qmd}{\mathcal{Q}_{m,d}}
\newcommand{\la}{\lambda}
\newcommand{\R}{\mathbb{R}}
\newcommand{\C}{\mathbb{C}}
\newcommand{\F}{\mathbb{F}}
\newcommand{\Cl}{{\text {\rm Cl}}}
\newcommand{\Tr}{{\text {\rm Tr}}}
\newcommand{\rk}{{\text {\rm rk}}}
\newcommand{\Q}{\mathbb{Q}}
\newcommand{\Qd}{\mathcal{Q}_d}
\newcommand{\Z}{\mathbb{Z}}
\newcommand{\N}{\mathbb{N}}
\newcommand{\SL}{{\text {\rm SL}}}
\newcommand{\GL}{{\text {\rm GL}}}
\newcommand{\textmod}{{\text {\rm mod}}}
\newcommand{\textMod}{{\text {\rm Mod}}}
\newcommand{\sgn}{\operatorname{sgn}}
\newcommand{\calP}{\mathcal{P}}
\newcommand{\PSL}{{\text {\rm PSL}}}
\newcommand{\lcm}{{\text {\rm lcm}}}
\newcommand{\Gal}{{\text {\rm Gal}}}
\newcommand{\add}{{\text {\rm add}}}
\newcommand{\sub}{{\text {\rm sub}}}
\newcommand{\Sym}{{\text {\rm Sym}}}
\newcommand{\End}{{\text {\rm End}}}
\newcommand{\Frob}{{\text {\rm Frob}}}
\newcommand{\Disc}{{\text {\rm Disc}}}
\newcommand{\Stab}{{\text {\rm Stab}}}
\newcommand{\Op}{\mathcal{O}_K}
\newcommand{\h}{\mathfrak{h}}
\newcommand{\G}{\Gamma}
\newcommand{\g}{\gamma}
\newcommand{\zaz}{\Z / a\Z}
\newcommand{\znz}{\Z / n\Z}
\newcommand{\ve}{\varepsilon}
\newcommand{\Div}{{\text {\rm Div}}}
\newcommand{\tr}{{\text {\rm tr}}}
\newcommand{\odd}{{\text {\rm odd}}}
\newcommand{\bk}{B_k}
\newcommand{\rr}{R_r}
\newcommand{\sump}{\sideset{}{'}\sum}
\newcommand{\gkr}{\mathfrak{g}_{k,r}}
\newcommand{\re}{\textnormal{Re}}
\newcommand{\im}{\textnormal{Im}}
\newcommand{\Res}{\textnormal{Res}}
\newcommand{\calS}{\mathcal{S}}
\newcommand{\Aut}{\textnormal{Aut}}
\def\H{\mathbb{H}}



\begin{document}
\title[Analytic number theory]
{Analytic number theory, Fall 2011}
\author{Frank Thorne}
\address{Department of Mathematics, University of South Carolina,
1523 Greene Street, Columbia, SC 29208}
\email{fthorne@math.stanford.edu}

\maketitle
\section{Details}
LeConte 310, MWF 2:30-3:20. Office hours by appointment; or as for Math 141.
\\
\\
{\bf Learning outcomes.} Analytic number theory includes the study of various sums, estimates, and averages of various quantities that
arise in number theory (notably the primes). The student will master a variety of basic tools used throughout the subject, such as partial
summation and M\"obius inversion. The student will also learn about the analytic theory of $L$-functions, and will study the proofs of
famous theorems concerning the distribution of the primes. Finally, the student will gain introductory exposure to some advanced topics
in the subject.

{\bf Grading.} Your grade will consist of {\bf homework} (75\%) as well as a short {\bf term project} (25\%). For the term project you are asked to
independently learn about some topic in analytic number theory, write a paper of 5+ pages on it, and give a 20-minute presentation to the class
or in the number theory seminar. A list of suggsted topics will be furnished later.

The homeworks will be long and difficult and will present a variety of different problems for you to solve. However,
an average of 50\% will suffice for an A, so
newcomers or students in other discplines will not be expected to master everything. Aspiring number theorists are naturally encouraged to be more ambitious.

There will be no exams.

\section{Textbook and topics}
We will be using Davenport, {\itshape Multiplicative number theory} as a basic reference. We should be able to cover about half the book, including
Dirichlet's theorem on primes in arithmetic progression, characters and cyclotomy, the class number formula, the Riemann zeta function and its
analytic properties, and the proof of the prime number theorem.

We will also spend at least a little time discussing some more elementary aspects of the subject:
general Dirichlet series, multiplicative functions, and ``elementary'' methods for estimating their partial sums.
(``Elementary'' doesn't mean ``easy'', but rather that we will not need complex analysis.)

I will also plan on a couple of seminar-style lectures which will introduce some more advanced topics, and which will not be the basis
for homework exercises. In particular I will say a little bit about how this theory
intersects with research areas being pursued by myself and the other number theorists at USC.

\section{Prerequisites}
The course has a number of ``soft'' prerequisites. Soft as in, the following material is foundational in studying analytic number theory
and you will need to become proficient in order to succeed.
It is not expected that you will be familiar with all of this material, and we will devote some time to discussing each of these topics.
The student who has not seen any of these topics should probably plan on additional background reading and hard work to succeed.
\\
\\
{\bf Abstract algebra.} We will use the language of groups, rings, and fields when it helps to illuminate the subject material. For example,
the multiplicative group $\mathbb{F}_p^{\times}$ is cyclic, and is isomorphic to its character group (the group of homomorphisms
$\mathbb{F}_p^{\times} \rightarrow \mathbb{C}^{\times}).$ It will be good to know what all this means, and ideal if you know how to prove it.
\\
\\
{\bf Complex analysis.} The Riemann zeta function 
\begin{equation}
\zeta(s) = 1 + \frac{1}{2^s} + \frac{1}{3^s} + \cdots
\end{equation}
defines an analytic function when the infinite sum above converges, i.e., when $\Re(s) > 1$. It is a nontrivial and very useful fact
that this function has an {\itshape analytic continuation} to the whole complex plane; 
i.e., that there is an analytic function $\zeta(s)$ which is given by the infinite
sum above when the infinite sum is defined.

It would be good to know what this means, and even better to know why this is surprising. (We will cover this for people who aren't familiar
with this.) Riemann's proof of the analytic continuation
is one of the cornerstones of analytic number theory and we will present the proof in detail.

Another example is the contour integral
\begin{equation} \int_{2 - i \infty}^{2 + i \infty} X^s \frac{ds}{s},
\end{equation}
where $X$ is a positive real number, and 
the integral is taken over the real line $\Re(s) = 2$. The first order of business (which we will briefly cover) 
is to understand what the above notation means; i.e., what is a ``contour integral''? If that is familiar, why does it converge,
but not absolutely? Finally, can you evaluate it?
\\
\\
{\bf Asymptotic analysis.} It will simplify many of the tedious details in our proofs to use the language of asymptotic analysis,
big-O notation, and implied constants. If $f(x)$ and $g(x)$ are two real-valued functions, we say that $f(x) = O(g(x))$ or
(equivalently) $f(x) \ll g(x)$ if $|f(x)| < C g(x)$ for some constant $C$. Commonly, we instead require the weaker condition
that there exists some $X, C > 0$ such that $|f(x)| < C g(x)$ for all $x > X$.

Can you prove the following?
\begin{equation}
1 + x + 10 x^2 \ll x^3,
\end{equation}
\begin{equation}
x^{100} \ll e^x,
\end{equation}
\begin{equation}
\frac{1}{\sqrt{x}} + \log(x) + \exp((\log x)^{1/2}) \ll x.
\end{equation}
Fluency in proving these sorts of inequalities are a requirement in analytic number theory, and developing this fluency
will be one of the aims of our course.

\end{document}


