\documentclass[12pt]{article}
\usepackage{amssymb,amsmath,amsthm}
\usepackage{enumerate, verbatim, url}
\textwidth 7.0in
\oddsidemargin -0.4in
\evensidemargin -0.4in
\textheight 9.0in 
\pagestyle{empty}
\begin{document}
% 8.5in paper width -2x1in margin = 6.5in text width
\setlength{\topmargin}{-2mm}



% 11in paper height -2x1in margin = 9in text height


\begin{center}{\bf Homework 11 - Analytic number theory}
\end{center}
\begin{center}Frank Thorne, thornef@mailbox.sc.edu
\end{center}
\begin{enumerate}
\item (8 points)
If $N \not \equiv 0 \mod p$, compute the number of triples $(n_1, n_2, n_3) \in (\mathbb{Z}/p)^3$
such that $n_1 + n_2 + n_3 = N$ (modulo $p$) and none of the $n_i$ are zero.

Do the same calculation if $N = 0$.

In light of these calculations, explain the constant $\mathfrak{S}(N)$ appearing
in Vinogradov's three primes theorem.

\item (10 points)
Let $\{a_1, a_2, \cdots a_k\}$ be any $k$-tuple of positive integers. Conjecture
how many integers $n$ there are less than $x$ such that the $n + a_i$ are simultaneously
prime. Your conjecture should include a qualitative description of when your answer
is zero (or finite). 

For additional points, test your conjecture and report the results.

\end{enumerate}

\end{document}
