\documentclass[12pt]{article}
\usepackage{amssymb,amsmath,amsthm}
\usepackage{enumerate, verbatim, url}
\textwidth 7.0in
\oddsidemargin -0.4in
\evensidemargin -0.4in
\textheight 9.0in 
\pagestyle{empty}
\begin{document}
% 8.5in paper width -2x1in margin = 6.5in text width
\setlength{\topmargin}{-2mm}



% 11in paper height -2x1in margin = 9in text height


\author{Term project.}
\begin{center}{\bf Term project - Analytic number theory}
\end{center}
\begin{center}Frank Thorne, thornef@webmail.sc.edu
\end{center}

(100 points.) Learn something about analytic number theory, or a related subject; write a paper
of 5 pages or more on the subject, and give a 20-minute talk on it in class or in the number theory seminar.

Group submissions are welcome; please write a joint paper of at least 5 pages per author, and each author
should speak for 20 minutes (so you can address multiple aspects of a subject).

You are very much welcome to get suggestions from Michael Filaseta, Ogy Trifonov, or Matt Boylan (or from anyone
else in the department whose interests overlap with analytic number theory). {\bf Beware:} If you speak on 
something I don't know well, I will ask a ton of questions.

The topic is up to you. The following are some suggestions.
\begin{enumerate}
\item There are {\bf a lot} of topics in Davenport which we skipped or glossed over, and {\bf all of them}
are interesting. Pick your favorite.

If you have the nerve, crack open Iwaniec and Kowalski's book and take your pick of topics. (If you get through
a half dozen pages of that, you're doing well.)

\item Learn something about the {\itshape pretentious} approach to analytic number theory, being
developed by Granville and Soundararajan. (Ask me for a copy of their notes)

\item {\itshape Sieve methods} are a powerful and versatile way to bound the number of primes in
various sequences. The easiest place is probably the book by Cojocaru and Murty ({\itshape An Introduction to
Sieve Methods and Their Applications}). A good project would be to explain the Selberg sieve.

You might crack the covers of {\itshape Opera de Cribro} by Friedlander and Iwaniec if you are feeling
(extremely) ambitious.

\item {\itshape Additive Combinatorics} is a very interesting subject, addressing the problem of finding
``additive structure'' in sets such as $\mathbb{Z}/N$. An excellent place to start is 
\url{http://math.stanford.edu/~ksound/Notes.pdf}; read the introduction and the proof of Roth's
theorem. 

\item The {\itshape circle method} is a useful and versatile tool; it was used by Vinogradov to prove
that every large odd integer is the sum of three primes. See Soundararajan's additive combinatorics notes,
or the later section of Davenport.

\item Zeta and $L$-functions can be given an {\itshape axiomatic definition} (known as the {\itshape Selberg
class}). Read about this at the beginning of Chapter 5 of Iwaniec and Kowalski, {\itshape Analytic number theory}.
Then, skim through the rest of the chapter and see what kind of examples there are, and what generality
theorems can be proved in.

\item The {\itshape prime number theorem} was given an elementary proof (i.e. no complex analysis) by Erd\"os 
and Selberg. Ask me for references (I don't know a good one of the top of my head)

\item Learn about {\itshape Chebyshev's bias} (see the paper of Rubinstein and Sarnak) and learn why there
are more primes congruent to 3 mod 4 than 1. Also see work of Granville, Martin, on others on prime number
races. (Google it!)

\item If you know something about elliptic curves and/or modular forms, learn about their $L$-functions.
(Ask Matt Boylan for references.)

\item There are {\bf many} proofs of the functional equation for the Riemann zeta function. Get a copy
of Titchmarsh's book (borrow it from me) and learn a couple of them. (There are, of course, many other
goodies in Titchmarsh as well.)

\item Learn about {\itshape Maier matrices} and irregularities in the distribution of the primes.
Good references are A. Granville, {\itshape Unexpected irregularities in the distribution of prime
numbers} and K. Soundararajan, {\itshape The distribution of prime numbers}.

\end{enumerate}

\end{document}
