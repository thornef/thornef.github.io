\documentclass[12pt]{article}
\usepackage{amssymb,amsmath,amsthm}
\usepackage{enumerate, verbatim, url}
\textwidth 7.0in
\oddsidemargin -0.4in
\evensidemargin -0.4in
\textheight 9.0in 
\pagestyle{empty}
\begin{document}
% 8.5in paper width -2x1in margin = 6.5in text width
\setlength{\topmargin}{-2mm}



% 11in paper height -2x1in margin = 9in text height


\begin{center}{\bf Homework 9 - Analytic number theory}
\end{center}
\begin{center}Frank Thorne, thornef@mailbox.sc.edu
\end{center}
\begin{enumerate}
\item(5+ points)
Consider a variant Gauss sum
\begin{equation}
G(p^2) = \sum_{m = 1}^{p^2} \Big( \frac{m}{p} \Big) e(m/p^2),
\end{equation}
where $p$ is an odd prime. Is $|G(p^2)| = p$ always? Prove or disprove.
Further prove as much as you can about this Gauss sum.

\item (5 points; Iwaniec and Kowalski, (4.24))
If $u$ and $v$ are real numbers with $v$ positive, prove the Poisson summation variant
\begin{equation}
\sum_{m \in \mathbb{Z}} f(vm + u) = \frac{1}{v} \sum_{n \in \mathbb{Z}} \widehat{f}(n/v) e(un/v).
\end{equation}
(The easiest proof starts with the Poisson summation formula and does a change of variables.)

\item (5 points)
For any $N > 0$, rigorously justify the existence of the integral $\int_{- \infty}^{\infty} e(N x^2) dx$.
(Note that it does not converge absolutely.)

({\bf Bonus,} 5 points)
Evaluate it without any appeal to Gauss sums. (Warning: Off the top of my head I don't know how to do this.)

\item (5 points)
Prove, for any $\alpha \in \mathbb{R}$, and any $x > 0$, that
\begin{equation}
\sum_{n \in \mathbb{Z}} e^{-(n + \alpha)^2 \pi /x} = x^{1/2} \sum_{n \in \mathbb{Z}} e^{-n^2 \pi x + 2 \pi i n \alpha}.
\end{equation}

\item (5 points)
Prove that $\zeta(0) = -1/2$.

\item (5 points)
If $\chi$ is a primitive, nonprincipal character mod $q$, prove that
\begin{equation}
\chi(n) \tau(\overline{\chi}) = \sum_{m = 1}^q \overline{\chi}(m) e(mn/q)
\end{equation}
if $(n, q) = 1$.

\item(3 points) If $\chi$ is primitive mod $q$, prove that
\begin{equation}
L(1, \chi) = \sum_{n \leq x} \frac{\chi(n)}{n} + O(q^{1/2} \log q/x)
\end{equation}
for any $x \geq 1$ and $q \geq 1$. 

\item(4 or 8 points)
Conclude from the previous problem that
\begin{equation}
\sum_{\chi \neq \chi_0} L(1, \chi) = \phi(q) + O(q^{1/2} \log q),
\end{equation}
where the summation is over all nontrivial characters modulo $q$.
(For 4 points, assume $q$ is an odd prime; for 8 points, don't.)

\end{enumerate}

\end{document}
