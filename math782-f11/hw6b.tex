\documentclass[12pt]{article}
\usepackage{amssymb,amsmath,amsthm}
\usepackage{enumerate, verbatim, url}
\textwidth 7.0in
\oddsidemargin -0.4in
\evensidemargin -0.4in
\textheight 9.0in 
\pagestyle{empty}
\begin{document}
% 8.5in paper width -2x1in margin = 6.5in text width
\setlength{\topmargin}{-2mm}
\newtheorem{problem}{Problem}
\newtheorem{theorem}{Theorem}[section]
\newtheorem{lemma}[theorem]{Lemma}
\newtheorem{corollary}[theorem]{Corollary}
\newtheorem{proposition}[theorem]{Proposition}
\newtheorem{conjecture}{Conjecture}
\newtheorem{definition}{Definition}


% 11in paper height -2x1in margin = 9in text height


\begin{center}{\bf Homework 6b - Analytic number theory}
\end{center}
\begin{center}Frank Thorne, thornef@mailbox.sc.edu
\end{center}
\begin{center}
{\bf Due Friday, October 7}
\end{center}
{\bf
Complex analysis practice.}
If you work on this, please ignore Homework 6a.

\begin{enumerate}
\item (3 points)
If $f$ and $g$ are holomorphic functions, then so are $f + g$, $fg$ (and, by induction,
$f^n$ for any positive integer $n$), and $cf$, where $c$ is a constant.

\item (5 points)
Suppose that $f(z) = \sum_{n \geq 0} a(n) (z - z_0)^n$ converges absolutely whenever
$z - z_0 < \rho$. (In fact, if it converges throughout this region, then it converges absolutely).

Prove that $f$ is holomorphic in this region.

\item (5 points)
Prove the Cauchy-Riemann equations.

\item (3 points)
Prove that $f(z) = \overline{z}$ is not holomorphic.

\item (10 points)
{\bf Important.} {\bf Carefully} evaluate the integral
\begin{equation}
\int_{2 - i \infty}^{2 + i \infty} \frac{dz}{z^3 + 1},
\end{equation}
via the following procedure. First of all, reduce the problem to
an evaluation of 
\begin{equation}
\int_{2- i T}^{2 + i T} \frac{dz}{z^3 + 1},
\end{equation}
and explain why this is justified. Then, ``shift the contour'' to
a rectangular contour from $2 - iT$ to $A - iT$ (where $A$ is a big real number),
to $A + iT$ to $2 + iT$. Explain what ``shifting the contour'' means.
Then, bound the value of this contour from above. Your bound will
be in terms of $A$ and $T$. Prove that this can be made arbitrarily small
by choosing $A$ and $T$ appropriately.

Colloquially this is referred to as ``shifting the contour all the way to the right''.
Why is this a good way to think about the problem?

\item (10+ points)
(Some of this will be done in lecture, but not all. I haven't decided exactly what yet.
Please fill in all the gaps.)

The gamma function is defined by
\begin{equation}
\Gamma(s) = \int_0^{\infty} e^{-t} t^{s - 1} ds.
\end{equation}
Note that if $s \in \mathbb{C}$, $t^s := \exp(s \log t)$. There is no ambiguity if $t$ is positive real.

(a) Determine (with proof) the set of all $s$ for which this integral converges, and the
set of all $s$ for which it defines a holomorphic function of $s$.

(b) Integrating by parts, prove the functional equation
\begin{equation}
\Gamma(s + 1) = s \Gamma(s)
\end{equation}
in the region of absolute convergence. Explain why this continues $\Gamma(s)$
to a function holomorphic on the whole complex plane, except for simple poles
at $s = 0, -1, -2, -3, \cdots$

(c) Evalute the residues at the poles.

(d) It can be shown that
\begin{equation}
\Gamma(s) \Gamma(1 - s) = \frac{\pi}{\sin \pi s}.
\end{equation}
(For bonus points, prove this.) Using this identity, prove that $\Gamma(s)$ is never equal
to zero.

\item
I might add some more problems if I come up with anything good. (But tackling all of this
is already good.)

\end{enumerate}

\end{document}
