\documentclass[12pt]{article}
\usepackage{amssymb,amsmath,amsthm}
\usepackage{enumerate, verbatim}
\textwidth 7.0in
\oddsidemargin -0.4in
\evensidemargin -0.4in
\textheight 9.0in 
\pagestyle{empty}
\begin{document}
% 8.5in paper width -2x1in margin = 6.5in text width
\setlength{\topmargin}{-2mm}



% 11in paper height -2x1in margin = 9in text height


\begin{center}{\bf Homework 3 - Analytic number theory}
\end{center}
\begin{center}Frank Thorne, thornef@mailbox.sc.edu
\end{center}
\begin{center}
{\bf Due Friday, September 16}
\end{center}
\begin{enumerate}
\item (8 points)
Let $d(n)$ be the divisor function. Prove that $d(n) \ll n^{\epsilon}$ for any $\epsilon > 0$, with the implied
constant depending on $\epsilon$.

\item (12- points)
Find, with proof, the value of $n < 10^{50}$ such that $d(n)$ is maximized.

(Partial credit for partial results: $d(n)$ is within a factor of two of the maximum possible value, etc.)

\item (3 points)
Let $d_k(n)$ the number of ways of writing $n$ as a product of $k$ factors. Explain why
\begin{equation}
\sum_{n \geq 1} \frac{ d_k(n) }{n^s} = \zeta(s)^k
\end{equation}
for $\Re(s) > 1$.

\item (5-12 points)
For fixed $k$, determine an asymptotic formula for $\sum_{n \leq x} d_k(n)$. (Five points for an asymptotic
formula; more points for secondary terms (you can be a little bit vague here) and good error terms.

\item (5 points)
Determine a formula for $\sum_{n \leq x} \phi(n)$. You should be able to come up with a very good error term.

\item (5+ points)
Using a computer, compute $\sum_{n \leq x} \mu(n)$ for a good range of $x$ (You should be able to do it at
least to $x = 10^6$). Describe and/or graph your results. (Conjectures encouraged!)

\end{enumerate}

\end{document}
