\documentclass[12pt]{article}
\usepackage{amssymb,amsmath,amsthm}
\usepackage{enumerate, verbatim}
\textwidth 7.0in
\oddsidemargin -0.4in
\evensidemargin -0.4in
\textheight 9.0in 
\pagestyle{empty}
\begin{document}
% 8.5in paper width -2x1in margin = 6.5in text width
\setlength{\topmargin}{-2mm}



% 11in paper height -2x1in margin = 9in text height


\begin{center}{\bf Homework 2 - Analytic number theory}
\end{center}
\begin{center}Frank Thorne, thornef@mailbox.sc.edu
\end{center}
\begin{center}
{\bf Due Friday, September 9}
\end{center}
\begin{enumerate}
\item (3 points)
Describe explicitly (say, by computing tables) all of the Dirichlet characters of modulus $\leq 10$.

\item (5+ points)
Write a program in PARI/GP, Sage, Java, C, or any other computer language to test Dirichlet's theorem on 
primes in arithmetic progressions numerically. For example, compute whether there are more primes less than $X$
congruent to 1 or 3 modulo 4, for a variety of values of $X$. Turn in your code and report your findings.

(If you don't know any of these languages, I strongly recommend you learn one! PARI/GP and Sage are specialized
for mathematics, are open source, and can be downloaded for free.)

5 points for some relevant data, 5 more points if you find and describe anything ``interesting'', a further 5 points
for good guesses on rules for when and how the data is ``interesting''.

\item
(3 points) Let $\chi_4$ be the nontrivial Dirichlet character modulo 4. Prove that $L(1, \chi_4) = \frac{\pi}{4}$.

(10 points) Discover and prove an exact formula for $L(1, \chi)$ for any other nontrivial character $\chi$.

\item (3 points)
Prove that if $\delta > 0$ and $\chi$ is a Dirichlet character, then the Dirichlet $L$-function $L(s, \chi)$
converges uniformly for all complex numbers $s$ with $\Re(s) \geq \delta$.

This was basically proved in class, but the end of the proof was only sketched, so give a detailed proof of this.

\item (5 points)
If $L(1, \chi) = 0$ for some Dirichlet character $\chi$, prove that $L(s, \chi) \ll s - 1$ for $s \in (1, 2)$.

There is a proof of this on p. 6 of Davenport which you are free to give, but please use our notation (which Davenport
reverts to later in his book) and spell out more of the details.

\item (5 points)
It was proved in lecture that if $\chi$ is a real, nontrivial Dirichlet character to a prime modulus $q$, then
$\chi$ is unique and is in fact the quadratic residue symbol modulo $q$.

(a) Write down all of the four real Dirichlet characters modulo 8. 8 is of course not prime; why does the proof
from lecture fail to prove that there are only two real $\chi$ mod 8?

(b) Find an odd (but not prime) modulus $q$ for which there are more than two real characters modulo $q$, and describe
all of these characters.
\end{enumerate}

\end{document}
