\documentclass[12pt]{article}
\usepackage{amssymb,amsmath,amsthm}
\usepackage{enumerate, verbatim, url}
\textwidth 7.0in
\oddsidemargin -0.4in
\evensidemargin -0.4in
\textheight 9.0in 
\pagestyle{empty}
\begin{document}
% 8.5in paper width -2x1in margin = 6.5in text width
\setlength{\topmargin}{-2mm}



% 11in paper height -2x1in margin = 9in text height


\begin{center}{\bf Homework 10 - Analytic number theory}
\end{center}
\begin{center}Frank Thorne, thornef@mailbox.sc.edu
\end{center}
\begin{enumerate}
\item (15 points; do part for partial credit)
Consider the integral
\begin{equation}\label{eqn_int}
\int_{c - i \infty}^{c + i \infty} \zeta(s) \frac{X^s}{s} ds.
\end{equation}

(a) Shift the contour to the following sequence of line segments: from $c - i \infty$ to
$c - i T$, to $a - i T$, to $a + iT$, to $c + iT$, to $c + i \infty$, where $a < 1$.
Assuming that the resulting integral is an error term, evaluate the main term of \eqref{eqn_int}
(the integrand has a residue).

(b) Bound the resulting integral in terms of $T$. The portions on the line $\Re(s) = c$
can be estimated as in lecture (the error term for truncations of Perron's formula).
For the rest, you can take $a < 0$ and use the functional equation, or alternatively take
$a = 1/2$ and use the convexity bound.

(c) Choosing $T$ optimally, obtain a good error term for this integral.

(d) In light of Perron's formula, you have proved an arithmetic density theorem
for the most important (and most widely studied) sequence in analytic number theory.
Write down the profundity which you have now just proved.

(Note: You may have also seen other proofs of your conclusion in (d).)

\item (5 points each part)
[add the calculations from Montgomery and Vaughan!]

\item (5 points)
Given
\begin{equation}
1 + \frac{1}{4} + \frac{1}{9} + \frac{1}{16} + \cdots = \frac{\pi^2}{6},
\end{equation}
\begin{equation}
1 + \frac{1}{16} + \frac{1}{81} + \frac{1}{256} + \cdots = \frac{\pi^4}{90},
\end{equation}
calculate $\zeta(-1)$ and $\zeta(-3)$. Is
\begin{equation}
1 + \frac{1}{8} + \frac{1}{27} + \frac{1}{64} + \cdots = \frac{\pi^3}{n}
\end{equation}
for some integer $n$?

\item (7 points)
Obtain an explicit constant $C$ for which the Polya-Vinogradov inequality holds in the
form
$$\sum_{n = M + 1}^{M + N} \chi(n) < C q^{1/2} \log q$$
for any nonprincipal character $\chi$ modulo $q$. (Recall that if $\chi$ is primitive,
we can take $C = 1$.)

\item
By request I will add some further exercises on Fourier analysis; but these will have
to wait until I get my hands on a good book on the subject.

\end{enumerate}

\end{document}
