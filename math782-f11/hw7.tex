\documentclass[12pt]{article}
\usepackage{amssymb,amsmath,amsthm}
\usepackage{enumerate, verbatim, url}
\textwidth 7.0in
\oddsidemargin -0.4in
\evensidemargin -0.4in
\textheight 9.0in 
\pagestyle{empty}
\begin{document}
% 8.5in paper width -2x1in margin = 6.5in text width
\setlength{\topmargin}{-2mm}



% 11in paper height -2x1in margin = 9in text height


\begin{center}{\bf Homework 7 - Analytic number theory}
\end{center}
\begin{center}Frank Thorne, thornef@mailbox.sc.edu
\end{center}
\begin{center}
{\bf Due Friday, October 14}
\end{center}
\begin{enumerate}
\item(7 points)
Prove that for $0 < a < 1$, we have
\begin{equation}
\int_0^{\infty} \frac{v^{a - 1}}{1 + v} dv = \frac{\pi}{\sin(\pi a)}.
\end{equation}
Here is one way to prove it. By change of variables, the integral is equal to
\begin{equation}
\int_{-\infty}^{\infty} \frac{e^{ax}}{1 + e^x} dx.
\end{equation}
To evaluate the latter integral, evaluate it from $-R$ to $R$, and let $R \rightarrow \infty$.
To do this, consider a contour integral from $-R$ to $R$, up to $R + 2\pi i$, left
to $-R + 2 \pi i$, and back down to $-R$. The sides should go to 0 as $R$ gets big, and
there is a relation between the top integral and the bottom integral. Use should
use Cauchy's residue theorem to evaluate the integral over the entire contour.

\item{5 points}

We proved that
\begin{equation}
\Gamma(s) = \int_{n = 0}^{\infty} \frac{(-1)^n}{n! (n + s)} 
+ \int_1^{\infty} e^{-t} t^{s - 1} dt
\end{equation}
for $\Re(s) > 0$. Prove that this converges for all values of $s$
other than $0, -1, -2, \cdots$. Further prove that it converges uniformly
in any closed disc which avoids the poles.

It therefore follows from standard theorems of complex analysis
that $\Gamma(s)$ is a meromorphic function of $s$.

\item(7 points)
Prove that $\frac{1}{\Gamma(s)}$ is of order 1; namely, prove the bound
\begin{equation}
\frac{1}{|\Gamma(s)|} \ll \exp(C |s| \log |s|)
\end{equation}
for some constant $C$. 

One way to prove this is using the identity in the previous problem.
You can show that the two terms above are both of order 1. For the integral,
show it first for positive real $s$, then for complex $s$; finally you will
use the identity $\Gamma(s) \Gamma(1 - s) = \frac{\pi}{\sin(\pi s)}$.

For the sum, you can argue separately for $|\Im(s) \geq 1|$ and $|\Im(s) \leq 1|$.

\item(5 points)
Starting from Hadamard's product formula, prove the identity
\begin{equation}
\frac{\sin(\pi z)}{\pi} = z \prod_{n = 1}^{\infty} \bigg(1 - \frac{z^2}{n^2} \bigg).
\end{equation}

\item(5-10 points)
It was claimed that an integral function of order 1 has $\ll R^{1 + \epsilon}$
zeroes $\rho$ with $|\rho| < R$.

Read the proof of this, in pp. 75-76 of Davenport or elsewhere, and explain it in your
own words. (A 10-point answer will include a justification of Jensen's formula.)



\end{enumerate}

\end{document}
