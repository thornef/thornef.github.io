\documentclass[12pt]{article}
\usepackage{amssymb,amsmath,amsthm}
\usepackage{enumerate, verbatim}
\textwidth 7.0in
\oddsidemargin -0.4in
\evensidemargin -0.4in
\textheight 9.0in 
\pagestyle{empty}
\begin{document}
% 8.5in paper width -2x1in margin = 6.5in text width
\setlength{\topmargin}{-2mm}



% 11in paper height -2x1in margin = 9in text height


\begin{center}{\bf Homework 4 - Analytic number theory}
\end{center}
\begin{center}Frank Thorne, thornef@mailbox.sc.edu
\end{center}
\begin{center}
{\bf Due Friday, September 23}
\end{center}
\begin{enumerate}
\item (5 points)
(a) Prove (as was taken for granted in lecture) that if $f$ and $g$ are multiplicative
functions, then so is their convolution $f \ast g$.

(b) If $f$ is multiplicative and $g$ is not, must $f \ast g$ be multiplicative? Prove or
find a counterexample.

\item (5 points)
Prove that $\sum_n d(n)^2 n^{-s} = \frac{\zeta^4(s)}{\zeta(2s)}.$

\item (5 points)
Prove that $\sum_n d(n^2) n^{-s} = \frac{\zeta^3(s)}{\zeta(2s)}$.

\item (5 points)
(Trick question. Explain.) 
Write out the character tables for all primitive real characters to the following moduli:
14, 15, 16, 20, 22, 24, 25, 27.

\item (7 points)
Write $g(n)$ be the number of primitive (not necessarily real) characters modulo $n$.
Prove an explicit formula for $g(n)$. {\bf No messy computations allowed.}
\end{enumerate}

\end{document}
