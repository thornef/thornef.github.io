\documentclass[12pt]{article}
\usepackage{amssymb,amsmath,amsthm}
\usepackage{enumerate, verbatim, url}
\textwidth 7.0in
\oddsidemargin -0.4in
\evensidemargin -0.4in
\textheight 9.0in 
\pagestyle{empty}
\begin{document}
% 8.5in paper width -2x1in margin = 6.5in text width
\setlength{\topmargin}{-2mm}



% 11in paper height -2x1in margin = 9in text height


\begin{center}{\bf Homework 6a - Analytic number theory}
\end{center}
\begin{center}Frank Thorne, thornef@mailbox.sc.edu
\end{center}
\begin{center}
{\bf Due Friday, October 7}
\end{center}
{\bf
Some challenge problems.
If you prefer complex analysis review or practice, please ignore this assignment and work on
Homework 6b instead.}

\begin{enumerate}
\item (10 points)
Find all the automorphs of $x^2 + xy + y^2$. Observe that they form a group.

Write down an explicit, {\itshape natural} bijection between these matrices and the sixth roots of unity.
No points for just showing that one of your matrix has exact order 6! Explain how, and why, it corresponds
to a root of unity.

\item (12 points)
Prove that for any discriminant $d > 0$, there are infinitely many integer solutions to Pell's equation
$x^2 - d y^2 = 4$. One path to a solution follows the steps in Andrew Granville's notes; see the bottom
of p. 13 in 
\url{http://www.dms.umontreal.ca/~andrew/Courses/Chapter4.pdf}.

\item
Granville's notes are an interesting starting point for a term project. One natural project (if you 
know some algebraic number theory) is to explain the correspondence between binary quadratic forms
and quadratic fields in much greater detail.

\end{enumerate}

\end{document}
