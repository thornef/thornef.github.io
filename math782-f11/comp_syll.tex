\documentclass[12pt]{article}
\usepackage{amssymb,amsmath,amsthm}
\usepackage{enumerate, verbatim}
\textwidth 7.0in
\oddsidemargin -0.4in
\evensidemargin -0.4in
\textheight 9.0in 
\pagestyle{empty}
\begin{document}
% 8.5in paper width -2x1in margin = 6.5in text width
\setlength{\topmargin}{-2mm}



% 11in paper height -2x1in margin = 9in text height


\begin{center}{\bf Comprehensive exam syllabus for Math 782, Analytic Number Theory (Fall 2011)}
\end{center}
\begin{center}Frank Thorne, thorne@math.sc.edu
\end{center}

The exam will consist of {\bf two parts}. The first part will consist of {\bf core material} that is
foundational in analytic number theory. All students should thoroughly master this material. The second
part will consist of additional questions which test additional material covered during the course
(and possibly outside it), and which will give you a chance to show off what else you know. Some part II 
questions may be individualized -- you are invited to e-mail me (at least two weeks before the exam, please!) 
and tell me what topics beyond Part I you have studied.
\\
\\
The cutoff for passing will be roughly 80 points. Part I will be worth 100 points, and Part II will be worth
a substantial number of additional points. 
You should plan on doing most or all of Part I correctly, and getting at least something
in Part II.
\\
\\
{\bf Formulas will be provided where appropriate}, including Stirling's formula and the Hadamard product formula
for the zeta function.
\\
\\
\begin{center}
{\bf Syllabus for Part I}
\end{center}
\begin{enumerate}
\item
Definition of the Riemann zeta function. The Dirichlet series and Euler product. Elementary manipulation
such as taking logarithmic derivatives, etc. Prove there are only finitely many primes.

\item
Arithmetic functions, especially $\Lambda(n)$ and $\mu(n)$. Be able to prove 
identities such as $\log(n) = \sum_{d | n} \Lambda(d)$.

\item
Partial summation (or, equivalently, Stieltjes integration). Equivalence of various forms of the prime
number theorem. Proof that $\sum_{p \leq x} \frac{1}{p}$ is asymptotic to $\log \log x$.

\item
Dirichlet characters. Computations (e.g. list all the Dirichlet characters of any modulus), orthogonality
relations (with proofs), real Dirichlet characters, primitive and induced characters.

\item
Dirichlet $L$-functions. Prove of the prime number theorem for arithmetic progressions, given $L(1, \chi) \neq 0$.
Prove that $L(1, \chi) \neq \infty$.

\item
The divisor function $d(n)$. Estimates for $\sum_{n \leq x} d(n)$. (Know at least one proof)

\item
M\"obius inversion (with proof). Applications such as bounding the number of primes in $[x, x + y]$; counting
squarefree integers $\leq x$ (or related).

\item
Convolution of arithmetic sequences, and relation to Euler products.

\item
Additional arithmetic functions such as the Euler $\phi$ function and the sum of divisors function $\sigma(n)$.
Proofs of identities tying these functions together.

\item
Positive definite integral binary quadratic forms. The equivalence relation. Discriminants of quadratic forms. Automorphs. 
Definition of the class
number. Computation of class numbers.

\item
Elementary complex analysis. Know what a meromorphic function is, what a pole is (and what the order of a pole is); 
know and be able to apply Cauchy's residue theorem. Evaluation of (relatively simple) contour integrals.

\item
Understand the statement of Riemann's theorem on analytic continuation and functional equation of the zeta
function. (Proofs won't be covered in Part I.)

\item
Understand what the {\itshape explicit formula} is. Explain, in broad terms, how it is proved and what it says
about the primes. Understand what Perron's formula says.

\item
Gauss sums. Evaluation of $G^2$. 

\item
Good homework problems to review (these or similar would be fair game for Part I on the exam): 
1 (1, 2, 3, 6, 7), 2 (1, 4, 6), 3 (3, 5), 4 (1-4), 5 (2-5), 6b (1-5), 7 (1, 2), 8 (1), 9 (1, 6-8).
 
\end{enumerate}
This leaves {\bf a lot} of material for Part II, including the proofs of analytic statements about zeta and
$L$-functions (including the prime number theorem), the Poisson summation formula, Dirichlet's class number formula,
Polya-Vinogradov, the circle method, the Riemann hypothesis and its consequences, and much more! You certainly
don't have to know all of this, but you should know at least a little bit of it. There will be enough of a variety
of questions that you should feel safe picking and choosing.

\end{document}
