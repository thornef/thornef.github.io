\documentclass[12pt]{article}
\textwidth 7.0in
\oddsidemargin -0.4in
\evensidemargin -0.4in
\textheight 9.0in 
\pagestyle{empty}
\usepackage{enumerate}
\begin{document}
% 8.5in paper width -2x1in margin = 6.5in text width
\setlength{\topmargin}{-2mm}



% 11in paper height -2x1in margin = 9in text height


\begin{center}{\bf Practice Examination - Math 142, Frank Thorne (thorne@math.sc.edu)}
\end{center}
\begin{center}
{\bf The Exam is Thursday, October 24, 2013}
\end{center}

{\bf Instructions and Advice:} 

\begin{itemize}
\item
There are eight questions, some of which are shorter than others.
\item
You are welcome to as much scratch paper as you need. Turn in everything you want graded,
and throw away everything you do not want graded.
\item
{\bf Draw pictures where appropriate.} If you have any doubt, then a picture is appropriate.
\item
Be clear, write neatly, explain what you are doing, and show your work. {\bf This is especially
important for earning partial credit} in case your work contains one or more mistakes.
Be warned that {\bf work I cannot understand will not receive any credit.}
\item
75 minutes is a long time. Don't dilly-dally, but don't rush. {\bf You are strongly advised
to take the entire 75 minutes to complete the examination.} If you finish early, you have the
opportunity to check your work.
\item
Please work without books, notes, calculators, or any assistance from others. 
\item
I will be at the front of the room; if you have
any questions, feel free to ask me. 
\end{itemize}

\begin{center}
{\bf GOOD LUCK!}
\end{center}
\newpage

\begin{enumerate}[(1)]
\item
(12 points) Find the volume of sphere of radius $6$.

({\itshape Warning:} You will not get credit for remembering the formula
and plugging in $6$. A correct answer will carry out the calculus
computation. You {\itshape may} use the fact that a circle of radius $r$ has
area $\pi r^2$, without further justification.)

Draw a picture illustrating your computation.

\item
(10 points) The region bounded by $y = \frac{1}{9} x^2$, $x = 3$, and $y = 0$ 
is rotated around the
$y$-axis. Find the volume of the resulting solid.

As part of your answer, sketch the region, the resulting solid, and a typical disk or washer.

\item
(10 points) Explain what it means for a curve to be defined by parametric equations. Give and graph
an example. (Your example should be different than other questions on this exam.)

\item
(10 points) Sketch the curve given by the equations $x = 2 \cos t$, $y = t - \cos t$ for $0 \leq t \leq 2\pi$.

(10 points) In addition, find the equation of the tangent line when $t = \pi/3$, and sketch it on your graph.

\item
(10 points) Three graphs are given, along with parametric equations for them (in a different order). 
Match the graphs with the equations,
and give reasons for your choices.
\item
(10 points) Plot the point whose polar coordinates are $(-3, \pi/4)$. In addition, find Cartesian
coordinates of this point, and find another set of polar coordinates $(r, \theta)$ for the
same point with $r > 0$.
\item
(10 points) Sketch the curve with the polar equation $r = 2 \cos 2 \theta$ for $0 \leq \theta \leq 2 \pi$.

(8 points) Also, find the slope of the
tangent line when $\theta = \pi/3$.

\item
(10 points) A graph of the curve $r = 2 + 2 \cos \theta$ is provided. Write down a definite integral
which represents the area of the curve. In addition, use the graph to give a rough estimate for the value of this
integral.

\end{enumerate}

\end{document}
