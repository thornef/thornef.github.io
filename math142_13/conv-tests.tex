\documentclass[12pt]{article}
\textwidth 7.0in
\oddsidemargin -0.4in
\evensidemargin -0.4in
\textheight 9.0in 
\pagestyle{empty}
\usepackage{enumerate, url}
\newcommand{\dx}{dx}
\begin{document}

% 8.5in paper width -2x1in margin = 6.5in text width
\setlength{\topmargin}{-2mm}



% 11in paper height -2x1in margin = 9in text height


\begin{center}{\bf Convergence Tests for Math 142 --- Frank Thorne (thorne@math.sc.edu)}
\end{center}
\begin{center}
{\bf This sheet will be provided to you on the exam.}
\end{center}
\vskip 10pt
\vskip 10pt
The convergence tests below concern infinite series
\[
\sum_{n = 1}^{\infty} a_n = a_1 + a_2 + a_3 + \cdots
\]
Sometimes we write $f(n)$ instead of $a_n$.
\vskip 10pt
{\bf Guidelines:}
{\bf 1.} We wrote down a series starting at $a_1$, but in fact the starting
value isn't important. If we have a series $\sum_{n = 0}^{\infty} a_n$
or $\sum_{n = 91887}^{\infty} a_n$ or $\sum_{n = -5189234}^{\infty} a_n$ then
the convergence tests work equally well.
\vskip 10pt
{\bf 2.} Only the {\itshape eventual behavior} of the series matters. In {\bf all}
of the convergence tests below, it is permissible to pick some $N$ and only
look at the $a_n$ with $n \geq N$.
\vskip 10pt
{\bf 3.} Some of the convergence tests assume that all of the $a_n$ are
nonnegative. They also work if {\bf all} of the $a_n$ are nonpositive, but not
necessarily if they have mixed signs.
\vskip 20pt
{\bf 1. The $n$th term test.} If it is not true that $\lim_{n \rightarrow \infty} a_n = 0$, then
$\sum_{n = 1}^{\infty} a_n$ diverges.
\vskip 10pt
{\bf 2. The integral test.} Suppose that $f(x)$ is a positive, continuous, and decreasing function
for $x \geq 1$.

Then, $\sum_{n = 1}^{\infty} f(n)$ converges if and only if $\int_{x = 1}^{\infty} f(x) dx$ converges.

Also, if we estimate $\sum_{n = 1}^{\infty} f(n)$ by
\[
f(1) + f(2) + \cdots + f(k) + \int_{k + 1}^{\infty} f(x) dx,
\]
then the estimate is too low by somewhere between $0$ and $f(k + 1)$. (You should know how to draw pictures
which explain why this is true.)
\vskip 10pt
{\bf 3. The alternating series test.} Given an alternating series
\[
\sum_{n = 1}^{\infty} (-1)^{n + 1} a_n = a_1 - a_2 + a_3 - a_4 + \cdots,
\]
where (1) all the $a_i$ are positive, (2) $a_i > a_{i + 1}$ for each $i$, and $\lim_{n \rightarrow \infty} a_n = 0$,
then the series converges.

If you approximate the series by the first $k$ terms, the error is between $0$ and the $k + 1$st term.

\vskip 10pt
{\bf 4. The comparison test.} Suppose you have two series
\[ \sum_{n = 1}^{\infty} a_n, \ \ \ \sum_{n = 1}^{\infty} b_n,\]
where all the $a_n$ and $b_n$ are nonnegative.

\begin{itemize}
\item
If $a_n \leq b_n$ for each $n$, and $\sum_{n = 1}^{\infty} b_n$ converges, then
$\sum_{n = 1}^{\infty} a_n$ also converges and $\sum_{n = 1}^{\infty} a_n \leq
\sum_{n = 1}^{\infty} b_n$.
\item
If $a_n \geq b_n$ for each $n$, and $\sum_{n = 1}^{\infty} b_n$ diverges, then
$\sum_{n = 1}^{\infty} a_n$ also diverges.
\end{itemize}
\vskip 10pt
{\bf 5. The limit comparison test.}
Suppose you have two series
\[ \sum_{n = 1}^{\infty} a_n, \ \ \ \sum_{n = 1}^{\infty} b_n,\]
where all the $a_n$ and $b_n$ are nonnegative.

If $\lim_{n \rightarrow \infty} \frac{a_n}{b_n}$ exists, and equals some
{\itshape positive} number other than $0$ or $\infty$, then either both
series converge or both series diverge.
\vskip 10pt
{\bf 6. The absolute convergence test.}
If a series converges absolutely, then it converges.
\vskip 10pt
{\bf 7. The ratio test.}
Suppose that $\lim_{n \rightarrow \infty} \bigg| \frac{ a_{n + 1} }{a_n} \bigg|$
exists and equals some number $L$.

\begin{itemize}
\item
If $L < 1$, then $\sum_{n = 1}^{\infty} a_n$ converges absolutely.
\item
If $L > 1$, then $\sum_{n = 1}^{\infty} a_n$ diverges.
\end{itemize}

\end{document}
