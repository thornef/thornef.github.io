\documentclass[12pt]{article}
\textwidth 7.0in
\oddsidemargin -0.4in
\evensidemargin -0.4in
\textheight 9.0in 
\pagestyle{empty}
\usepackage{enumerate}
\begin{document}
% 8.5in paper width -2x1in margin = 6.5in text width
\setlength{\topmargin}{-2mm}



% 11in paper height -2x1in margin = 9in text height


\begin{center}{\bf Final Examination - Math 142, Frank Thorne (thorne@math.sc.edu)}
\end{center}
\begin{center}
{\bf Thursday, December 12, 2013}
\end{center}

{\bf Instructions and Advice:} 

\begin{itemize}
\item
There are fifteen questions, some of which are shorter than others.
\item
You are welcome to as much scratch paper as you need. Turn in everything you want graded,
and throw away everything you do not want graded.
\item
{\bf Draw pictures where appropriate.} If you have any doubt, then a picture is appropriate.
\item
Be clear, write neatly, explain what you are doing, and show your work. {\bf This is especially
important for earning partial credit} in case your work contains one or more mistakes.
Be warned that {\bf work I cannot understand will not receive any credit.}
\item
150 minutes is a long time. Don't dilly-dally, but don't rush. {\bf You are strongly advised
to take the entire 150 minutes to complete the examination.} If you finish early, you have the
opportunity to check your work.
\item
You are welcome and encouraged to refer to the list of convergence tests provided with the exam.
\item
Please work without books, notes, calculators, or any assistance from others. 
\item
I will be at the front of the room; if you have
any questions, feel free to ask me. 
\end{itemize}

\begin{center}
{\bf GOOD LUCK!}
\end{center}
\newpage

\begin{enumerate}[(1)]
\item
Evaluate
\[
\int_1^3 r^3 \ln(r) dr.
\]
\item
Evaluate
\[
\int_{\sqrt{2}/3}^{2/3} \frac{dx}{x^5 \sqrt{9x^2 - 1}}.
\]
\item
Evaluate
\[
\int \frac{ax}{x^2 - bx} dx,
\]
where $a$ is a nonzero constant and $b$ is a positive constant.
\item
Evaluate
\[
\int_{- \infty}^{\infty} xe^{-x^2} dx.
\]
\item
Sketch the region enclosed by the curves listed below. Decide whether to integrate with
respect to $x$ or $y$. Draw a typical approximating rectangle and label its height and width. Then,
find the area of the region.
\[
y = 1 + \sqrt{x}, \ y = (3 + x)/3
\]
\item
The integral
\[
\pi \int_0^{\pi/2} \cos^2 x dx
\]
represents the area of a solid. Describe the solid.
\item
Give parametric equations for a line and an ellipse (other than a circle) of your choice.
Graph your curves.
\item
Find an equation of the tangent line to the curve at the given point by two methods: (a)
without eliminating the paramter, and (b) by first eliminating the parameter.
\[
x  = 1 + \ln(t), \ y = t^2 + 2.
\]
\item
Sketch the curve given by the polar coordinates $r = 1 - 2 \sin \theta$.
\item
Let a sequence be determined by the rule $a_n = \arctan(2n)$. Determine whether it converges
or diverges. If it converges, find the limit.
\item
Use the integral test to determine whether the series is convergent or divergent.
\[
1 + \frac{1}{3} + \frac{1}{5} + \frac{1}{7} + \frac{1}{9} + \cdots
\]
If the series diverges, then draw a graph which
represents both the series and the integral you're comparing it to.

If the series converges, give upper and lower bounds on the value of your series
which are guaranteed to be accurate within $0.01$. Draw a graph which represents your
lower bound.
\item
Using the comparison test, or otherwise, determine whether the following series converges or diverges:
\[
\sum_{n = 1}^{\infty} \frac{n^2 - 1}{\sqrt{3n^4 + 1}}.
\]
\item
Test the alternating series for convergence or divergence.
\[
\frac{1}{\sqrt{2}} - \frac{1}{\sqrt{3}} + \frac{1}{\sqrt{4}} - \frac{1}{\sqrt{5}} + \frac{1}{\sqrt{6}} - \cdots
\]
\item
What is the radius of convergence of a power series? How do you find it?
\item
Explain why the Taylor series for $e^x$ gives you a formula for $e$.
\end{enumerate}

\end{document}
