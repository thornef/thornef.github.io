\documentclass[12pt]{article}
\textwidth 7.0in
\oddsidemargin -0.4in
\evensidemargin -0.4in
\textheight 9.0in 
\pagestyle{empty}
\usepackage{enumerate}
\begin{document}
% 8.5in paper width -2x1in margin = 6.5in text width
\setlength{\topmargin}{-2mm}



% 11in paper height -2x1in margin = 9in text height


\begin{center}{\bf Examination 2 - Math 142, Frank Thorne (thorne@math.sc.edu)}
\end{center}
\begin{center}
{\bf Thursday, October 24, 2013}
\end{center}

{\bf Instructions and Advice:} 

\begin{itemize}
\item
There are nine questions, some of which are shorter than others.
\item
You are welcome to as much scratch paper as you need. Turn in everything you want graded,
and throw away everything you do not want graded.
\item
{\bf Draw pictures where appropriate.} If you have any doubt, then a picture is appropriate.
\item
Be clear, write neatly, explain what you are doing, and show your work. {\bf This is especially
important for earning partial credit} in case your work contains one or more mistakes.
Be warned that {\bf work I cannot understand will not receive any credit.}
\item
75 minutes is a long time. Don't dilly-dally, but don't rush. {\bf You are strongly advised
to take the entire 75 minutes to complete the examination.} If you finish early, you have the
opportunity to check your work.
\item
Please work without books, notes, calculators, or any assistance from others. 
\item
I will be at the front of the room; if you have
any questions, feel free to ask me. 
\end{itemize}

\begin{center}
{\bf GOOD LUCK!}
\end{center}
\newpage

\begin{enumerate}[(1)]
\item
(14 points) Find the volume of a circular cone of radius $4$ and height $6$.

({\itshape Warning:} You will not get credit for remembering the formula
and plugging in $4$ and $6$. A correct answer will carry out the calculus
computation. You {\itshape may} use the fact that a circle of radius $r$ has
area $\pi r^2$, without further justification.)

Draw a picture illustrating your computation.

\item
(6 points) Say what it means for a curve to be defined by parametric equations. Explain why the next curve
is an example.

(8 points) Sketch the curve given by the equations $x = e^{-t} + t$, $y = e^t - t$ for $-2 \leq t \leq 2$.

(8 points) In addition, find the equation of the tangent line when $t = 1$, and sketch it on your graph.

\item
(12 points) Graphs of two functions $x = f(t)$ and $y = g(t)$ are given on the next page. Use these graphs to sketch
the parametric curve $x = f(t), y = g(t)$, and write a few sentences to explain what you
are doing. 
Indicate with arrows
the direction in which the curve is traced as $t$ increases.
\item
(12 points) Plot the point whose polar coordinates are $(1, -5 \pi/6)$. In addition, find Cartesian
coordinates of this point, and find another set of polar coordinates $(r, \theta)$ for the
same point with $r < 0$.
\item
(14 points) Sketch the curve with the polar equation $r = 1 - 3 \cos \theta$ for $0 \leq \theta \leq \pi/2$.

\item
A graph of the curve $r = 3 + 2 \sin \theta$ is provided on the next page.

(8 points) Write down a definite integral which represents the area inside the curve. 

(6 points) Estimate (by using the graph, or by other means) the numerical area inside this curve.

(12 points) Find the slope of the tangent line when $\theta = \pi/3$, and graph the tangent line on the provided
graph.

({\bf Extra Credit.} 6 points) Find the exact area bounded by the curve.

\end{enumerate}

\end{document}
