\documentclass[12pt]{article}
\textwidth 7.0in
\oddsidemargin -0.4in
\evensidemargin -0.4in
\textheight 9.0in 
\pagestyle{empty}
\usepackage{enumerate}
\begin{document}
% 8.5in paper width -2x1in margin = 6.5in text width
\setlength{\topmargin}{-2mm}



% 11in paper height -2x1in margin = 9in text height


\begin{center}{\bf Examination 3 - Math 142, Frank Thorne (thorne@math.sc.edu)}
\end{center}
\begin{center}
{\bf Thursday, November 21, 2013}
\end{center}

{\bf Instructions and Advice:} 

\begin{itemize}
\item
You are welcome to as much scratch paper as you need. Turn in everything you want graded,
and throw away everything you do not want graded.
\item
{\bf Draw pictures where appropriate.} If you have any doubt, then a picture is appropriate.
\item
Be clear, write neatly, explain what you are doing, and show your work. {\bf This is especially
important for earning partial credit} in case your work contains one or more mistakes.
Be warned that {\bf work I cannot understand will not receive any credit.}
\item
75 minutes is a long time. Don't dilly-dally, but don't rush. {\bf You are strongly advised
to take the entire 75 minutes to complete the examination.} If you finish early, you have the
opportunity to check your work.
\item
This exam is accompanied by a list of convergence tests which you should freely refer to.
Please work without books, notes, calculators, or any assistance from others. 
\item
I will be at the front of the room; if you have
any questions, feel free to ask me. 
\end{itemize}

\begin{center}
{\bf GOOD LUCK!}
\end{center}
\newpage

\begin{enumerate}[(1)]
\item
(10 points) Does the sequence
\[
a_n = \ln(n + 2) - \ln(n)
\]
converge or diverge? If it converges, find the limit. If it diverges, explain why.
\item
(10 points) Does the sequence
\[
a_n = \sqrt{ \frac{ n + 1} {9n + 1} }
\]
converge or diverge? If it converges, find the limit. If it diverges, explain why.

\item
(10 points) Does the series
\[
3 - 4 + \frac{16}{3} - \frac{64}{9} + \cdots
\]
converge or diverge? If it converges, find its sum.
\item
(a.) (14 points) 
By using the integral test, or otherwise, explain why the series
\[
\sum_{n = 1}^{\infty} \frac{1}{n^2 + 7n + 6}
\]
is convergent.

{\itshape Do only one of (b) and (c). If you turn in both, only (b) will be graded.}

(b.) (14 points)
Use the integral test to 
give an upper and a lower bound for the value of this series accurate within $0.1$.
Draw and explain a graph which represents your lower bound.

(c.) (8 points. Do this if you don't know how to do (b).)
Use any method you know to give {\itshape any}
upper bound for the value of the series.

\item
(14 points) Use the comparison test to 
determine whether the series converges or diverges. 
If it converges, determine an explicit upper bound for the value of the series.
\[
\sum_{n = 1}^{\infty} \frac{9^n}{2 + 10^n}
\]
\item
(14 points) 
Test the alternating series for convergence or divergence. If it converges, determine
({\itshape in simplified form})
an estimate for its value which is accurate within $0.25$.
\[
\sum_{n = 1}^{\infty} (-1)^n \frac{n^2}{2n^3 - 1}
\]
\item
(14 points) 
Determine whether the series is absolutely convergent, conditionally convergent, or divergent.
\[
\sum_{k = 1}^{\infty} k \bigg( \frac{2}{3} \bigg)^k
\]


\end{enumerate}

\end{document}
