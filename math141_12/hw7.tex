\documentclass[12pt]{article}
\textwidth 7.0in
\oddsidemargin -0.4in
\evensidemargin -0.4in
\textheight 9.0in 
\pagestyle{empty}
\usepackage{enumerate}
\begin{document}
% 8.5in paper width -2x1in margin = 6.5in text width
\setlength{\topmargin}{-2mm}



% 11in paper height -2x1in margin = 9in text height


\begin{center}{\bf Homework 7 - Math 141, Frank Thorne (thornef@mailbox.sc.edu)}
\end{center}
\begin{center}
{\bf Due Wednesday, October 17}
\end{center}

\begin{enumerate}[(a)]

\item
Stewart, Ch. 3.9, 2, 5, 11, 12, 13, 14, 15, 16.

You do not have to follow Stewart's
(a)-(e) exactly (although this is a very good idea!) but with these problems especially, 
you should {\bf explain what you are doing}
in complete sentences and {\bf draw a picture} if appropriate.

\item
Define the terms {\itshape local maximum}, {\itshape local minimum}, {\itshape absolute maximum},
{itshape absolute minimum}, and {\itshape critical number}.

\item
Describe how to find all the critical numbers of a function.

\item
Describe how to find all the local maxima of a function.

\item
(Trick question. Explain why.) Explain how to find all the absolute maxima of a function.

\item
Stewart, Ch. 4.1, 7-10, 13-14, 21-28, 47-54 (even).
\end{enumerate}
Additional problems: 
\begin{enumerate}[(a)]
\item
Stewart, Ch. 3.9, 17, 20.

\item
Stewart, Chapter 3 Review (pp. 262-263), 15-40, 57-61, 65-66, 83-84.
{\bf That is a lot of problems.} If you do only a selection, then that is good
enough! These problems are similar to previous problems and are fair game
for the exams.

\item
Stewart, Ch. 4.1, 7-10, 13-14, 21-28, 47-54 (odd).
 even required, odd recommended.

\end{enumerate}
Bonus (2 points): Problem 1 on ``Problems Plus'', p. 266.
\end{document}
