\documentclass[12pt]{article}
\textwidth 7.0in
\oddsidemargin -0.4in
\evensidemargin -0.4in
\textheight 9.0in 
\pagestyle{empty}
\usepackage{enumerate}
\begin{document}
% 8.5in paper width -2x1in margin = 6.5in text width
\setlength{\topmargin}{-2mm}



% 11in paper height -2x1in margin = 9in text height


\begin{center}{\bf Examination 2 - Math 141, Frank Thorne (thornef@mailbox.sc.edu)}
\end{center}
\begin{center}
{\bf Wednesday, October 31, 2012}
\end{center}

Please work without books, notes, calculators, or any assistance from others. If you have
any questions, feel free to ask me. Please do your work on separate paper; you should staple this sheet to your work (put this on top)
and turn in everything together. 

The first five questions are 16 points each and the last is 20 points.
\\
\\
\begin{enumerate}[(1)]
\item
Compute $\frac{df}{dx}$ for $f(x) = \tan(x)$ and for $f(x) = \csc(x)$.

\item
Find $\frac{dy}{dx}$ if $y = \ln(7 + 2 x^5)$.

\item
Explain the meaning and the origin of the equation $y(t) = y(0) e^{kt}$ in modeling situations
such as population growth, compound interest, radioactive decay, etc. When is $k$ positive, and
when is it negative?

\item
If a snowball melts so that its surface area decreases at a rate of 1 $cm^2/min$, find the rate
at which the diameter decreases when the diameter is $10$ $cm$.

(Recall that the surface area of a sphere of radius $r$ is $4 \pi r^2$.)

\item
Find the absolute maximum and minimum values of $\frac{x}{x^2 + 1}$ on $[0, 2]$.
 
\item
Graph the function $y = 2 + 2 x^2 - x^4$. Indicate where your graph is increasing or decreasing,
and where it is concave up and down. Indicate all the critical points and points of inflection.

\end{enumerate}

\end{document}
