

\documentclass{article}
\usepackage{amssymb,amsmath,amsthm}

%=========================================================================
% Settings
%=========================================================================
\setlength{\textwidth}{6.9in}             % Alter margins
\setlength{\textheight}{9.5in}
\setlength{\topmargin}{-1.8cm}
\setlength{\oddsidemargin}{-0.7cm}

\pagestyle{empty}                       % Suppress page number

%=========================================================================
% New Definitions
%=========================================================================
% \name{NAME-OF-PERSON}
%       Display a centered NAME-OF-PERSON of a certain size.
%       A line is placed underneath, which spans the width of the
%       page.  The name is lowered so that is hovers above 
%       this line.
%       NOTE: if you want text directly underneath this command,
%             do not put a blank line in between.
%
\newcommand{\name}[1]{
  \begin{center}
    \raisebox{-1.4ex}{\LARGE #1}
  \end{center}
}

%-------------------------------------------------------------------------
%\newcommand{\addrtype}{\sl}            % Type style for addresses
%\newcommand{\addrwidth}{2in}           % Width of addresses

% ENVIRONMENT: \begin{address}
%       Display an address.
%       The address is in a minipage of width \addrwidth.  The type
%       style of the address is \addrtype.
%       
%\newenvironment{address}{\begin{minipage}[t]{\addrwidth}
%       \addrtype}{\end{minipage}}

% David Carlton sez: I don't recall ever used the above, and I don't
% remember if it was commented out when I got this or if that's my
% innovation.  Feel free to play with it yourself...

%-------------------------------------------------------------------------

% The body of the resume should all be in one big categories
% environment.  Start a new category with \category, start a new entry
% within the category with \entry.  Leave a blank line between entries
% (and between categories).  \entry takes two arguments: the topic of
% the entry (which is in boldface and on a line by itself) and the
% body of the entry.  If you don't want that format, use \rawentry
% instead; it takes one argument, the body of the entry, and you can
% format that however you want.  Within that body, if you want a line
% break, use \\; also, \\ works in the argument to \category.

\newcommand{\categorywidth}{1in}        % Width of categories (left side)
\newcommand{\infowidth}{5.8in}          % Width of info (right side)
\newcommand{\categorysep}{5pt}

\newcommand{\catlistlabel}[1]%
{\raisebox{0pt}[1ex][0pt]{\makebox[\labelwidth][l]%
    {\parbox[t]{\labelwidth}{\hspace{0pt}\textbf{#1}}}}}
\newenvironment{categories}{\begin{list}{}{
      \setlength{\labelwidth}{\categorywidth}
      \setlength{\leftmargin}{\labelwidth}
      \addtolength{\leftmargin}{\labelsep}
%      \setlength{\rightmargin}{0pt}
      \setlength{\topsep}{20pt}
      \setlength{\itemsep}{\categorysep}
      \renewcommand{\makelabel}{\catlistlabel}
      }}{\end{list}}

\newcommand{\category}[1]{\item[#1]}

\newcommand{\topic}[1]{
  {\textbf{#1}}}
\newcommand{\rawentry}[1]{{\begin{minipage}[t]{\infowidth}{#1}
    \end{minipage}}}
\newcommand{\entry}[2]{\rawentry{{\topic{#1}\\}{#2}}}

%*************************************************************************
% Document
%*************************************************************************
\begin{document}
\begin{flushleft}

%=========================================================================
% Name
%=========================================================================

\name{FRANK THORNE}
\rule{7.0in}{.05cm}
\vspace{.15ex}

%=========================================================================
% Addresses
%=========================================================================

{\em Contact Information:} % \hfill Home Address:}

Department of Mathematics % \hfill Hibben Apartments 3-L

University of South Carolina % \hfill Faculty Road

1523 Greene Street % $\cdot$ Washington Road \hfill Princeton, NJ 08540

Columbia, SC 29208% \hfill (609)-688-9429

\texttt{thorne@math.sc.edu}

\texttt{http://www.math.sc.edu/\~{}thornef}

\vskip 0.1in
{\em Last updated: July 10, 2013.}
%=========================================================================
% Body
%       Must have space between sections.
%=========================================================================


\begin{categories}

  \category{Employment}

  \rawentry{NSF Postdoctoral Fellow, Stanford University, 2008-2011. \\
  Mentor: Kannan Soundararajan.}

  \rawentry{Assistant Professor, University of South Carolina, Fall 2008-present
 (on leave 2008-2011).}

  \category{Education}
  
  \rawentry{Ph.D., University of Wisconsin, Madison, WI, 2008. \\
    Dissertation under the supervision
    of Prof. Ken Ono: `Extensions of results on the distribution of primes.'}

  \rawentry{Rice University, Houston, TX. \\
    B.A. summa cum laude in Mathematics, May 1999.}

  \category{Research \\ Interests}

  \rawentry{Number theory; distribution of primes and broadly related questions;
  prehomogeneous vector spaces. \\}





  \category{Research \\ Publications}

  $\bullet$ {\itshape Bounded gaps between products of primes with applications
to elliptic curves and ideal class groups\upshape, International Mathematics
Research Notices (2008), 41 pp.}

  $\bullet$ {\itshape Irregularities in the distribution of primes in function fields\upshape, 
   J. Number Theory \textbf{128} (2008), 1784-1794.}

  $\bullet$ {\itshape Bubbles of congruent primes\upshape, submitted.}

  $\bullet$ {\itshape An uncertainty principle for function fields\upshape,
J. Number Theory \textbf{131} (2011), no. 8, 1363-1389. 

  $\bullet$ {\itshape Maier matrices beyond $\mathbb{Z}$\upshape.
Combinatorial number theory, 
Proceedings of the Integers Conference 2007, 185-192.}

  $\bullet$ {\itshape Analytic properties of Shintani zeta functions}, Proceedings
of the RIMS Symposium on automorphic forms, automorphic representations and related topics.

  $\bullet$ {\itshape Secondary terms in counting functions for cubic fields}, with T. Taniguchi, Duke Math. J., accepted for publication.

  $\bullet$ {\itshape Shintani's zeta function is not a finite sum of Euler products}, to appear,
Proc. Amer. Math. Soc.

  $\bullet$ {\itshape Orbital $L$-functions for the space of binary cubic forms}, with T. Taniguchi, submitted.

  $\bullet$ {\itshape Four perspectives on secondary terms in the Davenport-Heilbronn theorems}, Integers (12B),
Proceedings of the Integers Conference 2011, Carrollton, GA.

  $\bullet$ {\itshape An error estimate for counting $S_3$-sextic number fields}, with T. Taniguchi, submitted.

  $\bullet$ Book review: {\itshape Opera de cribro} and {\itshape
  An introduction to sieve methods and their applications}, to appear,
  Bull. Amer. Math. Soc.

  $\bullet$ {\itshape On the existence of large degree Galois representations for fields of small
discriminant}, with J. Rouse, submitted.

  $\bullet$ {\itshape Dirichlet series associated to cubic fields with given quadratic resolvent}, with H. Cohen,
submitted.


  $\bullet$ {\itshape Dirichlet series associated to quartic fields with given cubic resolvent}, with H. Cohen, submitted.

  $\bullet$ {\itshape Zeros of $L$-functions outside the critical strip}, with A. Booker, submitted.

  $\bullet$ {\itshape Additional papers in preparation.}

  \category{Awards and \\ Prizes}

  $\bullet$ {Invited Participant, Math Olympiad Summer Program. \\
  \itshape Participated, all expenses paid, in the summer training camp for the IMO. \upshape }

  $\bullet$ {Top 75, Putnam Examination.}

  $\bullet$ {NSF VIGRE Merit Fellowship, Spring 2007, Fall 2007, Spring 2008.}

  $\bullet$ {Letters and Science Teaching Fellow, Fall 2007. \\
\itshape Was selected as one of fifteen students (and one of two math students)
to plan and lead orientation for new teaching assistants in the fall.
\upshape }

  $\bullet$ {Excellence in Mathematical Research Award, Spring 2008.}

  $\bullet$ {NSF Postdoctoral Fellowship, 2008-2011.}

  $\bullet$ {NSF Individual Research Grant, \$144,926, 2012-2015..}

  $\bullet$ {NSF Grant, Southeast Number Theory Meetings (co-PI), 2012-2013.}

  $\bullet$ {NSA Grant, Southeast Number Theory Meetings (co-PI), 2012-2013.}

\category{Students \\ Supervised}
$\bullet$ Richard Oh, Ph. D. candidate.

$\bullet$ Matthew Corley, USC undergraduate (currently working on research project).

\category{Professional Service} 
$\bullet$ {Referee, Proceedings of the American Mathematical Society (19 articles),
American Mathematical Monthly,
Acta Arithmetica,
International Journal of Number Theory, Journal of Number Theory (4 articles), Communications in Number Theory and Physics,
American Journal of Mathematics, Acta Mathematica Sinica,
Journal Abhandlungen aus dem Mathematischen Seminar der Universit\"at Hamburg,
Canadian Journal of Mathematics,
Ramanujan Journal,
Journal of the London Mathematical Society,
Algebra and Number Theory}

$\bullet$ {Reviewer, Mathematical Reviews.}

$\bullet$ {Reviewer, NSA Mathematical Sciences Grant Program (3 proposals).}

$\bullet$ {Volunteer, Mega Math Meet and Sidewalk Math, University of Wisconsin.}

$\bullet$ {Volunteer, Julia Robinson Math Festival, Stanford, Spring 2010.}
 
$\bullet$ Organizer, Stanford Analytic Number Theory Seminar, 2008-2011.

$\bullet$ Reviewer, DOD SC Junior Science \& Humanities Symposium, December 2011.

$\bullet$ Committee Member, South Carolina State High School Math Contest, 2011-2012.

$\bullet$ Guest Lecturer/Teachers' Preparation, USC AP Calculus Practice Exam, April 2012.

$\bullet$ Hiring Committee, USC, 2012-2013.

$\bullet$ (other committee service at USC as well)

\category{Conferences Organized}

$\bullet$ Midwest Number Theory Conference for Grad Students V, University of Wisconsin, November 2007.

$\bullet$ Mathematics Research Community on ``The Pretentious View of Analytic Number Theory'', Snowbird, UT, June-July 2011.

$\bullet$ AIM SQuaRE on ``Four Perspectives on Secondary Terms in the Davenport-Heilbronn Theorem'', Palo Alto, CA, Oct.-Nov. 2012.

{\itshape The SQuaRE was a small workshop with seven mathematicians (including myself). I wrote a proposal which led to full funding
of the workshop by the American Institute of Mathematics; 
participants ranged from a graduate student to a Princeton professor and came from the US, Britain, and Japan. Followup workshops
will be held in 2013 and possibly beyond.}

$\bullet$ Palmetto Number Theory Series, University of South Carolina, December 2012.

$\bullet$ AMS Special Session on Modern Methods in Analytic Number Theory, University of Mississippi, Oxford, MS, March 2013.

\category{Research Talks}

$\bullet$ {\itshape Bounded gaps between products of primes with applications to elliptic curves
and ideal class groups, Midwest Number Theory Conference for Grad Students IV\upshape, UIUC, October 2006.}

$\bullet$ {\itshape Bounded gaps between products of primes with applications to elliptic curves
and ideal class groups\upshape, AMS Sectional, Fayetteville, AR, November 2006.}

$\bullet$ {\itshape Bounded gaps between products of primes with applications to elliptic curves
and ideal class groups\upshape, Number Theory Seminar, UW-Madison, December 2006.}

$\bullet$ {\itshape Bounded gaps between products of primes with applications to elliptic curves
and ideal class groups\upshape, Number Theory Seminar, UIUC, April 2007.}

$\bullet$ {\itshape Bounded gaps between products of primes with applications to elliptic curves
and ideal class groups\upshape, Illinois Number Theory Fest, UIUC, May 2007.}

$\bullet$ {\itshape Maier matrices beyond $\mathbb{Z}$\upshape, Number Theory Seminar,
University of South Carolina, October 2007.}

$\bullet$ {\itshape Maier matrices beyond $\mathbb{Z}$\upshape, Number Theory Seminar,
Stanford, October 2007.}

$\bullet$ {\itshape Maier matrices beyond $\mathbb{Z}$\upshape, Integers, Carrollton, GA, October 2007.}

$\bullet$ {\itshape Maier matrices beyond $\mathbb{Z}$\upshape,
Palmetto Number Theory Series, Furman, February 2008.}

$\bullet$ {\itshape Extensions of results on the distribution of primes\upshape, Colloquium, University of South Carolina, February 2008.}

$\bullet$ {\itshape Bounded gaps between products of primes with applications to elliptic curves and ideal class groups\upshape, 22nd Annual Workshop on Automorphic
Forms and Related Topics, Texas A\&M, March 2008.}

$\bullet$ {\itshape Analytic properties of Shintani zeta functions}, Zeta Function Days,
Seoul, Korea, September 2009.

$\bullet$ {\itshape Analytic properties of Shintani zeta functions}, University of South Carolina,
December 2009.

$\bullet$ {\itshape Analytic properties of Shintani zeta functions}, Palmetto Number Theory Series,
Columbia, SC, December 2009.

$\bullet$ {\itshape Analytic properties of Shintani zeta functions}, Kobe Number Theory Workshop,
Kobe, Japan, January 2010.

$\bullet$ {\itshape Analytic properties of Shintani zeta functions}, RIMS Symposium on automorphic
forms, automorphic representations and related topics, University of Tokyo, January 2010.

$\bullet$ {\itshape Analytic properties of Shintani zeta functions}, colloquium, Kinki University,
Osaka, Japan, January 2010.

$\bullet$ {\itshape The secondary term in the counting function for cubic fields}, 
ICM Satellite Conference in Analytic and Combinatorial Number Theory, Chennai, India, August 2010.

$\bullet$ {\itshape Secondary terms in counting functions for cubic fields},
University of Georgia, January 2011.

$\bullet$ {\itshape Secondary terms in counting functions for cubic fields},
Emory University, January 2011.

$\bullet$ {\itshape Secondary terms in counting functions for cubic fields},
University of South Carolina, January 2011.

$\bullet$ {\itshape Secondary terms in counting functions for cubic fields},
Workshop on Arithmetic Statistics, MSRI, Berkeley, CA, January 2011.

$\bullet$ {\itshape Secondary terms in counting functions for cubic fields},
Analytic Aspects of $L$-functions and Applications to Number Theory,
Calgary, May-June 2011.

$\bullet$ {\itshape Secondary terms in counting functions for cubic fields},
Special Session on $L$-functions and Number Theory, Meeting of the Canadian Mathematical
Society, June 2011.

$\bullet$ {\itshape Is Shintani's zeta function a finite sum of Euler products?},
Palmetto Number Theory Series, Emory University, Atlanta, GA, September 2011.

$\bullet$ {\itshape Secondary terms in counting functions for cubic fields},
Special Session on Modular Forms, Elliptic Curves, and Related Topics,
AMS Sectional Meeting, September 2011.

$\bullet$ {\itshape Four perspectives on a secondary term}, plenary lecture,
Integers Conference 2011, Carrollton, GA, October 2011.

$\bullet$ {\itshape Four perspectives on a secondary term},
Clemson University, November 2011.

$\bullet$ {\itshape Secondary terms in counting functions for cubic fields},
Quebec-Vermont Number Theory Seminar, Concordia University, November 2011.

$\bullet$ {$1 + 2 + 3 + 4 + \cdots$}, MAA Invited Paper Session on the Beauty and
Power of Number Theory}, AMS/MAA Joint Meetings, Boston, MA, January 2012.

$\bullet$ {\itshape Prehomogeneous vector spaces and counting field extensions},
Emory University, 2012.

$\bullet$ {\itshape Progress on counting discriminants of cubic fields}, 
Special Session on Automorphic and Modular Forms, AMS Sectional Meeting, March 2012.

$\bullet$ {\itshape Shintani's zeta function is not a finite sum of Euler products},
Southeast Regional Meeting on Numbers, Western Carolina University, April 2012.

$\École polytechnique fédérale de Lausannebullet$ {\itshape Secondary terms in counting functions for cubic fields},
Number Theory Seminar, Texas A\&M, April 2012.

$\bullet$ {\itshape Secondary terms in counting functions for cubic fields},
Number Theory Seminar, University of Wisconsin, April 2012.

$\bullet$ {\itshape Counting fields with analytic number theory},
Synergies and Vistas in Analytic Number Theory, Oxford University, September 2012.

$\bullet$ {\itshape Counting fields using geometry and zeta functions} (two lectures),
Workshop on Zeta Functions, Tokyo Institute of Technology, Tokyo, Japan, September 2012.

$\bullet$ {\itshape Large degree Galois representations for fields of small discriminant},
Special Session on Harmonic Maass Forms and $q$-series, University of Arizona, October 2012.

$\bullet$ {\itshape Explicit Dirichlet series for fields with given resolvent},
Atkin Workshop, University of Illinois at Chicago, May 2013.

$\bullet$ {\itshape $1 + 2 + 3 + 4 + \cdots$}, Colloquium, Dartmouth College, May 2013.

$\bullet$ {\itshape Explicit Dirichlet series for fields with given resolvent},
Dartmouth College, May 2013.

$\bullet$ {\itshape Four perspectives on secondary terms in the Davenport-Heilbronn Theorems},
Hong Kong University of Science and Technology, May 2013.

$\bullet$ {\itshape Four perspectives on secondary terms in the Davenport-Heilbronn Theorems},
Kobe University, May 2013.

$\bullet$ {\itshape Secondary terms in counting functions for cubic fields},
Tohoku University (Sendai, Japan), June 2013.

$\bullet$ {\itshape Explicit Dirichlet series for fields with given resolvent},
Workshop on Automorphic Forms, RIMS, Kyoto, Japan, 2013.

$\bullet$ {\itshape Four perspectives on secondary terms in the Davenport-Heilbronn Theorems},
\'Ecole Polytechnique F\'ed\'erale de Lausanne, Lausanne, Swizerland, June 2013.

\rawentry{\itshape I have also given additional talks at USC, not listed, 
since arriving in Fall 2011.}

\category{Additional Conferences Attended}

$\bullet$ {Midwest Number Theory Day, UW-Madison, October 2005.}

$\bullet$ {Midwest Number Theory Conference for Grad Students III, UW-Madison, October 2005.}

$\bullet$ {Number Theory and Random Matrix Theory, Rochester, June 2006.}

$\bullet$ {Summer School in Iwasawa Theory, Hamilton, Ontario, August 2007.}

$\bullet$ {Analytic Number Theory and Higher Rank Groups, NYU, May 2008.}

$\bullet$ {Arithmetic of $L$-functions, Park City Mathematics Institute, June-July 2009.}

$\bullet$ {Analytic number theory, Institute for Advanced Study, December 2009.}

$\bullet$ {Mock modular forms in combinatorics and arithmetic geometry, American Institute of Mathematics,
Palo Alto, CA, March 2010.}

$\bullet$ Canadian Number Theory Association XI, Acadia University, Wolfville, NS, July 2010.

$\bullet$ International Congress of Mathematicians, Hyderabad, India, August 2010.

$\bullet$ The Number Theory of Partitions, Emory University, Atlanta, GA, January 2011.

$\bullet$ The Cohen-Lenstra Heuristics for Class Groups, AIM, Palo Alto, CA, June 2011.

$\bullet$ Palmetto Number Theory Series, Clemson University, December 2011.

$\bullet$ Summer School on Diophantine Equations, Banff, AB, June 2012.

$\bullet$ Canadian Number Theory Association XII, Lethbridge, AB, June 2012.

$\bullet$ Mathematical Research Community on Arithmetic Statistics, Snowbird, UT, June 2012.

$\bullet$ Palmetto Number Theory Series, Wake Forest, NC, September 2012.

  \category{Teaching \\ Experience}

  $\bullet$ {Teaching Assistant, Talent Identification Program, Duke University, Summer 1998, 1999. \\
  \itshape TIP is an intensive academic summer program for academically gifted seventh through
tenth graders. \upshape}

  $\bullet$ {English Teacher, America Eigo Gakuin, Kyoto, Japan, 1999-2001.}

  \rawentry{\itshape As a graduate student at the University of Wisconsin:}

  $\bullet$ {Teaching Assistant, Math 221 (Calculus I), Fall 2004.}

  $\bullet$ {Teaching Assistant, Math 222 (Calculus II), Spring 2005.}

  $\bullet$ {Teaching Assistant, Math 114 (Algebra and Trigonometry), Fall 2005.*}

  $\bullet$ {Instructor, Math 112 (College Algebra), Spring 2006.}

  $\bullet$ {Instructor, Math 112 (College Algebra), Fall 2006.* }

  $\bullet$ {Instructor, REU in Number Theory, Summer 2007, 2008.}

  \rawentry{* = teaching was judged as Superior by the TA Evaluation Committee.}

  \rawentry{\itshape At Stanford University:}

  $\bullet$ {Instructor, Math 42 (Calculus), Fall 2008.}

  $\bullet$ {Teaching Assistant, Arizona Winter School, Tucson, AZ, March 2010.}

  $\bullet$ {Instructor, Math 108 (Combinatorics), Fall 2010.}

  $\bullet$ {Instructor, Math 249C (Topics in Number Theory: Additive Combinatorics), Spring 2011.}

  \rawentry{\itshape At the University of South Carolina:}

  $\bullet$ {Instructor, Math 141 (Calculus), Fall 2011.}

  $\bullet$ {Instructor, Math 782 (Analytic Number Theory), Fall 2011.}

  $\bullet$ {Instructor, Math 574 (Discrete Math), Spring 2012.}

  $\bullet$ {Instructor, Math 141 (Calculus), Fall 2012.}

  $\bullet$ {Instructor, Math 531 (Euclidean Geometry), Fall 2012.}


  \category{Non-Academic \\ Employment}
  
  $\bullet$ {English Teacher, America Eigo Gakuin, Kyoto, Japan, 1999-2001. \\
  \itshape Responsible for curriculum design as well as classroom instruction;
taught individuals and small and large groups from ages three to adult. \upshape}

  $\bullet$ {Software Developer, /n software, inc., Durham, NC, 2002-2004. \\
  \itshape Was lead developer for multiple commercial software projects in
Java, Delphi, and Microsoft .NET. \upshape}

  \category{Personal}

  $\bullet$ {Proficient in German and Japanese (spoken and written).}

  $\bullet$ {Black Belt in Shorin-Ryu Karate, April 2007.}

  \category{References}

  $\bullet$ {Ken Ono (doctoral advisor) \\
            Manasse Professor of Letters and Science, University of Wisconsin-Madison \\
            480 Lincoln Drive, Madison, WI 53706; \ ono@math.wisc.edu}

  $\bullet$ {Kannan Soundararajan (postdoctoral mentor)\\
            Professor of Mathematics, Stanford University \\
            450 Serra Mall, Building 380, Stanford, CA 94305; ksound@math.stanford.edu}
  
  $\bullet$ {Additional references available upon request.}

\end{categories}


%\vspace{0.3in}
%\begin{center}
%References available upon request.
%\end{center}

\end{flushleft}
\end{document}
%*************************************************************************
