\documentclass[12pt]{article}
\textwidth 7.0in

\oddsidemargin -0.4in
\evensidemargin -0.4in
\textheight 9.6in 
\pagestyle{empty}
\usepackage{amssymb,amsmath,amsthm}
\usepackage[mathscr]{euscript}
\usepackage{enumerate, verbatim, url}
\begin{document}
% 8.5in paper width -2x1in margin = 6.5in text width
\setlength{\topmargin}{-12mm}




% 11in paper height -2x1in margin = 9in text height


\begin{center}{\bf Practice ``Examination 3'' with Solutions - Math 574, Frank Thorne (thorne@math.sc.edu)}
\end{center}

There was no Exam 3, but enumerative combinatorics will be on the final. If I'd given a midterm,
it would have looked something like this (although probably this is a bit longer than a midterm would have been.)
\\
\\
\begin{enumerate}[1.]
\item
\begin{enumerate}[(a)]
\item
How many bit strings of length 9 are there?
\\
\\
$2^9 = 512$. There are two choices for the first bit, two choices for the second bit, and so on until the ninth
bit, so we multiply 2 nine times.
\item
How many bit strings of length 9 are there, which start with a 1 and which have at least one other 1?
\\
\\
$2^8 - 1 = 255$. Counting bit strings of length 9 which start with a 1 is the same as counting bit strings of length
8 with no condition. There are 256 of these, and we subtract one because we don't want to count the string $100000000$.

\item
How many bit strings of length 9 are there, such that every 1 is followed immediately by a zero?
\\
\\
This is a little bit harder. One good way to do this is to split into cases. There can be 0, 1, 2, 3, or 4 ones.
If there are no ones, there is one possibility. If there is one one, then we are permuting one string $10$ and seven
additional zeroes, and we can do this in any order, so there are ${8 \choose 1} = 8$ possibilities. If there are two
ones, then we are permuting two strings $10$ and five zeroes, for ${7 \choose 2} = 21$ possibilities. We can similarly
count the number of bit strings with 3 or 4 ones, so that the total is
$${9 \choose 0} + {8 \choose 1} + {7 \choose 2} + {6 \choose 3} + {5 \choose 4} = 1 + 8 + 21 + 20 + 5 = 55.$$

\end{enumerate}

\item
You are in a city with rectangular blocks. You want to walk five blocks north and three blocks east,
while walking north or east for every block.
\begin{enumerate}[(a)]
\item
In how many ways can do you this?
\\
\\
This is the same as counting bit strings of length 8 with 5 zeroes and 3 ones, where each zero means walk north
and each one means walk east. So the answer is ${8 \choose 5} = 56$.

\item
Suppose now you are not allowed to go east for the first block. Now in how many ways can you do this?
\\
\\
If we are not allowed to go east for the first block then we must go north for the first block. So we have
to choose how to walk four blocks north and three blocks east, and by the same logic the number of ways is
${7 \choose 4} = 35.$
\end{enumerate}

\item
\begin{enumerate}[(a)]
\item
How many integers between 1 and 1000 are multiples of 2 or multiples of 5?
\\
\\
There are 500 multiples of 2, 200 multiples of 5, and 100 multiples of both. So the answer, by the Inclusion-Exclusion
principle, is 
$$500 + 200 - 100 = 600.$$
\\
\\
{\bf Alternative solution.} An integer is a multiple of 2 or 5 if its last digit is 2, 4, 5, 6, 8, or 0.
The problem asks us to choose 3 digits so that the last is 2, 4, 5, 6, 8, or 0, and there are no conditions
on the first two digits. (If we choose 000, this corresponds to 1000.) The total number of possibilities is
thus 
$$10 \times 10 \times 6 = 600.$$

\item
(Careful!) How many integers between 1 and 1000 are multiples of 2, multiples of 5, or multiples of 10?
\\
\\
{\bf The long way:} There are 500 multiples of 2, 200 multiples of 5, 100 multiples of 10, 100 multiples of 2 and 5,
100 multiples of 2 and 10, 100 multiples of 5 and 10, and 100 multiples of 2, 5, and 10. So the answer is
$$500 + 200 + 100 - (100 + 100 + 100) + 100 = 600.$$
\\
\\
{\bf The shortcut:} An integer is a multiple of 10 if and only if it is a multiple of both 2 and 5. So if a number
is a multiple of 10, it will automatically satisfy both of the other conditions, and so adding ``or multiples of 10''
is a redundant condition -- it doesn't count anything new. So the answer is 600, same as last problem.

Note that this explains why the number 100 occurs five times in ``the long way''!


\end{enumerate}

\item
You pass five graded papers back to the five students who wrote them. In how many ways can you do this, so that
no student gets his or her own paper? (Explain fully.)
\\
\\
This is the inclusion-exclusion principle. We did in this in class (for the ``race game''). 

It is easier to count the number of ways so that at least one student gets their paper back, and then subtract
from 120. Suppose that the students are named Ann, Bob, Cid, Deb, and Eve. Let $A$ be the set of ways to pass the
papers back so that Ann gets her paper back; let $B$ be the set of ways to pass the papers back so that Bob gets his
paper back, and so on. We want to count $A \cup B \cup C \cup D \cup E$. The inclusion-exclusion principle says that
\begin{align*}
|A \cup B \cup C \cup D \cup E| = & |A| + |B| + |C| + |D| + |E|
\\ & - |A \cap B| - |A \cap C| - \cdots
\\ & + |A \cap B \cap C| + |A \cap B \cap D| + \cdots
\\ & - |A \cap B \cap C \cap D| - |A \cap B \cap C \cap E| - \cdots
\\ & + |A \cap B \cap C \cap D \cap E|.
\end{align*}
The second row includes all intersections of two sets, the third row counts all intersections of three sets, the fourth row
counts all intersections of four sets.

In row $i$, the size of each term is $(n - i)!$, because you specify where $i$ of the papers go and allow the rest to be
arbitrary. There are ${n \choose i}$ terms, because there is one term for each subset of $\{A, B, C, D, E\}$ of size $i$.

Therefore,
$$|A \cup B \cup C \cup D \cup E| = 4! {5 \choose 1} - 3! {5 \choose 2} + 2! {5 \choose 3} - 1! {5 \choose 4} + 0! {5 \choose 5}
= 120 - 6 \cdot 10 + 2 \cdot 10 - 1 \cdot 5 + 1 = 76.$$
So the total number of ways to pass all papers to the wrong students is $5! - 76 = 120 - 76 = 44.$

\item
How many numbers must you pick to ensure that at least three of them have the same remainder when divided by 11?
\\
\\
23. This doesn't have to happen if we pick 22 numbers. For example, if we pick 1 through 22 then we get each remainder
twice. However, if we have 23 numbers then we are trying to put 23 pigeons (the numbers) in 11 pigeonholes (the remainders).
Since $\frac{23}{11} > 2$, it follows that more than two (so, three) pigeons must go in the same hole, i.e., at least three
integers must have the same remainder when divided by 11.

\item
A computer programming team has 13 members -- six men and seven women.
\begin{enumerate}[(a)]
\item
In how many ways can you choose a team of five?
\\
\\
The answer is just ${13 \choose 5}$.
\item
In how many ways can you choose a team of five, with two men and three women?
\\
\\
We choose the men and women separately: ${6 \choose 2} \cdot {7 \choose 3} = 15 \cdot 35 = 525.$

\item
In how many ways can you choose a team of five, with at most three men?
\\
\\
We have to break up into cases. The total is
$${7 \choose 5} + {7 \choose 4}{6 \choose 1} + {7 \choose 3}{6 \choose 2} + {7 \choose 2}{6 \choose 3}.$$

\item
Suppose that one of the men and one of the women are a divorced couple, and refuse to work together.
In how many ways can you choose a team of five, with two men and three women?
\\
\\
We start with 525 as before. Then we subtract all the ways where we pick both the man and the woman from this couple.
There are ${5 \choose 1} {6 \choose 2} = 5 \cdot 15 = 75$ ways to do this, so there are $525 - 75 = 450$ ways left over.

\end{enumerate}

\item
How many functions are there from the set $X = \{1, 2, 3\}$ to the set $Y = \{A, B, C, D, E, F, G\}$?
How many of these are one-to-one?
\\
\\
There are $7^3 = 343$ functions total: For each element $x$ of $X$ we have seven different choices for where to send $x$.
There are $7 \cdot 6 \cdot 5 = 210$ one-to-one functions: Seven choices for the first element, six choices for the second
(because we used one already), five for the third.
 
\item
A store sells 8 kinds of balloons with at least 30 balloons of each kind. How many different combinations
of 30 balloons can be chosen? What if the store has only 10 red balloons, but at least 30 of every other kind
of balloon?
\\
\\
We solve this as a stars-and-bars problem. We represent the balloons by 30 stars. We have seven bars, so that
all the balloons before the first bar are of the first type, all the balloons after the first and before the second are
of the second type, and so on. (Seven bars partition the balloons into 8 regions.) Any permutation of 30 stars and 7 bars
yields a different way of buying balloons, so the total number is ${37 \choose 7}$.

For the second part, if we buy 11 red balloons, then $37 - 11 = 26$, so there are ${26 \choose 7}$ ways of buying balloons
with at least 11 red. Since this is not allowed, the final answer is ${37 \choose 7} - {26 \choose 7}$.

\item
How many solutions to $x_1 + x_2 + x_3 = 10$ are there, where $x_1, \ x_2, \ x_3$ are nonnegative integers?
What if, instead, $x_1, \ x_2, \ x_3$ are positive integers?
\\
\\
This is again a stars-and-bars problem. (Imagine we want to buy 10 balloons of three different colors.) We have
10 stars and 2 bars, so that the 2 bars divide the stars into three pieces, and the number of stars tell us $x_1$, $x_2$, $x_3$
respectively. The total number of possibilities is ${12 \choose 2}$.

If we require $x_1, \ x_2, \ x_3$ to be positive, then let $y_1 = x_1 - 1, \ y_2 = x_2 - 1, y_3 = x_3 - 1$ be nonnegative.
Our problem is the same as asking $y_1 + y_2 + y_3 = 7$, so there are 7 stars and two bars, and the total number of possibilities
is ${9 \choose 2} = 36$.

\item
Explain why the equation
$$2^n = {n \choose 0} + {n \choose 1} + {n \choose 2} + \cdots + {n \choose n}$$
is a consequence of the binomial theorem. In addition, {\itshape either} give a proof
of this identity which does not rely on the binomial theorem, {\itshape or} prove the binomial
theorem.
\\
\\
The binomial theorem says that
$$(x + y)^n = {n \choose 0} x^n + {n \choose 1} x^{n - 1} y^1 + {n \choose 2} x^{n - 2} y^2 + \cdots + {n \choose n} x^0 y^n.$$
If we plug in $x = y = 1$, then the left side is equal to $2^n$, and all the $x^k y^{n - k}$ terms are 1, so the right side becomes
the right side of the identity above, as desired.

We can give a combinatorial proof as follows. $2^n$ counts the number of bit strings of length $n$, total. Now ${n \choose 0}$
counts the number of bit strings of length $n$ and no ones; ${n \choose 1}$ counts the number of bit strings of length $n$ and one $1$;
and so on: ${n \choose k}$ counts the number of bit strings of length $n$ with $k$ ones. When we add up all the possibilities we get
the right side of our identity above, and this is the same as $2^n$.


\end{enumerate}

\end{document}
