\documentclass[12pt]{article}
\textwidth 7.0in
\oddsidemargin -0.4in
\evensidemargin -0.4in
\textheight 9.0in 
\pagestyle{empty}
\usepackage{enumerate}
\begin{document}
% 8.5in paper width -2x1in margin = 6.5in text width
\setlength{\topmargin}{-2mm}



% 11in paper height -2x1in margin = 9in text height


\begin{center}{\bf Homework 6 - Math 574, Frank Thorne (thornef@mailbox.sc.edu)}
\end{center}
\begin{center}
{\bf Due Friday, March 2 at 5:00.}
\end{center}
Problems 1-5 in Section 5.2 are good warmup induction problems. They are part of the ``additional''
problems in this set. Consider doing these first if you find these exercises difficult.
\\
\\
{\bf Core:}
\\
\\
5.2: 6, 11, 12, 13, 14, 19.
\\
\\
5.3: 2, 5, 10, 18, 21, 34, 35.
\\
\\
5.4: 1, 5.
\\
\\
{\bf Additional:}
\\
\\
5.2: 1-5, 7, 9.
\\
\\
5.3: 11, 19.
\\
\\
5.4: 4, 6.
\\
\\
{\bf Bonus:}
\begin{enumerate}
\item (2 points)
5.3, 37.

\item (2 points)
Find an integer $N$ for which the following statement is true, and prove it. (You are not
required to find the smallest possible $N$.)

If you only have coins worth 10 cents and 13 cents, then for any integer $n \geq N$, some combination
of coins is worth exactly $n$ cents.

\item (5 points - secret challenge!)
For all $m$ and $n \geq 0$, define a function $F(m, n)$ recursively 
as follows: If $m = 0$ then $F(m, n) = n + 1$. If $m > 0$ and $n = 0$ then
$F(m, n) = F(m-1, 1)$. If $m > 0$ and $n > 0$, then $F(m, n) = F(m - 1, F(m, n - 1))$.

As a warmup, show that $F(2, 3) = 9$ and $F(3, 2) = 29$. 

The secret challenge: Can you write down $F(5, 5)$? If so, write it down as an ordinary
number (in base 10). Prove all your claims.
\end{enumerate}
\end{document}
