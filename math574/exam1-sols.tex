\documentclass[12pt]{article}
\textwidth 7.0in

\oddsidemargin -0.4in
\evensidemargin -0.4in
\textheight 9.6in 
\pagestyle{empty}
\usepackage{amssymb,amsmath,amsthm}
\usepackage[mathscr]{euscript}
\usepackage{enumerate, verbatim, url}
\begin{document}
% 8.5in paper width -2x1in margin = 6.5in text width
\setlength{\topmargin}{-12mm}




% 11in paper height -2x1in margin = 9in text height


\begin{center}{\bf Midterm Examination 1 with Solutions - Math 574, Frank Thorne (thorne@math.sc.edu)}
\end{center}
\begin{center}
{\bf Thursday, February 9, 2012}
\end{center}
\begin{enumerate}[1.]
\item(3 points each)
For each sentence below, say whether it is logically equivalent to the sentence
\begin{center}
If it's snowing, the temperature must be at or below freezing.
\end{center}
(No explanation for your answers is required.)
\begin{enumerate}[(a)]
\item
If the temperature is above freezing, then it is not snowing.
\\
\\
{\bf Logically equivalent}; this is the contrapositive.
\item
It only snows if the temperature is at or below freezing.
\\
\\
{\bf Logically equivalent}. If $p$ then $q$ means the same thing as $p$ only if $q$
(i.e., $\neg p \vee q$)
\item
If it is not snowing, then the temperate must be above freezing.
\\
\\
{\bf Not equivalent}. This is the inverse.
\item
If the temperature is at or below freezing, then it is snowing.
\\
\\
{\bf Not equivalent.} This is the converse.
\end{enumerate}
\vskip 0.2in
\item(3 points each)
For each sentence below, say whether it is logically equivalent to the sentence
\begin{center}
For all animals $A$, if $A$ has claws then $A$ is dangerous.
\end{center}
(No explanation for your answers is required.)
\begin{enumerate}[(a)]
\item
Any animal that is dangerous has claws.
\\
\\
{\bf Not equivalent;} this is the converse.
\item
Among all the animals which have claws, some are dangerous.
\\
\\
{\bf Not equivalent.} The original statement describes all animals with claws. This describes only some animals with claws.
\item
Every animal with claws is dangerous.
\\
\\
{\bf Equivalent.} This is basically a rewording.
\item
Some of all the dangerous animals have claws.
\\
\\
{\bf Not equivalent.} This doesn't describe all animals with claws.
\end{enumerate}
\vskip 0.2in
\item(3 points each)
(For this question, no explanation for your answers is required. Recall that $\in$ means
``is an element of'' and $\subseteq$ means ``is a subset of''. Be careful!)
\begin{enumerate}[(a)]
\item
Is $\{y\} \in \{x, y, z\}$?
\\
\\
{\bf No.} The elements of $\{x, y, z\}$ are $x$, $y$, and $z$; $\{y\}$ is not among these.
\item
Is $\{y\} \in \{x, \{y\}, \{ \{ z \} \} \}$?
\\
\\
{\bf Yes,} it is one of the three elements on the right.
\item
Is $\emptyset \subseteq \{42, 1119, \{ \{ \emptyset \} \} \}$?
\\
\\
{\bf Yes,} the empty set is a subset of any set.
\item
How many elements are in the set $\{a, \{b, c\} \}$?
\\
\\
{\bf Two,} $a$ and $\{b, c\}$.
\end{enumerate}

\item(4 points each)
Formally rewrite each of the following statements using appropriate truth variables,
predicates, and quantifiers. Specify the domain for all your variables.

For example, an answer may look like
$\forall x \in \mathbb{Z} \ \  \big(P(x) \rightarrow Q(x)\big)$ where you should say what $P$ and $Q$ are.

\begin{enumerate}[(a)]
\item
If it walks like a duck and it quacks like a duck, then it is a duck.
\\
\\
Let $D$ be the domain of all animals (or of all things, or any other reasonable set which includes ducks.)

$P(x)$: $x$ walks like a duck.

$Q(x)$: $x$ quacks like a duck.

$R(x)$: $x$ is a duck.

The statement is $\forall x \in D \ (P(x) \wedge Q(x) \rightarrow R(x))$.
\item
The sum of any two even integers is even.
\\
\\
Let $\mathbb{Z}$ be the domain of all integers.

$P(x)$: $x$ is even.

The statement is $$\forall x \in \mathbb{Z} \ \forall y \in \mathbb{Z} \ (P(x) \wedge P(y) \rightarrow P(x + y)).$$

There are other formulations which don't involve any arithmetic in the statement. For example, let $Q(x, y)$ be
the statement ``$x + y$ is even'', and then you have
$$\forall x \in \mathbb{Z} \ \forall y \in \mathbb{Z} \ (P(x) \wedge P(y) \rightarrow Q(x,y)).$$

\end{enumerate}
\vskip 0.2in
\item(5 points each)
(For this question, no explanation for your answers is required.) Write negations of the following
statements.

\begin{enumerate}[(a)]
\item
If Tom is Ann's father, then Jim is her uncle and Sue is her aunt.
\\
\\
{\bf Negation:} Tom is Ann's father, but either Jim is not her uncle or Sue is not her aunt.
(You can use also the word ``and'' in place of but.)

\item
There is a computer program that gives the correct answer to every question that is posed to it.
(Your answer should include the phrase ``does not give the correct answer''.)
\\
\\
{\bf Negation:} For every computer program, there exists some question that can be posed to it
for which it does not give the correct answer.

\end{enumerate}

\item(10 points)
Determine (with proof) whether the following statement forms are logically equivalent.
$$p \rightarrow (q \rightarrow r), \ \ \ \ (p \rightarrow q) \rightarrow r.$$
\\
\\
The statements are {\bf not} logically equivalent.

A truth table proof can be given:
\begin{center}
\begin{tabular}{ccc|c c c c}
$p$ & $q$ & $r$ & $q \rightarrow r$ & $p \rightarrow q$ & $p \rightarrow (q \rightarrow r)$ & $(p \rightarrow q) \rightarrow r$
\\ \hline
T & T & T & T & T & T & T \\
T & T & F & F & T & F & F \\
T & F & T & T & F & T & T \\ 
T & F & F & T & F & T & T \\
F & T & T & T & T & T & T \\
F & T & F & F & T & T & F \\
F & F & T & T & T & T & T \\
F & F & F & T & T & T & F \\
\end{tabular}
\end{center}
We see that the last two columns are not the same, and so the statements are not equivalent.
(In particular, when $p$ is false, $q$ is true, and $r$ is false, we check that the first statement
above is true, and the second is false.)

\item(6 points each)
Formalize the arguments below using symbols, and identify whether each is valid or invalid.
Explain your answer. 

You may explain using truth tables, informally in English, or by citing the logical rule of inference
used by name. For at least one argument, please explain using truth tables.
\begin{enumerate}[(a)]
\item
If Frank studied hard for the test, then he got a good score.
Frank did not get a good score.
Therefore, Frank did not study hard for the test.
\\
\\
{\bf Valid.} Write

$p$: Frank studied hard for the test.

$q$: Frank got a good score.

The argument is

$p \rightarrow q$

$\neg q$

$\therefore \neg p$.

This is {\bf modus tollens}, a valid form of inference. This can be proved by the following truth table:
\begin{center}
\begin{tabular}{cc|cc|c}
$p$ & $q$ & $p \rightarrow q$ & $\neg q$ & $\neg p$
\\ \hline
T & T & T & F & F \\
T & F & F & T & F \\
F & T & T & F & T \\
F & F & T & T & T \\
\end{tabular}
\end{center}

We must check that whenever both premises (the third and fourth columns) are true, the conclusion
(the last) column is true. Thus, we only need to check the last row.
\item
If there are as many rational numbers as there are irrational numbers,
then the set of all irrational numbers is infinite.
The set of all irrational numbers is infinite.
Therefore, there are as many rational numbers as there are irrational numbers.
\\
\\
{\bf Invalid.} Write
$p$: There are as many rational numbers as there are irrational numbers.

$q$ The set of all irrational numbers is infinite.

The form of the argument is

$p \rightarrow q$

$q$

$\therefore p$.

This is the {\bf converse error} which is an invalid method of inference. This can be proved by truth tables:
\begin{center}
\begin{tabular}{cc|cc|c}
$p$ & $q$ & $p \rightarrow q$ & $q$ & $p$
\\ \hline
T & T & T & T & T \\
T & F & F & F & T \\
F & T & T & T & F \\
F & F & T & F & F \\
\end{tabular}
\end{center}
We must check that whenever both premises (the third and fourth columns) are true, the conclusion
(the last) column is true. Thus, we need to check the first and third rows, and the third row fails this test.

\end{enumerate}

\item(6 points each)
Indicate which of these statements are true and which are false. 
Explain your answers. ($\mathbb{Z}^+$ indicates the positive integers.)
\begin{enumerate}[(a)]
\item
$\forall x \in \mathbb{Z}^+$, $\exists y \in \mathbb{Z}^+$ such that $x = y + 6$.
\\
\\
{\bf False.} To show this is false, it is enough to find some counterexample. Let $x$ be 3.
Then if $x = y + 6$ then $y$ must be $-3$, and $-3$ is not an element of $\mathbb{Z}^+$.

\item
$\exists x \in \mathbb{R}$ such that $\forall y \in \mathbb{R}$, $x = y + 2.$
\\
\\
{\bf False.} Suppose that there exists some $x$ such that $x = y + 2$ for every $y$.
However, the equation $y = x - 2$ determines $y$ uniquely (in terms of $x$). Therefore,
this cannot be true for all values of $y$.

\end{enumerate}

\item(6 points)
Give an example (with explanation) of an argument which is valid but not sound.
\\
\\
Naturally, answers will vary. Here is one example:
\\
\\
If I am good at math, then I am a rock star.
\\
\\
I am good at math.
\\
\\
Therefore, I am a rock star.
\\
\\
The premises imply the conclusion and so the argument is valid. However, the first premise is
not true (regrettably) and so the argument implies a false conclusion, making it unsound.


\item(6 points)
What does it mean to say that
$$\lim_{x \rightarrow 0} \frac{\sin(x)}{x} \neq 0?$$
\\
\\
This means that for there exists some $\epsilon > 0$ such that for every
$\delta > 0$, there exists $x$ for which $0 < |x - 0| < \delta$ but $|\frac{\sin(x)}{x} - 0| \geq \epsilon$.

Equivalently,
$$
\exists \epsilon > 0 \ \ \forall \delta < 0 \ \ \exists x \ \Big( 0 < |x - 0| < \delta \Big) \wedge
\Big( \Big|\frac{\sin(x)}{x} - 0\Big| \geq \epsilon \Big).
$$
(Either of the above formulations is a good answer.)
\end{enumerate}

\end{document}
