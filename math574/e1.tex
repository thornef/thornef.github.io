\documentclass[12pt]{article}
\textwidth 7.0in

\oddsidemargin -0.4in
\evensidemargin -0.4in
\textheight 9.6in 
\pagestyle{empty}
\usepackage{amssymb,amsmath,amsthm}
\usepackage[mathscr]{euscript}
\usepackage{enumerate, verbatim, url}
\begin{document}
% 8.5in paper width -2x1in margin = 6.5in text width
\setlength{\topmargin}{-12mm}




% 11in paper height -2x1in margin = 9in text height


\begin{center}{\bf Midterm Examination 1 - Math 141, Frank Thorne (thorne@math.sc.edu)}
\end{center}
\begin{center}
{\bf Thursday, February 9, 2012}
\end{center}

Please work without books, notes, calculators, or any assistance from others.
\\
\\
\begin{enumerate}[1.]
\item(3 points each)
For each sentence below, say whether it is logically equivalent to the sentence
\begin{center}
If it's snowing, the temperature must be at or below freezing.
\end{center}
(No explanation for your answers is required.)
\begin{enumerate}[(a)]
\item
If the temperature is above freezing, then it is not snowing.
\item
It only snows if the temperature is at or below freezing.
\item
If it is not snowing, then the temperate must be above freezing.
\item
If the temperature is at or below freezing, then it is snowing.
\end{enumerate}
\vskip 0.2in
\item(3 points each)
For each sentence below, say whether it is logically equivalent to the sentence
\begin{center}
For all animals $A$, if $A$ has claws then $A$ is dangerous.
\end{center}
(No explanation for your answers is required.)
\begin{enumerate}[(a)]
\item
Any animal that is dangerous has claws.
\item
Among all the animals which claws, some are dangerous.
\item
Every animal with claws is dangerous.
\item
Some of all the dangerous animals have claws.
\end{enumerate}
\vskip 0.2in
\item(3 points each)
(For this question, no explanation for your answers is required. Recall that $\in$ means
``is an element of'' and $\subseteq$ means ``is a subset of''. Be careful!)
\begin{enumerate}[(a)]
\item
Is $\{y\} \in \{x, y, z\}$?
\item
Is $\{y\} \in \{x, \{y\}, \{ \{ z \} \} \}$?
\item
Is $\emptyset \subseteq \{42, 1119, \{ \{ \emptyset \} \} \}$?
\item
How many elements are in the set $\{a, \{b, c\} \}$?
\end{enumerate}

\item(4 points each)
Formally rewrite each of the following statements using appropriate truth variables,
predicates, and quantifiers. Specify the domain for all your variables.

For example, an answer may look like
$\forall x \in \mathbb{Z} \ \  \big(P(x) \rightarrow Q(x)\big)$ where you should say what $P$ and $Q$ are.

\begin{enumerate}[(a)]
\item
If it walks like a duck and it quacks like a duck, then it is a duck.
\item
The sum of any two even integers is even.
\end{enumerate}
\vskip 0.2in
\item(5 points each)
(For this question, no explanation for your answers is required.) Write negations of the following
statements.

\begin{enumerate}[(a)]
\item
If Tom is Ann's father, then Jim is her uncle and Sue is her aunt.
\item
There is a computer program that gives the correct answer to every question that is posed to it.
(Your answer should include the phrase ``does not give the correct answer''.)
\end{enumerate}

\item(10 points)
Determine (with proof) whether the following statement forms are logically equivalent.
$$p \rightarrow (q \rightarrow r), \ \ \ \ (p \rightarrow q) \rightarrow r.$$

\item(6 points each)
Formalize the arguments below using symbols, and identify whether each is valid or invalid.
Explain your answer. 

You may explain using truth tables, informally in English, or by citing the logical rule of inference
used by name. For at least one argument, please explain using truth tables.
\begin{enumerate}[(a)]
\item
If Frank studied hard for the test, then he got a good score.
Frank did not get a good score.
Therefore, Frank did not study hard for the test.
\item
If there are as many rational numbers as there are irrational numbers,
then the set of all irrational numbers is infinite.
The set of all irrational numbers is infinite.
Therefore, there are as many rational numbers as there are irrational numbers.
\end{enumerate}

\item(6 points each)
Indicate which of these statements are true and which are false. 
Explain your answers. ($\mathbb{Z}^+$ indicates the positive integers.)
\begin{enumerate}[(a)]
\item
$\forall x \in \mathbb{Z}^+$, $\exists y \in \mathbb{Z}^+$ such that $x = y + 6$.
\item
$\exists x \in \mathbb{R}$ such that $\forall y \in \mathbb{R}$, $x = y + 2.$
\end{enumerate}

\item(6 points)
Give an example (with explanation) of an argument which is valid but not sound.

\item(6 points)
What does it mean to say that
$$\lim_{x \rightarrow 0} \frac{\sin(x)}{x} \neq 0?$$
\end{enumerate}

\end{document}
