\documentclass[12pt]{article}
\textwidth 7.0in
\oddsidemargin -0.4in
\evensidemargin -0.4in
\textheight 9.0in 
\pagestyle{empty}
\usepackage{enumerate, url, amssymb, amsmath}
\begin{document}
\newcommand\Hom{\operatorname{Hom}}
\newcommand\Aut{\operatorname{Aut}}
\newcommand\Out{\operatorname{Out}}
\newcommand\Vol{\operatorname{Vol}}
\newcommand\Covol{\operatorname{Covol}}
\newcommand\C{\mathbb{C}}
\newcommand\F{\mathbb{F}}
\newcommand\Q{\mathbb{Q}}
\newcommand\R{\mathbb{R}}
\newcommand\Z{\mathbb{Z}}
\newcommand\Disc{\operatorname{Disc}}
\newcommand\disc{\operatorname{Disc}}
\newcommand\im{\operatorname{im}}
\newcommand\ord{\operatorname{ord}}

% 8.5in paper width -2x1in margin = 6.5in text width
\setlength{\topmargin}{-2mm}



% 11in paper height -2x1in margin = 9in text height


\begin{center}{\bf Problem Set 1 -- Arithmetic Geometry, Frank Thorne (thorne@math.sc.edu)}
\end{center}
\begin{center}
{\bf Due Friday, September 20, 2024}
\end{center}

\begin{enumerate}[(1)]
\item
Adapting the solution to the Gauss circle problem, compute an asymptotic for the number of lattice points within the ellipse $x^2 + 5y^2 = N$. 
Obtain an explicit bound on the error term without any $O$-notation. 

\item
Recall that {\itshape Minkowski's second theorem} states that the successive minima $\lambda_i$ of a complete lattice $\Lambda$ in $\R^n$ satisfy
\[
\frac{2^n}{n!} \Covol(\Lambda) \leq \lambda_1 \lambda_2 \cdots \lambda_n \cdot \Vol(B(0, 1)) \leq 2^n \Covol(\Lambda).
\]
Prove either of these two inequalities, either as stated, or with any other constant depending only on $n$. 
(Aim for a short and easy solution, rather than the best possible constant.)

\item Let $\alpha$ be a root of $f(x) := x^3 - 4x - 1$, and write $K := \Q(\alpha)$.
\begin{enumerate}[(a)]
\item Verify that $K$ has ring of integers $\Z[\alpha]$ and discriminant $229$, and is totally real. 
\item Explicitly write down the Minkowski embedding of $\mathcal{O}_K$ into $\R^3$. Use a calculator or computer to write
everything down in terms of decimal approximations, to at least three decimal places.
\item If $1, \alpha, \alpha^2$ were vectors whose lengths are the successive minima, would this be consistent with Minkowski's
second theorem? 
\item (Optional) You may wish to try to compute the successive minima, provably or not. This is in general a {\itshape highly} nontrivial 
problem, and you may be interested to poke around the Internet to see what you can learn about this.
\end{enumerate}
\item Suppose, {\itshape contrary to fact}, that the Riemann zeta function was absolutely bounded by some constant $M$ in the region
\[
\{ z \in \C \ : \ \Re(z) \geq \frac{1}{2}, \ \ |z - 1| > \frac{1}{10} \}.
\]
Moreover, define the $k$-divisor function $d_k(n)$ to be the number of ways to write $n$ as a product of $k$ natural numbers.

\begin{enumerate}[(a)]
\item 
Prove the identity
\[
\sum_n d_k(n)^s = \zeta(s)^k,
\]
valid when $\Re(s) = 1$.
\item
By Perron's formula we have
\[
\sum_{n < X} d_k(n) = \frac{1}{2 \pi i} \int_{2 - i \infty}^{2 + i \infty} \zeta(s)^k X^s \frac{ds}{s},
\]
whenever $X$ is a positive real number, not an integer. By shifting a portion of the contour left of the line $\Re(s) = 1$,
obtain an asymptotic formula for $\sum_{n < X} d_k(n)$ with a power saving error term. You should give {\itshape some} description
of all the constants in front of your main terms, although it need not be a simplified one. 

(Answer this relative to our counterfactual assumption.)
\item
Now, suppose instead of our counterfactual bound, we have a bound
\[
|\zeta(\sigma + it)| \ll (1 + |t|)^{\alpha}
\]
for some $\alpha > 0$. Repeat the previous question, obtaining an error term which is presumably worse, but still saves a power of $X$.

(Note: we can take any $\alpha > \frac{1}{4}$ by the so-called `convexity bound', any $\alpha > \frac{32}{205}$ by more intricate
work of Huxley, and any $\alpha > 0$ if the Riemann Hypothesis is true.)

\end{enumerate}

\end{enumerate}



\end{document}


A special case of {\bf Schanuel's Theorem} says that
\[
C(\mathbb{P}^N(\Q), B) := 
\{ P \in \mathbb{P}^N(\Q) \ : \ H(P) \leq B \} \sim \frac{2^N}{\zeta(N + 1)} B^{N + 1}.
\]
Prove this. A template for a proof is given on the next page:
\begin{itemize}
\item If you're unfamiliar with proofs of this nature, following the template should be very helpful. 
\item For more of a challenge, skim the template for a minute and then write up a proof without reading it further.
\item For still more challenge, don't read or follow the template.
\end{itemize}

{\itshape Note that Schanuel proved his result where $\Q$ is replaced with any number field, where the analysis became more difficult.
For definitions of height functions in number fields, see Chapter 8.5 of Silverman.}

\end{enumerate}

\newpage 
The following is a step-by-step outline for Schanuel's theorem. 
\begin{enumerate}[(a)]
\item
Prove that the set to be counted is in an exactly $2$-to-$1$ correspondence 
with $N + 1$-tuples of integers $(x_0, x_1, \cdots, x_N)$, where $x_i \in [-B, B]$ for each $i$, and the $x_i$ do not all share a common factor.
\item
Write 
$C(N, B)$ for the number of $N + 1$-tuples of integers $(x_0, x_1, \cdots, x_N)$, where $x_i \in [-B, B]$ for each $i$ (but with no
`no common factor' condition).
Prove that
\[
C(N, B) = (2B)^{N + 1} + O(B^N).
\]
(Be sure to justify that the constant implied by the $O$-notation is in fact independent of $B$. It may depend on $N$ however. You might wish
to be especially careful and actually compute the constant implied in the error term.)
\item
Let $\mu(d)$ be the {\itshape M\"obius function}, equal to $(-1)^{\omega(d)}$ if $d$ is squarefree, where $\omega(d)$ denotes the number of
prime factors of $d$, and equal to zero otherwise.

Prove that, for any positive integer $n$, $\sum_{d \mid n} \mu(d)$ is equal to $1$ if $n = 1$ and zero otherwise.

(Hint: the sum can be rewritten as $\prod_{p \mid n} (1 + \mu(p))$, where the product is over all primes dividing $n$. Why is this?)
\item
Write 
$C(N, B, d)$ for the number of $N + 1$-tuples of integers $(x_0, x_1, \cdots, x_N)$, where $x_i \in [-B, B]$ for each $i$, such that $d$ divides
all the $x_i$. Explain why $C(N, B, d) = C(N, B/d)$ and deduce an estimate for $C(N, B, d)$.
\item
Prove that 
\[
C(\mathbb{P}^N(\Q), B) = \sum_{d = 1}^B \mu(d) C(N, B, d)
\]
and use your previous estimates to conclude that
\[
C(\mathbb{P}^N(\Q), B) = 2^N B^{N + 1} \sum_{d = 1}^B \frac{\mu(d)}{d^{B + 1}} + o(B^{N + 1}).
\]
\item
Prove that
\[
\sum_{d = B + 1}^\infty \frac{\mu(d)}{d^{N + 1}} = o(1)
\]
and that
\[
\sum_{d = 1}^{\infty} \frac{ \mu(d) }{d^{N + 1}} = \frac{1}{\zeta(B + 1)}. 
\]
Conclude that
\[
\sum_{d = 1}^B \frac{\mu(d)}{d^{N + 1}} = \frac{1}{\zeta(N + 1)} + o(1).
\]
\item
Conclude the statement of Schanuel's theorem.


\end{enumerate}
\end{document}
