\documentclass[12pt]{article}
\textwidth 7.0in
\oddsidemargin -0.4in
\evensidemargin -0.4in
\textheight 9.0in 
\pagestyle{empty}
\usepackage{enumerate, url}
\newcommand{\dx}{dx}
\begin{document}

% 8.5in paper width -2x1in margin = 6.5in text width
\setlength{\topmargin}{-2mm}



% 11in paper height -2x1in margin = 9in text height


\begin{center}{\bf Homework 5 - Math 142, Frank Thorne (thornef@mailbox.sc.edu)}
\end{center}
\begin{center}
{\bf Due Tuesday, September 24.} 
\end{center}

\begin{enumerate}[(a)]
\item
{\bf To be added: Problems from 8.1.}
\item
Find the volume of a sphere with radius $r$.

\item
Find the volume of a hollowed out sphere of radius $r$, with a smaller sphere
of radius $s$ removed from the center. (Hint: there is an easy way!)

\item
Find the volume of a circular cone of radius $r$ and height $h$.

\item
Find the area of a square pyramid with base length $b$ and height $b$.
\\
\\
{\bf Important.} For all volume problems, please sketch the solid whose volume
you are computing, and draw and label a typical slice.
\item
Stewart, Ch. 6.2, 1-10; even required, odd recommended.

\item
Stewart, Ch. 6.2, 41, 42, 51, 68, 70.
%\item
%What is the formula for the arc length of a curve? Thoroughly explain why this formula
%is correct, and draw a picture.
%\item
%Use the arc length formula to calculate the length of the graph $y = x$ from $x = 0$ to
%$x = 2$. Why do you know your answer is correct?
%\item
%Use the arc length formula to calculate the length of the graph $y = \sqrt{1 - x^2}$ from
%$x = -1$ to $x = 1$. Why do you know your answer is correct?
%\item
%Stewart, 8.1, 7-14. Before you do your computation, 
%draw rough sketches of each of the functions, and use your sketch to guess the arc length.
%Your guesses should be fairly close, but not exactly correct.
\end{enumerate}
Additional problems:

\begin{enumerate}[(a)]
\item
Stewart, 6.2, 1-10 odd%; 8.1, 15-18 (and follow the instructions given for 7-14 above).
\end{enumerate}
{\bf Bonus} (2 points):
\begin{enumerate}[(a)]
\item
Stewart, "Discovery Project" on p. 532. There will be no contest, but find something interesting
to say.
\end{enumerate}

\end{document}
