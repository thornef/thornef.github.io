\documentclass[12pt]{article}
\textwidth 7.0in
\oddsidemargin -0.4in
\evensidemargin -0.4in
\textheight 9.0in 
\pagestyle{empty}
\usepackage{enumerate, url}
\newcommand{\dx}{dx}
\begin{document}

% 8.5in paper width -2x1in margin = 6.5in text width
\setlength{\topmargin}{-2mm}



% 11in paper height -2x1in margin = 9in text height


\begin{center}{\bf Homework 9 - Math 142, Frank Thorne (thornef@mailbox.sc.edu)}
\end{center}
\begin{center}
{\bf Due Friday, October 31}
\end{center}

\begin{enumerate}[(a)]

\item
What is the integral test? Explain why it works.

\item
11.3, 11-20. In addition:
\begin{itemize}
\item
For each series which diverges, if you use the integral test, then draw a graph which represents
both the series and the integral you're comparing it to.
\item
For each series which converges, give upper and lower bounds on the value of your series which are
guaranteed to be accurate within $0.01$. Draw a graph which represents your lower bound.
\item
Note that answers such as
\[
1 + \frac{1}{2 \sqrt{2}} + \frac{1}{3 \sqrt{3}} + \cdots + \frac{1}{99 \sqrt{99}} + \frac{1}{20} <
\sum_{n = 1}^{\infty} a(n),
\]
\[
1 + \frac{1}{2 \sqrt{2}} + \frac{1}{3 \sqrt{3}} + \cdots + \frac{1}{99 \sqrt{99}} + \frac{1}{100 \sqrt{100}} +
\frac{1}{20} >
\sum_{n = 1}^{\infty} a(n)
\]
are acceptable and expected. It will often be impractical to simplify the expressions you get. (But if you get
something easy, then please simplify it.)
\end{itemize}

\end{enumerate}
Additional problems:
11.3, 23-26 (Same instructions as above).
\\
\\
Bonus: Do 11.3, 22 subject to the instructions above. What is unusual about this computation?

\end{document}
