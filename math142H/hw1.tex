\documentclass[12pt]{article}
\textwidth 7.0in
\oddsidemargin -0.4in
\evensidemargin -0.4in
\textheight 9.0in 
\pagestyle{empty}
\usepackage{enumerate}
\begin{document}
% 8.5in paper width -2x1in margin = 6.5in text width
\setlength{\topmargin}{-2mm}



% 11in paper height -2x1in margin = 9in text height


\begin{center}{\bf Homework 1 - Math 142, Frank Thorne (thornef@mailbox.sc.edu)}
\end{center}
\begin{center}
{\bf Due Friday, August 29}
\end{center}

{\bf Important:} As with everything else in life, being right is not enough.
Please show your work, write in complete sentences, and explain your reasoning clearly. 
\\
\\
{\bf Thompson.}
Read Chapters 1-3 of Thompson, and in approximately a page (of mostly plain English, but include
equations and diagrams where appropriate) answer the following question.

Thompson defines the notion of a {\itshape derivative} without explicit reference to a
{\itshape limit}. What is Thompson's definition, and how does it relate to the usual
definition in Stewart?
\\
\\
{\bf Required problems.}

\begin{enumerate}[(a)]
\item
What is a function? (This is the most important question in all of mathematics.)

\item
Suppose that $f$ is a function whose domain and range are subsets of the real numbers.
Explain how to draw the graph of $f$, and what the graph represents.

\item
Explain what it means to say that
$$\lim_{x \rightarrow a} f(x) = c.$$

\item
Define the {\itshape derivative} of a function. Define it using an equation, and also explain
your definition in English. In addition, draw a picture and explain why your equation describes
the tangent line to the graph.

\item
Define an {\itshape inverse function.}
Explain how to find the derivative of $f^{-1}(x)$, if you know the derivative of $f(x)$.

\item
Define the functions $\sin(x)$, $\cos(x)$, $\tan(x)$, $\csc(x)$, $\sec(x)$, $\cot(x)$, $e^x$, and
$\ln(x)$, and say what each of their derivatives is. For $\sin(x)$, $\cos(x)$, and $e^x$, it is okay
to just state the derivatives. For the other functions, explain how you got your answer. (You will want
to use the derivatives of sin, cosine, and $e^x$!)

\item
State the power rule, the product rule, the quotient rule, and the chain rule for finding derivatives.

\item
Stewart, Ch. 3 Review, pp. 262-263: 6-13, 57, 89 (do (d) without a graphing calculator),
93.

\item
Define the {\itshape antiderivative} of a function.

\item
Define the {\itshape definite integral}
$$\int_a^b f(x) dx.$$
Give an equation, and explain why your equation gives the area underneath the curve from $x = a$
to $x =b$.

\item
What does the fundamental theorem of calculus say? Why is it true? Explain thoroughly.

\item
Stewart, Ch. 5.3, 19-24.

\end{enumerate}
{\bf Additional problems.}
\begin{enumerate}[(a)]
\item
Stewart, Ch. 3 Review, 1-5, 14-22.

\item
Stewart, Ch. 5.3, 25-32.

\end{enumerate}
{\bf Bonus} (2 points).
\begin{enumerate}[(a)]
\item
What is the Mean Value Theorem, and why does anybody care about it?

\end{enumerate}

\end{document}
