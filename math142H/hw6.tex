\documentclass[12pt]{article}
\textwidth 7.0in
\oddsidemargin -0.4in
\evensidemargin -0.4in
\textheight 9.0in 
\pagestyle{empty}
\usepackage{enumerate}
\begin{document}
% 8.5in paper width -2x1in margin = 6.5in text width
\setlength{\topmargin}{-2mm}



% 11in paper height -2x1in margin = 9in text height


\begin{center}{\bf Homework 6 - Math 142, Frank Thorne (thornef@mailbox.sc.edu)}
\end{center}
\begin{center}
{\bf Due Friday, October 3.}
\end{center}

\begin{enumerate}[(a)]

\item
What is a parametric curve? Draw a picture and give an example.

\item
Given a curve defined by parametric equations $x = f(t)$, $y = g(t)$, 
which variables are functions of the other variables?

\item
Give (and draw) examples of parametric curves that illustrate the following possibilities.
Explain thoroughly.

\begin{itemize}
\item
$y$ can be described as a function of $x$.
\item
$x$ can be described as a function of $y$.
\item
Both of the above are true.
\item
Neither of the above are true.
\end{itemize}

\item
Suppose you have a function $y = f(x)$ that you want to define by parametric
equations. How would you do it?

Could you do it a different way?

Is there any limit to the number of ways you could do it?
\item
Give parametric equations for a line, a circle, an ellipse (other than a circle), and a
parabola. Graph all of your curves.

\item
A bug sits on the edge of a bicycle tire while you ride the bicycle, and does not move
(relative to its position on the tire). Assume that the bicycle moves forward at a constant
speed and that the bug starts at the bottom of the wheel.

Give parametric equations describing the bug's motion, and draw a graph which describes it.

\item
Same as above, but now assume that the bug sits on a bicycle spoke, initially pointing
downwards, halfway between the center and the edge of the wheel.

\item
Same as above, but now assume that the bug sits at the center of the wheel.

\item
If you are given a parametric curve $y = f(t)$, $g = f(t)$, how do you find $\frac{dy}{dx}$?
Give the formula, explain why it is true, draw a picture, and give an example.

\item
10.1, 1-10, 24-28.

\item
Find $\frac{dy}{dx}$ for each of the ``bug'' problems described above. Draw sample tangent lines
on your graph, and explain how your answers differ from each other. 
\item
10.2, 1-10.
\end{enumerate}
Additional problems: 
\begin{enumerate}[(a)]
\item
10.1, 11-14.
\item
10.2: Graph all of the curves sketched in 1-8 along with their tangent lines.
\end{enumerate}
Bonus (2 points): 10.2, 73.
\end{document}
