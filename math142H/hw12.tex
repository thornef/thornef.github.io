\documentclass[12pt]{article}
\textwidth 7.0in
\oddsidemargin -0.4in
\evensidemargin -0.4in
\textheight 9.0in 
\pagestyle{empty}
\usepackage{enumerate, url}
\newcommand{\dx}{dx}
\begin{document}

% 8.5in paper width -2x1in margin = 6.5in text width
\setlength{\topmargin}{-2mm}



% 11in paper height -2x1in margin = 9in text height


\begin{center}{\bf Homework 12 - Math 142, Frank Thorne (thornef@mailbox.sc.edu)}
\end{center}
\begin{center}
{\bf Due Tuesday, November 19}
\end{center}
To be added: Some problems from 11.10.

\begin{enumerate}[(a)]
\item
What is a power series?

\item
Describe an example of a power series which converges for all values of $x$.

\item
Describe an example of a power series which converges only for some values of $x$.

\item
Are there any power series which converge for no values of $x$?

\item
What is the radius of convergence of a power series? How do you find it?

\item
11.8, 5-18. However, you do not have to test the endpoints of the interval for convergence.

\item
What is a Taylor series? Why is the formula for it true?

\item
Find the Maclaurin (Taylor) series for the following functions. Determine their radii of convergence.
\begin{itemize}
\item $f(x) = x^2$
\item $f(x) = e^x$
\item $f(x) = e^{2x}$
\item $f(x) = \cos(x)$
\item $f(x) = \sin(x)$
\item $f(x) = \cos(4x)$
\item $f(x) = \sin(x^2)$
\item $f(x) = x^3 \sin(x)$.
\item $f(x) = x + e^x$.
\end{itemize}
\item
Explain why the Taylor series for $e^x$ gives you a formula for $e$.
\item
Compute $e$, as a fraction or decimal, to fairly good accuracy. Your estimate should plausibly be within $\frac{1}{10}$, but you don't need to show this.
\item
Compute $1/e$, as a fraction or decimal, to fairly good accuracy. Your estimate should plausibly be within $\frac{1}{100}$,
but you don't need to show this.
\item
Compute $\sin(1/10)$, as a fraction or decimal, to fairly good accuracy. Your estimate should plausibly be within $\frac{1}{100}$,
\item
Compute $\sqrt{1.1}$, as a fraction or decimal, 
to fairly good accuracy. Your estimate should plausibly be within $\frac{1}{1000}$,
but you don't need to show this.
\item
Stewart, 11.10, 29, 30.
\end{enumerate}
Additional problems:
\begin{enumerate}[(a)]
\item
Stewart, 11.8, 19-24, 11.10, 15, 16, 31-36.

\end{enumerate}
Bonus: Use Taylor series to explain why $e^{ix} = \cos(x) + i \sin(x)$.
\\
\\

\end{document}
