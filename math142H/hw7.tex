\documentclass[12pt]{article}
\textwidth 7.0in
\oddsidemargin -0.4in
\evensidemargin -0.4in
\textheight 9.0in 
\pagestyle{empty}
\usepackage{enumerate}
\begin{document}
% 8.5in paper width -2x1in margin = 6.5in text width
\setlength{\topmargin}{-2mm}



% 11in paper height -2x1in margin = 9in text height


\begin{center}{\bf Homework 7 - Math 142, Frank Thorne (thornef@mailbox.sc.edu)}
\end{center}
\begin{center}
{\bf Due Friday, October 11}
\end{center}

\begin{enumerate}[(a)]
\item
What are polar coordinates?

\item
10.3, 1-12, 29-40, 56.

\item
What is the formula for the area of a region defined by polar coordinates? Why is it
true? Draw a picture, explain, and give an example.

\item
What is the formula for the slope of the tangent line to a curve defined by polar coordinates? 

\item
10.3, 57-60.

\item
10.4, 5-12.
\end{enumerate}

Additional problems:
\begin{enumerate}
\item
10.3, 41-48; 10.4, 13-14, 17-20.
\end{enumerate}
Bonus: Look through the exercises at all the ridiculous Greek or Greek-sounding
names for the various parametric curves. Find out, beyond the extent described in your
book, how at least two of them got their name.
\end{document}
