%This is for amsart
%\documentclass[12pt]{amsart}

%This is for article   
\documentclass[11pt, leqno]{article}
\newenvironment{proof}{\noindent {\bf Proof:}}{$\Box$ \vspace{2 ex}}

\usepackage{amsmath, amssymb, amscd}
\usepackage{latexsym, epsfig, color, url, relsize}
\usepackage{tikz, verbatim}

\usepackage{fullpage}    %HELLO!!!!

\newcommand*\mycirc[1]{%
  \begin{tikzpicture}
      \node[draw,circle,inner sep=1pt] {#1};
   \end{tikzpicture}}
   
\newcommand{\calP}{\mathcal{P}}
\newcommand{\C}{\mathbb{C}}
\newcommand{\N}{{\rm{N}}}
\newcommand{\ns}{{\rm{ns}}}
\newcommand{\PGL}{{\rm{PGL}}}
\newcommand{\Tr}{{\rm{Tr}}}
\renewcommand{\char}{{\rm{char}}}
\newcommand{\Z}{\mathbb{Z}}
\renewcommand{\P}{\mathbb{P}}
\newcommand{\Q}{\mathbb{Q}}
\newcommand{\R}{\mathbb{R}}
\newcommand{\F}{\mathbb{F}}
\newcommand{\A}{\mathbb{A}}
\newcommand{\cO}{\mathcal{O}}
\newcommand{\co}{\mathcal{O}}
\newcommand{\calB}{\mathcal{B}}
\newcommand{\nz}{+}
\newcommand{\QQ}{\mathcal{Q}}
\newcommand{\calW}{\mathcal{W}}
\newcommand{\GQ}{G_{\Q}}
\newcommand{\Qbar}{\bar{\Q}}
\newcommand{\ra}{\rangle}
\newcommand{\la}{\langle}
\newcommand{\beq}{\begin{equation}}
\newcommand{\eeq}{\end{equation}}
\newcommand{\calO}{\mathcal{O}}
\newcommand{\calF}{\mathcal{F}}
\newcommand{\wP}{\widehat{\Phi}}
\newcommand{\charfn}{\mathbf{1}}
\newcommand{\sillyX}{\mathfrak{X}}
\newcommand{\sumf}{\sideset{}{{}^\flat}\sum}


\newcommand{\six}[6]{ #1 #2 #3 #4 #5 #6}
\newcommand{\tw}[2]{#1 #2}
%\newcommand{\tw}[2]{ \begin{matrix} #1 \\ #2 \end{matrix}}

\newcommand\Hom{\operatorname{Hom}}
\newcommand\Aut{\operatorname{Aut}}
\newcommand\Out{\operatorname{Out}}
\newcommand\Vol{\operatorname{Vol}}
\newcommand\Disc{\operatorname{Disc}}
\newcommand\disc{\operatorname{Disc}}
\newcommand\im{\operatorname{im}}
\newcommand\Stab{\operatorname{Stab}}
\newcommand\Gal{\operatorname{Gal}}
\newcommand\Res{\operatorname{Res}}
\newcommand\SL{\operatorname{SL}}
\newcommand\GL{\operatorname{GL}}
\newcommand\gl{\operatorname{GL}}
\newcommand\Dist{\operatorname{Dist}}
\newcommand\dist{\operatorname{dist}}
\newcommand\lcm{\operatorname{lcm}}
\newcommand\fc{\operatorname{fc}}
\newcommand\calI{\mathcal{I}}

%\newcommand\beq{\begin{equation}}
%\newcommand\eeq{\end{equation}}

\newcommand\cF{\mathcal{F}}
\newcommand\Sym{\operatorname{Sym}}
\newcommand\tra{\mathsmaller T}
%\newcommand{\tw}[12]{ \begin{matrix} #1 #2 #3 #4 #5 #6 \\ #7 #8 #9 #10 #11 #12}
%This numbers everything by section
\newtheorem{proposition}{Proposition}%[section]
\newtheorem{theorem}[proposition]{Theorem}
\newtheorem{corollary}[proposition]{Corollary}
\newtheorem{example}[proposition]{Example}
\newtheorem{question}[proposition]{Question}
\newtheorem{lemma}[proposition]{Lemma}
\newtheorem{remark}[proposition]{Remark}
\newtheorem{defn}[proposition]{Definition}
\newtheorem{exm}[proposition]{Example}
\newtheorem{assmp}{Assumption}
\newtheorem{aim}{Aim}


%This will keep definitions and notations unnumbered and non-italicized.
\newenvironment{definition}{\vspace{2 ex}{\noindent{\bf Definition. }}}{\vspace{2 ex}}
\newenvironment{notation}{\vspace{2 ex}{\noindent{\bf Notation. }}}{\vspace{2 ex}}


\title{766666666622222222222222229950000003333333333311488}
    
\author{Speaker: Michael Filaseta}
       
\begin{document}

\maketitle
The eponymous number of the title illustrates {\it blocks} in base $10$: it has ten of them -- the initial $7$, the string of sixes, etc.
According to the abstract, it is the largest number divisible by $2^{52}$, relatively prime to 5, which consists of blocks where no two of the blocks are formed using the same digit.
(This fact was not actually mentioned during the talk however.)

There were two main theorems. The most interesting of them (in my view), and the one to which Michael devoted most of his energy to proving, is the following:
if $a$ and $b$ are integers with $\frac{ \log a}{ \log b} \not \in \Q$, then $\lim_{n \rightarrow \infty} B(a^n, b) = \infty$, where $B(x, b)$ denotes the number of blocks
in the integer $x$ when written base $b$. So, for example, it is not possible that arbitrarily high powers of $7$ could consist of a string of all ones, or a string of twos followed by
a string of ones, etc. This feels decidedly nontrivial.

Indeed, nontrivial enough that it feels like it shouldn't just follow from some sort of elementary manipulations. A big tool should be required, and here that tool is {\it Baker's theorem
on linear forms of logarithms}. That theorem says the following. Suppose we are given algebraic numbers $\alpha_1$ through $\alpha_r$ and $\beta_0$ through $\beta_r$ such that
the quantity
\[
\Lambda := \beta_0 + \beta_1 \log(\alpha_1) + \cdots + \beta_r \log(\alpha_r)
\]
is nonzero. Then its absolute value can be bounded below. The bound is explicit, and depends on the {\itshape heights} of the algebraic numbers involved. The height of an integer
is just its absolute value, the height of a rational number (written in lowest terms) is the maximum of the height of the numerator and demoninator, and the height of a general
algebraic number is the absolute value of the largest coefficient of its minimal polynomial (written over $\Z$). An application which Michael presented is that (for a certain constant $C$)
$|\log(2) - \frac{a}{b}| > \frac{1}{|b|^C}$ for every rational number $\frac{a}{b}$.

This brought Michael to the main body of his talk. He first reinterpreted the main theorem in terms of polynomial identities rather than `blocks', and established that
it follows from a finiteness lemma (Lemma 1) for certain systems of polynomial equalities and inequalities. He then began to present a complete proof of how this lemma from Baker's theorem. (He didn't finish his talk; he pledged to do so next week, although I will be out of town for the sequel.)

For me, the talk illustrated a difference between Michael's taste in mathematics and my own, as it illustrated a strength of Michael's which I don't share. He clearly put a lot of
his effort into his presentation, but the end of the talk involved a huge variety of notation, as well as identities which I didn't really `understand', and I became lost during the proof.
I tend to appreciate `machinery' to hang my hat on, but Michael demonstrated his mastery of working with equations exactly as they are, without trying to fit them onto
some other framework. Ultimately he (and his collaborators) succeeded in deducing a striking result from Baker's theorem.

\end{document}