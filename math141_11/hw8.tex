\documentclass[12pt]{article}
\textwidth 7.0in
\oddsidemargin -0.4in
\evensidemargin -0.4in
\textheight 9.0in 
\pagestyle{empty}
\usepackage{enumerate}
\begin{document}
% 8.5in paper width -2x1in margin = 6.5in text width
\setlength{\topmargin}{-2mm}



% 11in paper height -2x1in margin = 9in text height


\begin{center}{\bf Homework 8 - Math 141, Frank Thorne (thornef@mailbox.sc.edu)}
\end{center}
\begin{center}
{\bf Due Friday, October 14}
\end{center}

These questions will not appear on the second midterm exam. 
\\
\\
\begin{enumerate}[(a)]
\item
What does Rolle's Theorem say? Draw a picture and explain.

\item
What does the Mean Value Theorem say? Draw a picture and explain.

\item
You drive 120 miles from Columbia to Charlotte, and it takes you exactly two hours.
At some point, your speed is exactly 60 mph. Explain how you know this. Must there be
some point at which your speed is exactly 65 mph?

\item
({\bf Careful!} You have to think about this one...) In the question above, must there
be some point at which your speed is exactly 55 mph?

\item
You throw a ball straight up in the air, and eventually it falls back to the earth.
Is there some time in its flight when the ball is not moving at all? Explain.

\item
Consider the function $f(x) = |x|$. Find its derivative. We have $f(1) = f(-1) = 0$,
but show that there is no point $x$ for which $f'(x) = 0$. Explain why this doesn't contradict
the Mean Value Theorem. Draw a picture which illustrates your conclusions.

\item
Consider the function $f(x) = x^2$. Use the Mean Value Theorem to show
that there is a $c \in (-1, 2)$ with $f'(c) = 1$. Then, find $f'(x)$ and figure
out all possible values of $c$. Draw a picture which illustrates your conclusions.

\item
Consider the function $f(x) = \sin(x)$. Use the Mean Value Theorem to show
that there is a $c \in (0, 4\pi)$ with $f'(c) = 0$. Then, find $f'(x)$ and figure
out all possible values of $c$. Draw a picture which illustrates your conclusions.

\item
Consider the function $f(x) = \sin(x)$. Use the Mean Value Theorem to show
that there is a $c \in (0, 5\pi/2)$ with $f'(c) = 2/5\pi$. Then, find $f'(x)$ and figure
out, {\bf approximately}, all possible values of $c$. Draw a picture which illustrates your conclusions.

\item
Stewart, Ch. 4.2, 7-8.

\item
Stewart, Ch. 4.2, 1-4, 11-14; even required, odd recommended.

\end{enumerate}

\end{document}
