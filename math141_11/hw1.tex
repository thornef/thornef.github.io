\documentclass[12pt]{article}
\textwidth 7.0in
\oddsidemargin -0.4in
\evensidemargin -0.4in
\textheight 9.0in 
\pagestyle{empty}
\usepackage{enumerate}
\begin{document}
% 8.5in paper width -2x1in margin = 6.5in text width
\setlength{\topmargin}{-2mm}



% 11in paper height -2x1in margin = 9in text height


\author{Math 42 Pretest - Frank Thorne}
\begin{center}{\bf Homework 1 - Math 141, Frank Thorne (thornef@mailbox.sc.edu)}
\end{center}
\begin{center}
{\bf Due Friday, August 26}
\end{center}

{\bf Important:} As with everything else in life, being right is not enough.
Please show your work, write in complete sentences, and explain your reasoning clearly. 
\\
\\
\begin{enumerate}[(a)]
\item 
Stewart, Ch. 1.1, 1, 5-6, 12, 13.

\item
What is a function? (This is the most important question in all of mathematics.)

\item
Describe examples of functions from at least three of the following categories:
biology; physics or chemistry; geometry; economics or business; geography.

\item
What are the domain and the range of a function? Give an example of a function whose
domain is $[0, 5]$ and whose range is $[0, 3]$.

\item
Does the equation $x^2 + y^2 = 1$ describe $y$ as a function of $x$? Why or why not?
Answer the same for the equation $x^2 + y = 1$.

\item
Stewart, Ch. 1.2, 10-12, 16.

\item
Define the trigonometric 
functions $\sin(x)$, $\cos(x)$, $\tan(x)$, $\sec(x)$, $\csc(x)$,
and $\cot(x)$.

\item
Stewart, Ch. 1.3, 11-18 ({\bf show your work}), 31, 32, 53, 56.

\item
Define the exponential and logarithmic functions $e^x$ and $\ln x$. 

\item
Stewart, Ch. 1.5, 9-10.

\item
Define the term {\itshape inverse function}. Give an example of a function that has an inverse,
and of a function that does not.

\item
Define the logarithmic functions $\log_a(x)$ and $\ln(x)$.

\item
Stewart, Ch. 1.6, 18 (in addition, graph the inverse of $f$), 21-24, 47-48.

\end{enumerate}

\end{document}
