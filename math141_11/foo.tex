\documentclass[12pt]{article}
\textwidth 7.0in
\oddsidemargin -0.4in
\evensidemargin -0.4in
\textheight 9.0in 
\pagestyle{empty}
\usepackage{enumerate}
\begin{document}
% 8.5in paper width -2x1in margin = 6.5in text width
\setlength{\topmargin}{-0.6in}



% 11in paper height -2x1in margin = 9in text height


\begin{center}{\bf Examination 1 - Math 141, Frank Thorne (thornef@mailbox.sc.edu)}
\end{center}
\begin{center}
{\bf Wednesday, September 14, 2011}
\end{center}

Please work without books, notes, calculators, or any assistance from others. If you have
any questions, feel free to ask me. 

Please do your work on separate paper; you should staple this sheet to your work (put this on top)
and turn in everything together. 
\\
\\
{\bf YOUR NAME: }
\\
\\
\begin{enumerate}[(1)]
\item
Sketch a rough graph of the outdoor temperature as a function of time during a
typical spring day.

\item
Graph the function $y = 1 + 2 \cos x$, not by plotting points, but by
starting with the graph of $y = \cos x$ and applying the appropriate transformations.
(Be sure to explicitly explain your work.)

\item
Explain what is meant by the equation
$$\lim_{x \rightarrow 2} f(x) = 5.$$
Is it possible for this statement to be true and yet $f(2) = 3$? Explain.

\item
Evaluate the limit
$$\lim_{x \rightarrow -1} \frac{x^2 - 4x}{x^2 - 3x - 4}$$
if it exists.

\item
Evaluate the limit
$$\lim_{x \rightarrow - \infty} \frac{1 - x - x^2}{2 x^2 - 7}$$
if it exists.

\item
Evaluate the limit
$$\lim_{x \rightarrow 0} \bigg( \frac{1}{x} - \frac{1}{x^2 + x}\bigg)$$
if it exists.

\item
Give the definition of the {\itshape derivative} of a function
$f(x)$ at the point $x = a$. (Please give the algebraic definition,
using an equation.)

Draw a picture and explain why your
equation gives the slope of the tangent line to the graph of $f(x)$
at $x = a$.

\item
If $f(x) = 1 - x^3$, find $f'(0)$ (directly from the definition 
of the derivative) and use it to find an equation of the tangent line
to the curve $y = 1 - x^3$ at the point $(0, 1)$.

\end{enumerate}

\end{document}
