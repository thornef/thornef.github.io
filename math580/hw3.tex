\documentclass[12pt]{article}
\textwidth 7.0in
\oddsidemargin -0.4in
\evensidemargin -0.4in
\textheight 9.0in 
\pagestyle{empty}
\usepackage{enumerate}
\begin{document}
% 8.5in paper width -2x1in margin = 6.5in text width
\setlength{\topmargin}{-2mm}



% 11in paper height -2x1in margin = 9in text height


\begin{center}{\bf Homework 3 - Math 580, Frank Thorne (thornef@mailbox.sc.edu)}
\end{center}
\begin{center}
{\bf Due Friday, September 26}
\end{center}
\begin{enumerate}[(1)]
\item 
Write out the multiplication table for the integers modulo 11.

If your answer is correct, each row and each column (other than the one with all zeroes)
will contain every residue class modulo 11. Why do you know that this must be the case?
\item
Write out the multiplication table for the integers mod 10.

The phenomenon of the previous problem will {\itshape not} happen again. Explain why not.
\item
Suppose that $11$ divides the sum of two squares. Prove that $11$ divides each of the squares.
\item
Dudley, p. 32-33: 1, 3, 4, 5, 7, 9, 13, 15, 16.
\item
Dudley, p. 32-33, *either* 2, 6, 8, 10, 12, 14 *or* 19, 20.
\item
Dudley, p. 40-41, 1, 3, 6, 7, 10, 12, 13, 14, 15, 16.
\end{enumerate}
{\bf Bonus problems:}
\begin{enumerate}[(1)]
\item
Dudley, p. 40-41, 18, 20.
\item
Let $p$ be a prime with $p \equiv 3 \pmod 4$. Suppose that $p$ divides the sum of two squares.
Prove that $p$ divides each of the squares.
\item
Write out a multiplication table modulo $m$ (you must determine a suitable value of $m$)
such that the number $3$ appears exactly six times in some row.

(It is okay to write out only part of the multiplication table once you have shown the existence
of the row you claim.)
\end{enumerate}

\end{document}
