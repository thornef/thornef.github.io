\documentclass[12pt]{article}
\textwidth 7.0in
\oddsidemargin -0.4in
\evensidemargin -0.4in
\textheight 9.0in 
\pagestyle{empty}
\usepackage{enumerate}
\begin{document}
% 8.5in paper width -2x1in margin = 6.5in text width
\setlength{\topmargin}{-2mm}



% 11in paper height -2x1in margin = 9in text height


\begin{center}{\bf Homework 2 - Math 580, Frank Thorne (thornef@mailbox.sc.edu)}
\end{center}
\begin{center}
{\bf Due Thursday, September 11}
\end{center}
\begin{enumerate}[(1)]
\item 
Dudley, p. 19: 2, 6, 13, 15; p. 26, 1, 2, 5, 7, 10; p. 33, 1, 3, 4, 8.

\item
Let $S$ be the set of integers $3n + 1$, for $n \geq 0$. Recall that $n$ is prime
in $S$ if it has no divisors other than $1$ and itself which are in $S$.

Find the first ten primes in $S$, and determine whether unique factorization holds in $S$.

\item
Repeat the above problem, where $S$ is now the set of integers $5n + 1$.

\item
Repeat the above problem, where $S$ is now the set of integers coprime to $7$.

\end{enumerate}
{\bf Bonus problems:}
\begin{enumerate}[(1)]
\item
Generalizing the examples of $S$ above, conjecture one or more theorems which you believe to be true,
which describe whether or not these types of sets have unique factorization.

Prove as much as you can. Your proofs might depend on statements which you suspect are true but don't know
how to prove. If so, say so.

\end{enumerate}

\end{document}
