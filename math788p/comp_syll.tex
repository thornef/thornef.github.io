\documentclass[12pt]{article}
\usepackage{amssymb,amsmath,amsthm}
\usepackage{enumerate, verbatim}
\textwidth 7.0in
\oddsidemargin -0.4in
\evensidemargin -0.4in
\textheight 9.0in 
\pagestyle{empty}
\newcommand{\p}{\mathfrak{p}}
\newcommand{\Q}{\mathbb{Q}}
\newcommand{\Z}{\mathbb{Z}}
\begin{document}
% 8.5in paper width -2x1in margin = 6.5in text width
\setlength{\topmargin}{-2mm}



% 11in paper height -2x1in margin = 9in text height


\begin{center}{\bf Comprehensive exam syllabus for Math 788p, Algebraic Number Theory (Spring 2013)}
\end{center}
\begin{center}Frank Thorne, thorne@math.sc.edu
\end{center}

{\itshape Note: Probably this will be combined with Michael Filaseta's comp for Fall 2012. He covered
the basics, which you should be sure to learn very well. The list of topics here is mostly new, but there
might be some overlaps: that means we both agree it's really important.}

\begin{enumerate}
\item
Know the definitions of a {\itshape ring of integers} and a {\itshape discriminant}, and be able to compute them.
Be able to compute rings of integers and discriminants for arbitrary number fields. (This can get {\itshape very}
messy, and you will not be asked extremely messy problems on the exam.)

\item
Understand that unique factorization may or may not hold in rings of integers. Be able to prove it in typical
examples, such as $\Z[i]$, and be able to disprove it in typical examples, such as $\Z[\sqrt{-5}]$.

\item
Be familiar with the $p$-adic numbers $\Z_p$ and $\Q_p$. Understand and work with definitions in terms of an
inverse limit, or in terms of a completion, and prove that they are the same. Understand and be able to prove
the ideal structure of $\Z_p$. Prove that $\Z$ and $\Z_{(p)}$ embed into $Z_p$, and that $\Q$ embeds into $\Q_{(p)}$.

\item
Solve equations in $\Z_p$ and $\Q_p$. Be able to represent fractions, square roots, etc. as $p$-adic
numbers where appropriate. When no such representations exist, be able to prove it. Know the statement of Hensel's lemma.

\item
Be able to describe the deomposition of prime ideals in number fields. Know the Chinese remainder theorem; be able
to compute norms of ideals and elements. Know the $efg$ theorem, and be able to enumerate the possible splittings of
primes in extensions (including the Galois case, covered later in the semester).

\item
Know the theorem relating factorization of polynomials mod $p$ to factorization of $p$ in rings of integers of number fields.
Use it to compute splitting of prime ideals in extensions. 

\item
Be able to define fractional ideals and the class group. Compute the class group in appropriate special cases. (You will
not be asked to memorize the Minkowski bound.)

\item
Know the statement of Dirichlet's unit theorem. Be able to find the torsion part of the unit group, and be able to use the
theorem to compute the rank. 

\item
Know the definitions of the decomposition and inertia groups. Know what their sizes are, and understand why. Be able to
prove elementary consequences, for example as seen in the homework.

\item
Know the definition of the Artin symbol, be able to prove its basic properties, and know how to compute it in quadratic fields,
and in cyclotomic fields and their subfields. Understand the relation between the Artin symbol and Galois groups of polynomials;
be able to verify (for example) that individual number fields have Galois group $S_n$ based on the Artin symbol.

\end{enumerate}

\end{document}
