\documentclass[12pt]{article}
\usepackage{amssymb,amsmath,amsthm}
\usepackage{enumerate, verbatim}
\textwidth 7.0in
\oddsidemargin -0.4in
\evensidemargin -0.4in
\textheight 9.0in 
\pagestyle{empty}
\newcommand{\p}{\mathfrak{p}}
\newcommand{\Q}{\mathbb{Q}}
\newcommand{\Z}{\mathbb{Z}}
\begin{document}
% 8.5in paper width -2x1in margin = 6.5in text width
\setlength{\topmargin}{-2mm}



% 11in paper height -2x1in margin = 9in text height


\begin{center}{\bf Algebraic number theory (Spring 2013), Homework 3}
\end{center}
\begin{center}Frank Thorne, thornef@webmail.sc.edu
\end{center}
\begin{center}
{\bf Due Monday, February 10}
\end{center}

\begin{enumerate}
\item (5 points)
Represent $23$, $\frac{1}{4}$, $-7$, and $-\frac{1}{14}$ as $7$-adic numbers. Which of them
are $7$-adic integers?

\item (5 points)
Write out a formal proof that there exists an injection $\Z_{(p)} \rightarrow \Z_p$.

\item (*7 points)
Look up and write out the definition of an {\itshape inverse limit} in general,
in terms of a universal property. (For example, see the Wikipedia page.) Prove that $\Z_p$
is the inverse limit of the rings $\Z/(p^n)$, under the projection morphisms, according
to this definition.

\item (5 points) Represent $\sqrt{6}$ as a $5$-adic integer (find the first few $5$-adic
digits, and prove that you can keep going without quoting Hensel's lemma), and prove that
you cannot represent $\sqrt{6}$ as a $7$-adic integer.

\item (10 points) Starting from the completion definition of $\Q_p$ (Cauchy sequences mod 
Cauchy sequences converging to zero), prove the following properties, less sketchily than was done
in lecture:
\begin{itemize}
\item $\Q_p$ is a field.
\item $\Z_p$ is a ring, and $(p)$ is the unique maximal ideal.
\item $\Q_p$ and $\Z_p$ possess an absolute value which agrees with the $p$-adic absolute value
on $\Q$ and $\Z$, and are complete with respect to this absolute value.
\end{itemize}

\item (5 points)
Prove that addition or multiplication by any fixed element of $\Q_p$ is (topologically) a homeomorphism from $\Q_p$
to itself.

{\itshape If you want to study valuations in general, the adeles, Tate's thesis, etc., please be sure to
do this exercise. (Or just convince yourself it's ``obvious''.)}

\item (7 points)
$\Z_p$ is {\itshape compact}.

\end{enumerate}

\end{document}
