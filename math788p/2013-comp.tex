
\documentclass[12pt]{amsart}
\usepackage{amssymb,amsmath,amsthm}
\usepackage{enumerate}
\usepackage{comment}
\usepackage{ifthen, url}
\newcommand{\textmod}{{\text {\rm mod}}}
\newcommand{\textMod}{{\text {\rm Mod}}}


\oddsidemargin = -1cm \evensidemargin = -1cm \textwidth = 6.8in
\textheight =8.5in
\topmargin =0.0in

\title{Comprehensive Exam in Algebraic Number Theory (Fall 2013)}
\begin{document}           % End of preamble and beginning of text.
\maketitle

Recall that the {\bf Minkowski bound} is
\[
N(\mathfrak{a}) \leq \frac{n!}{n^n} \bigg(\frac{4}{\pi}\bigg)^s |\Delta_K|^{1/2}.
\]

\begin{enumerate}[1.]
\item
Let $K$ be the cubic field generated by a root of $x^3 + 2x - 1$. Determine the
following data associated to $K$:
\begin{enumerate}[(a)]
\item its {\bf discriminant};
\item the {\bf ring of integers};
\item the number of {\bf real and complex embeddings};
\item the isomorphism class of its {\bf unit group} (your answer should look like,
e.g., $\mathbb{Z}^2 \times \mathbb{Z}/(6)$; you do not need to find the actual units);
\item the list of {\bf ramified primes};
\item the splitting types of the ideals $(2), (3), (5), (7)$;
\item whether or not $K$ is {\bf Galois} over $\mathbb{Q}$; and if not, the Galois group
of its Galois closure (i.e., the splitting field of $x^3 + 2x - 1$);
\item (bonus) the {\bf the class group};
\item (bonus) the {\bf proportion of primes} which have the splitting types you found above.
\end{enumerate}
\vskip 0.5cm
\item
\begin{enumerate}[(a)]
\item Define the $p$-adic integers $\mathbb{Z}_p$.
\item (bonus) Give an alternative definition of the $p$-adic integers $\mathbb{Z}_p$.
(further bonus) Prove that these definitions are equivalent.
\item Let $p = 7$. Determine which of $\frac{1}{3}$, $7$, $\frac{1}{7}$, $\sqrt{2}$,
and $\sqrt{5}$ are $7$-adic integers. For those that are, compute the $7$-adic expansion
to at least three decimal places. 

Note that one of the two square roots is in $\mathbb{Z}_7$; give a detailed proof of this, without
quoting Hensel's lemma. For the others, a very brief explanation is enough.
\end{enumerate}

\vskip 0.5cm
\item

\begin{enumerate}[(a)]
\item Determine, with proof, the class group of $\mathbb{Q}(\sqrt{-5})$.
\item Is $\mathbb{Q}(\sqrt{100000001})$ a principal ideal domain?
\end{enumerate}
\vskip 0.5cm
\item
\begin{enumerate}[(a)]
\item
Determine the discriminant of a quadratic field $\mathbb{Q}(\sqrt{D})$, for a general $D$.
\item
Let $\ell$ be an odd prime. Then it is known that the discriminant of the {\bf cyclotomic field}
$K = Q(\zeta_{\ell})$ is equal to $\pm \ell^{\ell - 2}$, and also that $(\ell) = (1 - \zeta_{\ell})^{\ell - 1}$
as ideals of $\mathcal{O}_K$.
What fact about the splitting of prime ideals in $\mathcal{O}_K$ is reflected in both of these facts?
\item
Prove that $\mathbb{Q}(\zeta_{\ell})$ contains a unique quadratic subfield, and determine what it is.
\item
(bonus) Subject to knowing the field is unique, determine what it is in a completely different way.
\end{enumerate}
\end{enumerate}

\end{document}
