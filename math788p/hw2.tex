\documentclass[12pt]{article}
\usepackage{amssymb,amsmath,amsthm}
\usepackage{enumerate, verbatim}
\textwidth 7.0in
\oddsidemargin -0.4in
\evensidemargin -0.4in
\textheight 9.0in 
\pagestyle{empty}
\newcommand{\p}{\mathfrak{p}}
\newcommand{\Q}{\mathbb{Q}}
\newcommand{\Z}{\mathbb{Z}}
\begin{document}
% 8.5in paper width -2x1in margin = 6.5in text width
\setlength{\topmargin}{-2mm}



% 11in paper height -2x1in margin = 9in text height


\begin{center}{\bf Algebraic number theory (Spring 2013), Homework 1}
\end{center}
\begin{center}Frank Thorne, thornef@webmail.sc.edu
\end{center}
\begin{center}
{\bf Due Friday, February 1}
\end{center}

Please do at least one of the cubic field integral basis calculations, if you do nothing else.
Questions 2 and 3 can be skipped without loss of continuity, but are great questions for prospective
algebraists or algebraic geometers.

\begin{enumerate}
\item (3 points)
Is $\frac{3 + 2 \sqrt{6}}{1 - \sqrt{6}}$ an algebraic integer?

\item (5 points)
If the integral domain $A$ is integrally closed, then so is the polynomial ring $A[t]$.

\item (5 points)
In the polynomial ring $A = \mathbb{Q}[X, Y]$, consider the principal ideal $\mathfrak{p} = (X^2 - Y^3)$.
Show that $\mathfrak{p}$ is a prime ideal, but $A/\p$ is not integrally closed.

\item (8 points)
Find the discrimimant of, and an integral basis for, $\Q(2^{1/3})$.

\item (10 points)
Find the discriminant of, and an integral basis for, $\Q(\theta)$ where $\theta^3 - \theta - 4 = 0$.

\item (10 points)
Find the discriminant of, and an integral basis for, $\Q(\theta)$ where $\theta^7 - 3 = 0$.

\item (*5 points)
Prove that the discriminant of an algebraic number field is always congruent to 0 or 1 modulo 4.

Hint: the determinant of an integral basis is a sum of terms, each prefixed by a positive or a 
negative sign. Writing $P$ and $N$ for the positive and negative terms respectively, the discriminant
is $(P - N)^2 = (P + N)^2 - 4PN$. The solution is most naturally explained using a little Galois theory.

\item (5 points)
In the ring $\Z[\sqrt{-5}]$, verify that
$$(2) = (2, 1 + \sqrt{-5})(2, 1 - \sqrt{-5})$$
$$(3) = (3, 1 + \sqrt{-5})(3, 1 - \sqrt{-5})$$
$$(1 + \sqrt{-5}) = (2, 1 + \sqrt{-5})(3, 1 + \sqrt{-5})$$
$$(1 - \sqrt{-5}) = (2, 1 - \sqrt{-5})(3, 1 - \sqrt{-5}).$$
In each of the four factorizations above, are the two ideals on the right different from each other?

\item
(5 points) Verify that the equation $a^2 + 5b^2 = 2$ does not have any solutions over $\Q$.

\end{enumerate}

\end{document}
