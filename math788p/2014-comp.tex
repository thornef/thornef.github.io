
\documentclass[12pt]{amsart}
\usepackage{amssymb,amsmath,amsthm}
\usepackage{enumerate}
\usepackage{comment}
\usepackage{ifthen, url}
\newcommand{\textmod}{{\text {\rm mod}}}
\newcommand{\textMod}{{\text {\rm Mod}}}
\newcommand{\Disc}{{\text {\rm Disc}}}
\newcommand{\Gal}{{\text {\rm Gal}}}
\newcommand{\calO}{\mathcal{O}}
\newcommand{\p}{\mathfrak{p}}

\oddsidemargin = -1cm \evensidemargin = -1cm \textwidth = 6.8in
\textheight =8.5in
\topmargin =0.0in

\title{Comprehensive Exam in Algebraic Number Theory (Fall 2014)}
\begin{document}           % End of preamble and beginning of text.
\maketitle

Recall that the {\bf Minkowski bound} is
\[
N(\mathfrak{a}) \leq \frac{n!}{n^n} \bigg(\frac{4}{\pi}\bigg)^s |\Delta_K|^{1/2}.
\]

Please choose either the (E) (elementary) or (A) algebraic questions. You don't need
to do both (although you can!)

\begin{enumerate}[1.]
\item
Let $K$ be the cubic field generated by a root of $x^3 + x^2 + 1$. Determine the
following data associated to $K$:
\begin{enumerate}[(a)]
\item its {\bf discriminant};
\item the {\bf ring of integers};
\item the number of {\bf real and complex embeddings};
\item the isomorphism class of its {\bf unit group} (your answer should look like,
e.g., $\mathbb{Z}^2 \times \mathbb{Z}/(6)$; you do not need to find the actual units);
\item the list of {\bf ramified primes};
\item the splitting types of the ideals $(2), (3), (5), (7)$;
\item whether or not $K$ is {\bf Galois} over $\mathbb{Q}$; and if not, the Galois group
of its Galois closure (i.e., the splitting field of $x^3 + 2x - 1$);
\item (bonus: if you can) the {\bf the class group};
\item (bonus: if you can) the {\bf proportion of primes} which have the splitting types you found above.
\end{enumerate}
\vskip 0.5cm
\item
Let $K$ be the quartic field generated by a root of $f(x) := x^4 - x^3 - 2x^2 - 3x - 2$,
and let $L$ be its Galois closure. The aim of this exercise is to give a (partial) proof
that its Galois group is $S_4$.
\begin{enumerate}[(a)]
\item
The discriminant of the ring generated by a root $\alpha$ of $f(x)$ is $-3751$. 
Give a sketch of how you would prove this.
\item
What are the possible values for $\Disc(K)$? (Bonus: actually determine $\Disc(K)$.)
\item
State an appropriate theorem which identifies an element of $\Gal(L/\mathbb{Q})$ 
in terms of the splitting of a prime $p$ in $\mathcal{O}_K$.
\item
By proving that $\Gal(L/\mathbb{Q})$ contains `enough' elements, prove that it is all of
$S_4$. If this becomes tedious, begin a proof and explain in detail how you would finish.

Be sure not to assume that $\mathcal{O}_K = \mathbb{Z}[\alpha]$ unless you
proved it previously.
\item
If $f(x)$ instead generated a $D_4$ extension, it would be much more difficult to prove
this. Explain why.
\end{enumerate}

\vskip 0.5cm

\item
\begin{enumerate}[(a)]
\item Define the $p$-adic integers $\mathbb{Z}_p$ (in any way you choose). Determine,
with proof, the maximal ideal $\mathfrak{m}$ and the residue class field
$\mathbb{Z}_p / \mathfrak{m}$.
\item Write out $5$-adic expansions for $\frac{1}{4}$, $7$, and $\sqrt{6}$ in $\mathbb{Q}_5$.
(First three digits for $\sqrt{6}$.)

\item (E) It is known that any element of $\mathbb{Q}$ has a repeating decimal expansion when
written in $\mathbb{R}$. Determine, with a proof or counterexamples, whether this result
extends in a natural way to the $p$-adics.

\item Suppose you defined the $10$-adic integers in an analogous way to the $p$-adics.
State your definition precisely, and then prove that you don't obtain an integral domain.

\item (A) Like $Z$, $\mathbb{C}[x]$ is also a principal ideal domain. Define what should 
be the analogue of $\mathbb{Z}_p$ for $\mathbb{C}[x]$, and describe the rings you obtain
by your construction.

\end{enumerate}

\vskip 0.5cm
\item
Compute the ring of integers and class group of $\mathbb{Q}(\sqrt{39})$.

\vskip 0.5cm
\item
(E) Let $K$ be the number field generated by a root of $x^5 + 2x + 10$. It turns out that
its ring of integers is generated by a root of this polynomial. Given this, compute $\Disc(K)$.

\vskip 0.5cm
\item
(A) Let $L/K$ be an extension of number fields, let $\calO_L$ and $\calO_K$ be their rings
of integers, and let $\p$ be a prime of $\calO_K$. We are interested
in proving that $\calO_L / \p \calO_L$ is an $\calO_K/\p$-vector space of
dimension $n := [L : K]$. (This is an important step in the $efg$ theorem.)

\begin{enumerate}[(a)]
\item
Can we write the following, for some $\alpha_i \in \calO_K$?
\begin{equation}\label{eq:assume}
\calO_L = \calO_K \alpha_1 \oplus \cdots \oplus \calO_K \alpha_n
\end{equation}
If not in general, by quoting a relevant theorem 
give an example of a $K$ for which we can do this. 
\item
Assume that \eqref{eq:assume} holds. Prove that the images of the
$\alpha_i$ under the natural reduction map modulo $\p$ form a basis
for $\calO_L / \p \calO_L$ as an $\calO_K/\p$-vector space, thereby
obtaining the result.

You may (and almost certainly will) give a proof that requires an
condition on $\calO_K$. This is allowed, if you state your condition
explicitly and if it does not restrict your proof to only $K = \mathbb{Q}$.
\end{enumerate}

\end{enumerate}
\end{document}
