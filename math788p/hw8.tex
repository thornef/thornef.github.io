\documentclass[12pt]{article}
\usepackage{amssymb,amsmath,amsthm}
\usepackage{enumerate, verbatim, url}
\textwidth 7.0in
\oddsidemargin -0.4in
\evensidemargin -0.4in
\textheight 9.0in 
\pagestyle{empty}
\newcommand{\p}{\mathfrak{p}}
\newcommand{\Q}{\mathbb{Q}}
\newcommand{\Z}{\mathbb{Z}}
\begin{document}
% 8.5in paper width -2x1in margin = 6.5in text width
\setlength{\topmargin}{-2mm}



% 11in paper height -2x1in margin = 9in text height


\begin{center}{\bf Algebraic number theory (Spring 2013), Homework 8}
\end{center}
\begin{center}Frank Thorne, thorne@math.sc.edu
\end{center}
\begin{center}
{\bf Due Wednesday, May 8}
\end{center}

{\itshape Note: Any homework turned in after Thursday, May 2 at noon should be sent to me by e-mail,
so I can grade it from the road.}

\begin{enumerate}
\item (10 points)
Summarize what you learned in Robert Lemke Oliver's talk.

\item (10 points)
Robert's paper is available here:
\begin{center}
\url{http://www.mathcs.emory.edu/~rlemkeo/papers/08MultiplicativeArtin.pdf}
\end{center}
Elaborate on his Example 1, give more thorough justification of the algebraic number theory claims being made.

\item (15 points)
Do this same for Example 3.

\end{enumerate}

\end{document}
