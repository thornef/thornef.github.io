\documentclass[12pt]{article}
\usepackage{amssymb,amsmath,amsthm}
\usepackage{enumerate, verbatim, url}
\textwidth 7.0in
\oddsidemargin -0.4in
\evensidemargin -0.4in
\textheight 9.0in 
\pagestyle{empty}
\newcommand{\p}{\mathfrak{p}}
\newcommand{\Q}{\mathbb{Q}}
\newcommand{\Z}{\mathbb{Z}}
\begin{document}
% 8.5in paper width -2x1in margin = 6.5in text width
\setlength{\topmargin}{-2mm}



% 11in paper height -2x1in margin = 9in text height


\begin{center}{\bf Algebraic number theory (Spring 2013), Homework 5}
\end{center}
\begin{center}Frank Thorne, thornef@webmail.sc.edu
\end{center}
\begin{center}
{\bf Due Friday, March 22}
\end{center}
Recall that starred (*) exercises may involve background beyond what is assumed in this course.

\begin{enumerate}
\item (20 points)
In a page or so, 
explain what you learned, and/or some related topics you would like to learn better, from John Voight's lecture.

\item (5+ points)
As we proved, the Minkowski bound gives a lower bound for the discriminant of a number field $K$ in terms
of the degree, and also the number of complex embeddings.

For each $n \leq 6$, write out (as a decimal) the Minkowski lower bound for the discriminant of a number field
$K$, and look up (using the Jones-Roberts tables, or otherwise) the smallest discriminant of any number field with that
degree. Compare the data. If you have any interesting observations about number fields of small discriminant,
be sure to share them.

\item (5 points)
Compute the class group of $\Q(\sqrt{-33})$.

\item (5 points)
Compute the class group of $\Q(\sqrt{-163})$.

\item (5 points)
Compute the class group of $\Q(\sqrt{-14})$.

\item (5 points)
Compute the class group of $\Q(7^{1/3})$.

\item (8 points)
Compute the class group of $\Q(\sqrt{\alpha})$, where $\alpha^3 - \alpha - 7 = 0$.

\item (8 points)
Compute the class group of $\Q(\zeta_{11})$.

\item (20 points)
Prove that the class number of $\Q(\sqrt{-13947137572})$ is at least 100.

{\itshape Hint.
Use Dirichlet's class number formula, in combination with a computer and some analytic number theory. 
But just asking PARI or SAGE what the class number is (it's 17852) is cheating.}

\end{enumerate}

\end{document}
