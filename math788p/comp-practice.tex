
\documentclass[12pt]{amsart}
\usepackage{amssymb,amsmath,amsthm}
\usepackage{enumerate}
\usepackage{comment}
\usepackage{ifthen, url}
\newcommand{\textmod}{{\text {\rm mod}}}
\newcommand{\textMod}{{\text {\rm Mod}}}


\oddsidemargin = -1cm \evensidemargin = -1cm \textwidth = 6.8in
\textheight =8.5in
\topmargin =0.0in

\title{Comprehensive Exam in Algebraic Number Theory (Fall 2013)}
\begin{document}           % End of preamble and beginning of text.
\maketitle

Recall that the {\bf Minkowski bound} is
\[
N(\mathfrak{a}) \leq \frac{n!}{n^n} \bigg(\frac{4}{\pi}\bigg)^s |\Delta_K|^{1/2}.
\]

\begin{enumerate}[1.]
\item
Let $K$ be the cubic field generated by a root of $x^3 + x - 4$. Determine the
following data associated to $K$:
\begin{enumerate}[(a)]
\item its {\bf discriminant};
\item the {\bf ring of integers};
\item the number of {\bf real and complex embeddings};
\item the isomorphism class of its {\bf unit group} (your answer should look like,
e.g., $\mathbb{Z}^2 \times \mathbb{Z}/(6)$; you do not need to find the actual units);
\item the list of {\bf ramified primes};
\item the splitting types of the ideals $(2), (3), (5), (7)$;
\item whether or not $K$ is {\bf Galois} over $\mathbb{Q}$; and if not, the Galois group
of its Galois closure (i.e., the splitting field of $x^3 + x - 4$);
\item the {\bf the class group};
\item the {\bf proportion of primes} which have the splitting types you found above.
\end{enumerate}
\vskip 0.5cm
\item
\begin{enumerate}[(a)]
\item Give two definitions of the $p$-adic integers $\mathbb{Z}_p$. One should be analytic
(in terms of Cauchy sequences) and another should be algebraic (an inverse limit). Prove their equivalence.
Define also $\mathbb{Q}_p$.
\item Determine (with proof) the maximal ideal $\mathfrak{m}$ of $\mathbb{Z}_p$, as well as the 
residue field $\mathbb{Z}_p / \mathfrak{m}$.
\item Let $p = 7$. Determine which of $\frac{1}{3}$, $7$, $\frac{1}{7}$, $\sqrt{2}$,
and $\sqrt{5}$ are $7$-adic integers. For those that are, compute the $7$-adic expansion
to at least three decimal places. 

Note that one of the two square roots is in $\mathbb{Z}_7$; give a detailed proof of this, without
quoting Hensel's lemma. For the others, a very brief explanation is enough.
\end{enumerate}
\vskip 0.5cm
\item
Prove that no two $p$-adic fields $\mathbb{Q}_p$ are isomorphic to each other, nor to $\mathbb{R}$.
(Bonus: Prove that no finite extension of $\mathbb{Q}_p$ is isomorphic to any finite extension of
$\mathbb{Q}_{p'}$ for $p \neq p'$.

\vskip 0.5cm
\item
What is Hensel's lemma? State it, give an example of its use, and give a proof of it in a special case
of your choosing.

\vskip 0.5cm
\item
Does $x^2 + y^2 + 7z^2 = 0$ have any rational solutions? Prove or disprove. What about
$x^2 + y^2 + 7z^2 = 1$?

\vskip 0.5cm
\item
Write down a bunch of quadratic fields at random and compute their class groups.

\vskip 0.5cm
\item
Let $K$ be a quintic field, whose Galois closure is of degree $10$ over $\mathbb{Q}$ and
has Galois group $D_5$.

For (ordinary) primes $p$, determine all possibilities for how $p \mathcal{O}_K$ can decompose
into prime ideals of $\mathcal{O}_K$.

\vskip 0.5cm
\item
Let $\mathfrak{a}$ be an ideal of $\mathcal{O}_K$ for some $K$. Prove directly 
that $\mathfrak{a}$ contains an integer other than $0$, and then explain the relationship of
this fact to the norm of $\mathfrak{a}$.

\vskip 0.5cm
\item
Suppose that $\mathcal{O}$ an order in a quadratic field $K$, i.e. $\mathcal{O}$ is a subring
of $K$, containing $1$, finitely generated as a $\mathbb{Z}$-module, and containing a
$\mathbb{Q}$-basis of $K$.

Prove that $\mathcal{O}$ is contained in the ring of integers $\mathcal{O}_K$.

\vskip 0.5cm
\item
Let $K$ be a field for which $\mathcal{O}_K = \mathbb{Z}[\alpha]$ for some $\alpha$. Let
$f(x)$ be the minimal polynomial of $\alpha$.
\begin{enumerate}[(a)]
\item
Explain why $f(x)$ is monic.
\item
Explain why a prime $p$ ramifies in $K$ if and only if $f(x)$ has a multiple root modulo $p$.
\item
Prove the classical fact (in this special case)
that a prime $p$ divides the discriminant of $K$ if and only if it ramifies in $K$.
\end{enumerate}

\vskip 0.5cm
\item
Let $K$ be a quadratic field, and $p$ a prime. Compute the tensor product $K \otimes_{\mathbb{Q}} \mathbb{Q}_p$.
(There are three cases....)
\end{enumerate}

\end{document}
