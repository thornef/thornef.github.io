\documentclass[12pt]{article}
\usepackage{amssymb,amsmath,amsthm}
\usepackage{enumerate, verbatim, url}
\textwidth 7.0in
\oddsidemargin -0.4in
\evensidemargin -0.4in
\textheight 9.0in 
\pagestyle{empty}
\newcommand{\p}{\mathfrak{p}}
\newcommand{\Q}{\mathbb{Q}}
\newcommand{\Z}{\mathbb{Z}}
\begin{document}
% 8.5in paper width -2x1in margin = 6.5in text width
\setlength{\topmargin}{-2mm}



% 11in paper height -2x1in margin = 9in text height


\begin{center}{\bf Algebraic number theory (Spring 2013), Homework 6}
\end{center}
\begin{center}Frank Thorne, thorne@math.sc.edu
\end{center}
\begin{center}
{\bf Due Friday, April 19}
\end{center}
Recall that starred (*) exercises may involve background beyond what is assumed in this course.

\begin{enumerate}
\item (5 points)
Prove that any finite subgroup of the unit circle is cyclic.

\item (5 points)
Determine the unit group $\mathcal{O}_K^{\times}$ for all imaginary quadratic fields
$K = \Q(\sqrt{-D})$.

\item (7 points)
Determine the fundamental unit in $\mathbb{Q}(\sqrt{D})$ for each positive $D \leq 10$.

\item (5 points)
Compute the Galois group of $\Q(\zeta_8)$, and find all fields intermediate between
$\Q$ and $\Q(\zeta_8)$.

\item (5 points)
Compute the Galois group of $\Q(\zeta_{11})$, and find all fields intermediate between
$\Q$ and $\Q(\zeta_{11})$.

\item (3 points)
Describe explicitly the splitting of all primes in the extension $\Q(\zeta_{11})$.

\item (12 points)
Let $f(x)$ be a cubic polynomial such that the splitting field $K$ of $f(x)$ over $\Q$
has degree $6$.

Enumerate all possible splitting types of primes in $K$. There will {\itshape not} be that
many. You should find examples for all splitting types you claim can occur, and prove that
other splitting types can't occur.

\item (20 points)
Let $m$ and $n$ be distinct squarefree integers not equal to $1$.

\begin{enumerate}[(a)]
\item
Prove that $\Q(\sqrt{m}, \sqrt{n})$ is Galois over $\Q$ with Galois group $\Z/2 \times \Z/2$.
Does this field have any quadratic subfields other than $\Q(\sqrt{m})$ and $\Q(\sqrt{n})$? 
\item
Suppose $p$ ramifies in both $\Q(\sqrt{m})$ and $\Q(\sqrt{n})$. What happens in $K$? Find an
example.
\item
Suppose $p$ splits in both of these fields. What happens in $K$? Find an example.
\item
Suppose $p$ is inert in both of these fields. What happens in $K$? Find an example.
\item
Suppose the splitting behavior of $p$ is different in both of these fields. What happens in $K$?
Find an example.
\item ({\itshape David Zureick-Brown's favorite algebraic number theory problem}) Does there exist
an irreducible polynomial which is reducible mod every prime?
\end{enumerate}

\end{enumerate}

\end{document}
