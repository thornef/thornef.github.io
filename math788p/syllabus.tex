\documentclass[11pt]{amsart}
\usepackage{amssymb,amsmath,amsthm}
\usepackage[mathscr]{euscript}
\usepackage{enumerate, verbatim, url}
\oddsidemargin = 0.0cm \evensidemargin = 0.0cm \textwidth = 6.5in
\textheight =8.5in
\newtheorem{case}{Case}
\newcommand{\caseif}{\textnormal{if }}
\newcommand{\leg}[2]{\genfrac{(}{)}{}{}{#1}{#2}}
\newcommand{\bfrac}[2]{\genfrac{}{}{}{0}{#1}{#2}}
\newcommand{\sm}[4]{\left(\begin{smallmatrix}#1&#2\\ #3&#4 \end{smallmatrix} \right)}
\newtheorem{theorem}{Theorem}
\newtheorem{lemma}[theorem]{Lemma}
\newtheorem{corollary}[theorem]{Corollary}
\newtheorem{conjecture}{\bf Conjecture}
\newtheorem{proposition}[theorem]{Proposition}
\newtheorem{definition}[theorem]{Definition}
\renewcommand{\theequation}{\thesection.\arabic{equation}}
\renewcommand{\thetheorem}{\thesection.\arabic{theorem}}
\theoremstyle{remark}
\newtheorem*{theoremno}{{\bf Theorem}}
\newtheorem*{remark}{Remark}
\newtheorem*{example}{Example}
\numberwithin{theorem}{section} \numberwithin{equation}{section}

\newcommand{\ord}{\text {\rm ord}}
\newcommand{\li}{\text {\rm li}}
\newcommand{\putin}{\text {\rm \bf ***PUT IN***}}
\newcommand{\st}{\text {\rm \bf ***}}
\newcommand{\mfG}{\mathfrak{G}}
\newcommand{\mfF}{\mathfrak{F}}
\newcommand{\pr}{\text {\rm pr}}
\newcommand{\sfpart}{\text {\rm sf}}
\newcommand{\calM}{\mathcal{M}}
\newcommand{\calW}{\mathcal{W}}
\newcommand{\calC}{\mathcal{C}}
\newcommand{\calO}{\mathcal{O}}
\newcommand{\GG}{\mathcal{G}}
\newcommand{\FF}{\mathcal{F}}
\newcommand{\QQ}{\mathcal{Q}}
\newcommand{\Mp}{\text {\rm Mp}}
\newcommand{\frakG}{\mathfrak{G}}
\newcommand{\Qmd}{\mathcal{Q}_{m,d}}
\newcommand{\la}{\lambda}
\newcommand{\R}{\mathbb{R}}
\newcommand{\C}{\mathbb{C}}
\newcommand{\F}{\mathbb{F}}
\newcommand{\Cl}{{\text {\rm Cl}}}
\newcommand{\Tr}{{\text {\rm Tr}}}
\newcommand{\rk}{{\text {\rm rk}}}
\newcommand{\Q}{\mathbb{Q}}
\newcommand{\Qd}{\mathcal{Q}_d}
\newcommand{\Z}{\mathbb{Z}}
\newcommand{\N}{\mathbb{N}}
\newcommand{\SL}{{\text {\rm SL}}}
\newcommand{\GL}{{\text {\rm GL}}}
\newcommand{\textmod}{{\text {\rm mod}}}
\newcommand{\textMod}{{\text {\rm Mod}}}
\newcommand{\sgn}{\operatorname{sgn}}
\newcommand{\calP}{\mathcal{P}}
\newcommand{\PSL}{{\text {\rm PSL}}}
\newcommand{\lcm}{{\text {\rm lcm}}}
\newcommand{\Gal}{{\text {\rm Gal}}}
\newcommand{\add}{{\text {\rm add}}}
\newcommand{\sub}{{\text {\rm sub}}}
\newcommand{\Sym}{{\text {\rm Sym}}}
\newcommand{\End}{{\text {\rm End}}}
\newcommand{\Frob}{{\text {\rm Frob}}}
\newcommand{\Disc}{{\text {\rm Disc}}}
\newcommand{\Stab}{{\text {\rm Stab}}}
\newcommand{\Op}{\mathcal{O}_K}
\newcommand{\h}{\mathfrak{h}}
\newcommand{\G}{\Gamma}
\newcommand{\g}{\gamma}
\newcommand{\zaz}{\Z / a\Z}
\newcommand{\znz}{\Z / n\Z}
\newcommand{\ve}{\varepsilon}
\newcommand{\Div}{{\text {\rm Div}}}
\newcommand{\tr}{{\text {\rm tr}}}
\newcommand{\odd}{{\text {\rm odd}}}
\newcommand{\bk}{B_k}
\newcommand{\rr}{R_r}
\newcommand{\sump}{\sideset{}{'}\sum}
\newcommand{\gkr}{\mathfrak{g}_{k,r}}
\newcommand{\re}{\textnormal{Re}}
\newcommand{\im}{\textnormal{Im}}
\newcommand{\Res}{\textnormal{Res}}
\newcommand{\calS}{\mathcal{S}}
\newcommand{\Aut}{\textnormal{Aut}}
\def\H{\mathbb{H}}



\begin{document}
\title[Topics in algebraic number theory]
{Topics in algebraic number theory, Spring 2013}
\author{Frank Thorne}
\address{Department of Mathematics, University of South Carolina,
1523 Greene Street, Columbia, SC 29208}
\email{fthorne@math.stanford.edu}

\maketitle
LeConte ???, MWF 12:20-1:10. Office hours TBA.
\\
\\
{\bf Learning outcomes.} The student will master a variety of tools used in modern algebraic number theory, and will also
learn their connection to classical number theoretic questions concerning the distribution of primes and the solutions of
Diophantine equations.
\\
\\
{\bf References.} Useful source materials include Neukirch, {\itshape Algebraic Number Theory}; Marcus, {\itshape Number Fields};
Milne, {\itshape Algebraic Number Theory} (available for free online); and my own notes which I will publish online. Additional
references will be suggested throughout the semester.
\\
\\
{\bf Instructional Delivery Strategy.} Lecture.
\\
\\
{\bf Grading.} Grading will be on an open points scale. 200 points will be necessary for an A, 140 for a B, 120 for C, 100 for D.
Assignments will include a variety of homework sets (worth at least 200 points total), 
an optional computational project (worth 50 points), and an open-book, take-home final exam (unlimited time, but students should work
by themselves; worth 50 points). As the total value of the assignments is more than 200 points, 
students will be able to exercise some choice as to which projects
best meet their research and learning interests.
 
Graduate students who have been admitted to Ph.D. candidacy may be offered a reduced workload if negotiated with the instructor.
\\
\\
{\bf Prerequisites.} Graduate standing in the math department or permission of instructor. 

It is expected that students will have taken 703/704 (algebra) or its equivalent elsewhere, or be concurrently enrolled in it.
Math 784 (Filaseta's algebraic number theory) is strongly recommended as a prerequisite, but the course will be accessible (with
extra effort) to students who didn't take this. Any additional background in abstract algebra and/or number theory will be helpful.
\\
\\
{\bf List of topics.} The following is a rough outline of some of the topics we are likely to cover. We are likely not to have time for all of them.
\begin{enumerate}
\item {\bf The $p$-adic numbers.} (4 weeks.) 
Introduced by Hensel, the $p$-adic numbers allow us to look ``one prime at a time'' and are ubiquitous in
modern algebraic number theory. We will carefully look at their algebraic and topological properties.

We will also prove the {\itshape Hasse-Minkowski theorem}, the first example of a {\itshape local-to-global} principle for the solution
of Diophantine equations, which will help motivate the study of the $p$-adics.

\item {\bf The geometry of numbers.} (4 weeks.) One way of understanding number fields is to regard their rings of integers as a lattice. We will
carry out this construction and learn what insight it gives us. We will study the {\bf discriminants} and {\bf class groups} 
of number fields, and prove classical bounds on these quantities. We will also see (in principle, at least) how to compute the class group
of any number field.

\item {\bf The decomposition of prime ideals in extensions.} (4 weeks.) A classical topic in algebraic number theory, due to Hilbert,
is how prime ideals split in extensions. They can `split', be `inert', or `ramify'. We will learn what this means and see what it has
to do with classical questions. For example, a classical result (probably proved in Filaseta's course) gives a concise description of
the integers which can be written as a sum of two squares. We will see how this can be interpreted in terms of Hilbert's theory.

\item {\bf Valuations and their extensions.} {\itshape If there is time and interest}, we will investigate the theory of valuations,
or (mostly) equivalently, the theory of {\itshape extensions} of the $p$-adic numbers. As we will see, this theory complements Hilbert's,
and allows us to study number fields globally by looking at their local properties.

\item {\bf Tate's thesis.} {\itshape If there is time and interest}, we will look at John Tate's visionary 1951 Princeton thesis, which
combined algebraic number theory with analytic. Tate started with the $p$-adic numbers and combined all of them to form the {\itshape
ring of adeles}, and observed that this was an excellent space on which to do harmonic analysis. This led to a beautiful proof of
the functional equation of the Riemann zeta functions (and more generally, of Hecke $L$-functions). Later, it led to a huge variety
of generalizations (e.g. the Langlands program), many of them conjectural, and all of them very important and widely studied. 
We will see what it was that Tate did.

\item {\bf Introduction to class field theory.} (2 weeks.) The subject of {\bf class field theory} classifies the {\bf abelian extensions} of
a number field in terms of the {\bf class group}. Indeed, there is a beautiful isomorphism between the class group of a number
field $K$, and the Galois group of its maximal abelian unramified extension, called the Artin map. This subject is frequently considered
to be the crowning achievement of algebraic number theory. The proofs of class field theory
are notoriously difficult, and so we will skip them, but we will study the basic theorems and derive some elementary consequences.
\end{enumerate}

{\bf Attendance policy.} No penalty will be enforced for missed classes.

\end{document}


