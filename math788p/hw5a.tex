\documentclass[12pt]{article}
\usepackage{amssymb,amsmath,amsthm}
\usepackage{enumerate, verbatim, url}
\textwidth 7.0in
\oddsidemargin -0.4in
\evensidemargin -0.4in
\textheight 9.0in 
\pagestyle{empty}
\newcommand{\p}{\mathfrak{p}}
\newcommand{\Q}{\mathbb{Q}}
\newcommand{\Z}{\mathbb{Z}}
\begin{document}
% 8.5in paper width -2x1in margin = 6.5in text width
\setlength{\topmargin}{-2mm}



% 11in paper height -2x1in margin = 9in text height


\begin{center}{\bf Algebraic number theory (Spring 2013), Homework 5a}
\end{center}
\begin{center}Frank Thorne, thornef@webmail.sc.edu
\end{center}
\begin{center}
{\bf Due Friday, March 22}
\end{center}
(These exercises concern the Tuesday lectures on valuations.)

\begin{enumerate}
\item (5 points)
Prove, without quoting any theorems about valuations, that $|z|$ is the only valuation of $\mathbb{C}$ which extends
the usual one on $\mathbb{R}$.

\item (20 points)
Do Exercises 1 and 2 from Neukirch, p. 152. Feel free to assume that the `henselian field' is a finite extension of
$\Q_p$ and ignore any separability hypotheses (which are automatic in this case.

Conclude that for any given $p$ and $n$, there are only finitely many extensions of $\Q_p$ of degree $n$.
(You may find it interesting to look at the Jones-Roberts database for a list of such extensions.)

\item (5+ points)
Exhibit cubic extensions $K$ and $L$ of $\Q_5$ which are ramified and unramified, respectively.
(Optional: generalize.)

\item (10 points)
Exhibit two different cubic ramified extensions of $\Q_3$, referring to the Jones-Roberts database if you like.
(The work is in proving that the fields are not isomorphic.)

\end{enumerate}

\end{document}
