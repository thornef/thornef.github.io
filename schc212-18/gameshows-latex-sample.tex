\documentclass[12pt, leqno]{article}
\newenvironment{proof}{\noindent {\bf Proof:}}{$\Box$ \vspace{2 ex}}

\usepackage{amsmath, amssymb, amscd, hyperref}
\usepackage{latexsym, epsfig, color, url, relsize, soul}
\usepackage[all]{xy}
\usepackage{tikz, verbatim, xcolor, skull, sgame}\renewcommand\gamestretch{2}
\usetikzlibrary{calc,matrix}
\usepackage{ifthen,environ,etoolbox}
\newcommand{\urlc}[1]{\begin{center} \url{#1} \end{center}}
\newcommand{\newgame}[2]{ smallskip {\bf Game Description} (#1): #2 \smallskip }

\newcommand{\hs}\heartsuit
\newcommand{\cs}\clubsuit
\renewcommand{\ss}\spadesuit
\newcommand{\ds}\diamondsuit
\newtheorem{proposition}{Proposition}[section]
\newtheorem{theorem}[proposition]{Theorem}
\newtheorem{corollary}[proposition]{Corollary}
\newtheorem{example}[proposition]{Example}
\newtheorem{question}[proposition]{Question}
\newtheorem{lemma}[proposition]{Lemma}
\newtheorem{remark}[proposition]{Remark}
\newtheorem{defn}[proposition]{Definition}
\newtheorem{exm}[proposition]{Example}
\newtheorem{assmp}{Assumption}
\newtheorem{aim}{Aim}


\setlength{\topmargin}{-0.2in}
\setlength{\oddsidemargin}{-0.1in}
\setlength{\evensidemargin}{-0.1in}
\setlength{\textwidth}{6.5in}
\setlength{\textheight}{8.6in}

\newtheorem{definition}[proposition]{Definition}
\newtheorem{notation}[proposition]{Notation}

%%%%%%%%%%%%%%%%
% What you see below is the important part! You don't need to understand or change anything above this line.
%%%%%%%%%%%%%%%%

\title{Term Project Sample File}
     
\author{Frank Thorne}
       
\begin{document}
\maketitle



This is an example of a paper written using the software LaTeX. It is intended to illustrate what LaTeX is good for and how to use it.


\section{What is LaTeX?}

LaTeX (pronounced lay-tech, and often written \LaTeX), is typesetting software written and designed by Donald Knuth and Leslie
Lamport. It can be used to write documents of any sort, but it was designed for mathematics and computer science.

The biggest difference between LaTeX and programs like Microsoft Word is that Word is a `WYSIWYG' (What You See Is What You Get)
editor and LaTeX is not. In Word you edit your document directly, as it will eventually appear. When you use LaTeX you edit a .tex file,
and then typeset it to produce a nice looking PDF file. For example if you type this
\begin{verbatim}
\[
\frac{3}{5} \cdot x + \frac{2}{5} \cdot (1 - x) = \frac{2 + x}{5},
\]
\end{verbatim}
then it will come out looking like this.
\[
\frac{3}{5} \cdot x + \frac{2}{5} \cdot (1 - x) = \frac{2 + x}{5},
\]
Another feature of LaTeX is that it auto-numbers sections, definitions, examples, and so forth. For example:

\begin{definition}\label{foo}
\begin{enumerate}
\item
A {\bf sample space} is the set of all possible outcomes of a some process.
\item
An {\bf event} is any subset of the sample space.
\end{enumerate}
\end{definition}

\begin{example}\label{bar}
Blah, blah, blah....
\end{example}

\begin{theorem}\label{baz}
Mathematics is fun!
\end{theorem}
\begin{proof} Obvious. \end{proof}

Let us now look at the LaTeX code which we used to produce this.

\begin{verbatim}
\begin{definition}\label{foo}
\begin{enumerate}
\item
A {\bf sample space} is the set of all possible outcomes of a some process.
\item
An {\bf event} is any subset of the sample space.
\end{enumerate}
\end{definition}

\begin{example}\label{bar}
Blah, blah, blah....
\end{example}

\begin{theorem}\label{baz}
Mathematics is fun!
\end{theorem}
\begin{proof} Obvious. \end{proof}
\end{verbatim}
We did not have to worry about the numbers -- LaTeX figures out how to number them for us. The labels are optional -- we can 
leave out the part that says \begin{verbatim}\label{foo}\end{verbatim} and so on. But if we leave it in, then we can refer to these --
for example if we write

\begin{verbatim}
By Theorem \ref{foo}, we now know that...
\end{verbatim}

then it ends up looking like this:

\medskip

By Theorem \ref{foo}, we now know that...

\medskip
If we reorder everything, then it will get the numbers right automatically.

\section{How to get started}
Here is a suggested procedure for how to get started with using LaTeX.

\begin{itemize}
\item Download a program that can process LaTeX files (which will have a .tex extension) and can produce nice PDFs. This will
consist of the LaTeX software itself, and it will often have nice additional features like a built-in text editor, spell checking, and so on.

There are many programs you can use, and most of them can be downloaded legally for free. (Indeed, most of them are
open-source, so you can download the source code too if you like.)

For Macintosh, I personally use TeXShop which you can download here:
\urlc{http://pages.uoregon.edu/koch/texshop/}

I don't use Windows, so I don't personally know of a good LaTeX editor for that platform. You might try MiKTeX:
\urlc{http://www.miktex.org}

\item
Once you have downloaded something, I recommend that you start with this sample file and start by compiling it. Make sure that
you get output matching what's on the web page!

After that, I would try making small tweaks and seeing what happens. There is a bunch of crap at the beginning which is probably
confusing -- you don't need to worry about what it means.

Within this sample file, you should ignore the parts which are labeled `verbatim' (i.e., between begin verbatim and end verbatim). I used that
to show LaTeX code in the final document without having LaTeX process it. This won't be relevant to your final project.

\item
For some changes, depending on your LaTeX editor you may need to compile the code {\itshape twice} to get it right.

\item
Here are some further resources:

\urlc{http://math.ucr.edu/~huerta/latexforbeginners.html}

In addition, I have posted the original LaTeX code for the course notes on the website, so you can read those too.

\item
Finally, {\bf feel free to ask me for help!} If you have your work on a laptop, you are encouraged to bring it to class and ask me questions
beforehand and aferwards.

\end{itemize}

\subsection{A subsection}
There's nothing here, but I added this so you can see how to add subsections if you like.

\section{The bibliography}
I included a sample of a bibliography at the end of the document. Each bibliographic item has a name and you can cite it like this
\cite{thorne}. The name you put inside the {\itshape bibitem} is not important and is just for your reference -- you will cite it using the same name you give.

I do not care about bibliographic formatting and style. Please just make sure that the bibliography looks neat, and most importantly
that {\bf all} sources are credited. This includes web sources and any game show clips you watch. 
Identify clips as such, with the title on Youtube
(or wherever), the name of the game show (if not part of the title), and the
date that they were aired if possible. Some samples follow below.


\begin{thebibliography}{10000}

\bibitem{BC} W. Butterworth and P. Coe,
{\itshape Come On Down: The Price is Right in Your Classroom},
Primus, Volume XIV, Number I, March 2004, 12--28.

\bibitem{epp} S. Epp,
{\itshape Discrete Mathematics with Applications}, Fourth Edition, Brooks/Cole, 2011.

\bibitem{tpir1} {\itshape The Price Is Right: FULL EPISODE from 2014}. Video clip, March 13, 2014.
\url{https://www.youtube.com/watch?v=qOWfz7ZN6PE}.

\bibitem{tpir2} {\itshape The Price is Right, NEW GAME ``Rat Race''}. Video clip, date unknown.
\url{https://www.youtube.com/watch?v=zIKO94qbWBc}.
 
\bibitem{thorne} F. Thorne,
{\itshape The Mathematics of Game Shows}. \url{http://people.math.sc.edu/thornef/schc212/gameshows-1110.pdf}.

\end{thebibliography}

\end{document}