\documentclass[11pt]{article}
%%%%%%%%%%%%%%%%%%%%%%%%%%%%%%%%%%%%%%%%%%%%%%%%%%%%%%%%%%%%%%%%%%%%%%%%%%%%%%%%%%%%%%%%%%%%%%%%%%%%%%%%%%%%%%%%%%%%%%%%%%%%%%%%%%%%%%%%%%%%%%%%%%%%%%%%%%%%%%%%%%%%%%%%%%%%%%%%%%%%%%%%%%%%%%%%%%%%%%%%%%%%%%%%%%%%%%%%%%%%%%%%%%%%%%%%%%%%%%%%%%%%%%%%%%%%
\usepackage{amsfonts}
\usepackage{amssymb}
\usepackage{amsmath}
\usepackage{color,hyperref}
\usepackage{graphics}
\usepackage{times}
\usepackage{fancyhdr}
\usepackage{datetime}



\definecolor{darkblue}{RGB}{0,0,50}
\hypersetup{colorlinks,breaklinks,
            linkcolor=darkblue,urlcolor=darkblue,
            anchorcolor=darkblue,citecolor=darkblue}


%%%%%%%%%%%%%%%%%%date in format Sunday, October 25, 2011$$$
\usdate
\def\theday	{\dayofweekname{\day}{\month}{\year}}
\def\mydate	{\theday , \today}


%%%%%%%%%%%%opening%%%%%%%%%
\def\person		{So and So}
\def\role		{An important person}
\def\deptname		{Some Department}
\def\university		{Blank College or University}
\def\inlinename		{the Blank College or University}
\def\citystatezip	{City, State, zippy}
\def\addressing		{So and So}
\def\salutation		{Dear \addressing ,}

%%%%%%%%%%make your own macros here%%%%%%%%%%%%%%%%%



%%%%%%%%%%%%%%%%%%%%%%%%%%%%%%%%%%%%%%%%%%%%%%%%%%%








%%%%%%%%%%return address%%%%%%%%%%
\def\myname		{Frank Thorne}
\def\mydeptname		{Department of Mathematics}
\def\myaffiliation	{University of South Carolina}
\def\mystreet		{1523 Greene Street}
\def\mycitystatezip	{Columbia,\ SC\ \ 29208}
\def\myphone		{{\it Phone:} (803)404-4057 (home)}
\def\office		{(803)777-4224}
\def\fax		{{\it Fax:}(803)777-3783}
\def\email		{thorne@math.sc.edu}
\def\url		{http://www.math.sc.edu/$\sim$thornef} % NOTE: use $\sim$ for tilde
%%%%%%%%%%%%%%%%%%%%%%%%%%%%%%%%%%%%

%%%%%%%%%%%%%%%margins%%%%%%%%%%%%%%%%%%
\oddsidemargin	 0in
\evensidemargin  0in
\topmargin	-0.75in
\textwidth	 6.5in
\thispagestyle{fancy}
\textheight	 6.25in
\headwidth	\textwidth
\headheight 	1.25in
\headsep 	1in	%This controls how far below the letterhead the text of the letter will begin
\parindent      0ex	%controls how the indentation of new paragraphs behaves
\parskip		6pt	%controls the space between paragraphs



\addtolength{\textheight}{.5in}
%%%%%%%%%%%%%%%%letterhead%%%%%%%%%%%%%%%%
\fancyfoot{}
\renewcommand{\headrulewidth}{0pt}
\renewcommand{\footrulewidth}{0pt}

%%%usc logo
\fancyhead[L]{\resizebox{2.5in}{!}{\includegraphics{USC_Linear.jpg}} \vspace{0.6in}}

%%%%%name, line, date%%%%%%
%\fancyhead[C]{
%\hspace{1.5in}\raggedright\textbf{\LARGE \bf \myname}\vspace{-6pt}\\
%\hspace{1.5in}\raggedright\hrulefill \vspace{2.5pt}\\
%\hspace{1.5in}\raggedright\textbf{\normalfont\large \mydate} 
%%%%%%the below space moves the above content so the line matches up with line in logo
%\vspace{.149in}}

%%%%%%%%%%return address on right%%%%%%%
\fancyhead[R]{
\parbox[t]{3.1in}{ \footnotesize   \em \vspace{0.5in}
				\myname \\
				\mydeptname   \\
				\myaffiliation \\
				\mystreet \\
				\mycitystatezip \\
				\email \\
\\ 
				\mydate
				}\hspace{-1.3in}
\vspace{0.2in}}








\begin{document}
\pagestyle{fancy}

\enlargethispage{\baselineskip}

\vskip 10pt
Dear Colleagues,
\vskip 10pt

I would like to ask you to fund my graduate student {\bf Dan Kamenetsky} for travel to the Arizona Winter
School. 

Dan is in his third year in the Ph.D. program in South Carolina, and his research interests are a direct
match for this year's program. Dan is interested in the geometry of numbers, and he is currently reading
the paper {\itshape On the Davenport-Heilbronn theorem and second order terms} by Bhargava, Shankar, and Tsimerman
with an eye towards coming up with a {\itshape geometric} proof of the biases in arithmetic progressions
proved by Taniguchi and myself. This is intended as a starter project (I know how to do it myself, and I'm sure
Manjul, Arul, and Jacob all do as well), and it then opens the door to him to all sorts of interesting
follow-up projects. For example, there are likely to be similar biases in quartic and quintic fields, and he
could figure out what they are (if not actually prove the secondary terms) by counting points in
$(\mathbb{Z} / m \mathbb{Z})^d$ for fixed values of $d$. This line of work might also have interesting connections
to the ongoing work on Selmer groups and ranks of elliptic curves by Bhargava, Shankar, Ho, and others.)

Dan's background includes: a year of algebraic number theory (a fall semester taught by Michael Filaseta, emphasizing
an elementary viewpoint as well as a variety of applications, and a spring semester which I taught,
equivalent to a substantial portion of Chapters 1 and 2 of Neukirch); a semester of algebraic geometry
(Riemann surfaces, taught by Jesse Kass); and various other courses (including fairly substantial background
in analysis). He also self-taught himself analytic number theory: he didn't decide on his interests quickly
enough to take my course in Fall 2011, but last summer he learned the material well enough to pass a
comprehensive exam in the subject. (Students are required to pass six comprehensive exams of their choice, 
usually based on their coursework, before being admitted to Ph.D. candidacy in his second year.) This spring
he is taking a second semester of algebraic geometry, along with a geometry of numbers course which I will
be teaching.

His research interests are a clear match to this year's program, and attendance would be of great benefit to him.
Indeed, in the course of describing his research interests, I already named at least two people who will be
there.
I think the problem sessions led by Wei Ho or Arul Shankar would be especially good for him, but I am happy
to recommend him for any of the groups except Bjorn Poonen's (he does not have any background in schemes).

AWS will be a wonderful, eye-opening, inspiring experience for Dan. I ask you to please fund his travel and accommodation.

Thank you very much.
\vskip0.5in
\hspace{.5\textwidth}\parbox[t]{2.95in}{
		    Sincerely,\\
		   \includegraphics{mysignature.jpg}\\ 

		    Frank Thorne\\
		    Assistant Professor of Mathematics
		   }
%if you want to sign the document yourself, replace Sincerely,\\ with Sincerely,\vspace{0.5in} and delete the line \includegraphics{}

\vfill





\end{document}
